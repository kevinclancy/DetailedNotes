
\documentclass{article}

\usepackage[a4paper, total={7.2in,8in}]{geometry}

\usepackage{amscd}
\usepackage{graphicx}
\usepackage{amsmath}
\usepackage{amssymb}
\usepackage{amsthm}
\usepackage{stmaryrd}
\usepackage{mathpartir}
\usepackage{multicol}
\usepackage{enumitem}
\usepackage[section]{placeins} % float barriers
\usepackage{natbib}
\usepackage{xcolor} 
\usepackage{bussproofs} 
\usepackage{diagrams}
\usepackage{tikz}
\usepackage{quiver}

\usetikzlibrary{cd}

\tikzset{
absepi to/.tip={Glyph[glyph math command=triangleright]},
}

\newtheorem{lemma}{Lemma}

%example: \limit{j \in J}{F_j}
\newcommand{\limit}[2]{\underset{\overset{\longleftarrow}{#1}}{\text{lim}}~#2}
\newcommand{\lims}[1]{\underset{\longleftarrow}{\text{lim}}~#1}
\newcommand{\mbf}{\mathbf}
\newcommand{\absepi}{~\stackinset{r}{}{c}{}{\scalebox{0.5}{$\triangleright$}}{$-$}~}
\newcommand{\absmono}{\mapsto}

\newcommand{\inconsist}{\mathrel{\substack{\smile \\ \frown}}}
\newcommand{\consist}{\mathrel{\substack{\frown \\ \smile}}}

\newcommand{\vrt}[2]{
\pile{
#1 \\
\downarrow \\
#2
}
}

\newcommand{\ddisp}[3]{
\left(
\scriptsize
\begin{tikzcd}
#1 \ar[d, "\footnotesize{#2}"] \\
#3
\end{tikzcd}
\normalsize
\right)
}

\newcommand{\disp}[3]{
\left(
\tiny
\begin{array}{c}
#1 \\
\downarrow\\
#3
\end{array}
\begin{array}{l}
~ \\
#2 \\
~
\end{array}
\normalsize
\right)
}

\newcommand{\dispp}[3]{
\tiny
\begin{tikzcd}
#1 \ar[d, "#2"] \\
#3
\end{tikzcd}
\normalsize
}

\title{Notes: The Formal Theory of Monads}
\author{Kevin Clancy} 

\begin{document}

\maketitle

\section*{Monads with Respect to a 2-category}

\subsection*{Theorem 1}

Because I am still a 2-category novice, the proof of this theorem initially seemed a bit foreign to me.
In 1-categories, an adjunction $F \vdash U : \mathbf{C} \leftrightarrows \mathbf{D}$ is a natural isomorphism
$\mathbf{C}(FX,Y) \cong \mathbf{D}(X,UY)$, i.e. it is a natural isomorphism between the functors $\mbf{C}(F-,-)$ and $\mbf{D}(-,U-)$, which both have type $\mbf{D}^{\mathit{op}} \times \mbf{C} \to \mbf{Sets}$.\\~\\
In contrast, in the setting of 2-categories $\mathbf{C}(FX,Y)$ is a hom-category rather than a hom-set; its objects are $\mbf{C}$-1-cells from $FX$ to $Y$, and its arrows are $\mbf{C}$-2-cells. Thus, the 2-functor $\mbf{C}(FX,-)$ maps a 0-cell to a category, a 1-cell to a functor, and a 2-cell to a natural transformation. E.g., visualize arrows in the hom-category $\mbf{C}(FX,Y)$ in the below diagram.\\~\\ 

\[\begin{tikzcd}
	&& Y \\
	\\
	FX &&& {} \\
	\\
	&& Z
	\arrow[from=3-1, to=1-3]
	\arrow[from=3-1, to=5-3]
	\arrow[""{name=0, anchor=center, inner sep=0}, "f"', curve={height=18pt}, from=1-3, to=5-3]
	\arrow[""{name=1, anchor=center, inner sep=0}, "g", curve={height=-18pt}, from=1-3, to=5-3]
	\arrow["\alpha", shorten <=7pt, shorten >=7pt, Rightarrow, from=0, to=1]
\end{tikzcd}\]

Post-composition (via horizontal composition) of these arrows (viewed as 2-cells of $\mbf{C}$) with $f : X \to Z$ is an ordinary functor between hom-categories $\mbf{C}(FX,f) : \mbf{C}(FX,Y) \to \mbf{C}(FX,Z)$. Given a $\alpha : f \to g \in \mbf{C}_2$ we obtain a natural transformation $\mbf{C}(FX,\alpha)$ from the ordinary functor $\mbf{C}(FX,f)$ to the ordinary functor $\mbf{C}(FX,g)$ such that for all $h : FX \to Y$, 
$$\mbf{C}(FX,\alpha)_{h} \doteq \alpha h$$
\\~\\
Let $\psi$ be the isomorphism described in theorem 1. For ordinary naturality of $\psi$ in the codomain position,
we must have for all $f : X \to Z$ that $\mbf{C}(Y,f);\psi = \psi;\text{Mnd}(\mbf C)((Y,T),(f,1))$.   
Let $h : X \to Z \in \mbf{C}_1$. We have

$$(h : X \to Z) \overset{\mbf{C}(Y,f)}{~~\mapsto~~} (h , h\eta_T) \overset{\psi}{\mapsto} (fh , fh \eta_T)$$ 

On the other hand, we have

$$(h : X \to Z) \overset{\psi}{~~\mapsto~~} (h , h \eta_T) \overset{\text{Mnd}(\mbf C)((Y,T),(f,1))}{\mapsto} (fh,fh\eta_T \cdot fh) = (fh,fh\eta_T) $$

Where $Mnd(\mbf{C})((Y,T),(f,1))$ is post-compostion with the monad morphism $(f,1)$, and $fh \eta_T \cdot fh$ is 
shown in the following pasting diagram.

\[\begin{tikzcd}[column sep=large]
	Y & X & Z \\
	Y & X & Z
	\arrow["T"', from=1-1, to=2-1]
	\arrow[equals, from=1-3, to=2-3]
	\arrow[equals, from=1-2, to=2-2]
	\arrow["h"', from=2-1, to=2-2]
	\arrow["f"', from=2-2, to=2-3]
	\arrow["h", from=1-1, to=1-2]
	\arrow["f", from=1-2, to=1-3]
	\arrow["{h \eta_T}", shorten <=4pt, Rightarrow, from=1-2, to=2-1]
	\arrow["{1_f}", shorten <=4pt, shorten >=4pt, Rightarrow, from=1-3, to=2-2]
\end{tikzcd}\]

In addition, we must show 2-naturality, i.e. that the following two natural transformations are equal.

\[\begin{tikzcd}
	{\mathbf{C}(Y,X)} &&& {\mathbf{C}(Y,Z)} &&&& {\text{Mnd}(\mathbf{C})((Y,T),(Z,1))} \\
	\\
	\\
	{\mathbf{C}(Y,X)} &&& {\text{Mnd}(\mathbf{C})((Y,T),(X,1))} &&&&& {\text{Mnd}(\mathbf{C})((Y,T),(Z,1))}
	\arrow[""{name=0, anchor=center, inner sep=0}, "{\mathbf{C}(Y,f)}", curve={height=-24pt}, from=1-1, to=1-4]
	\arrow[""{name=1, anchor=center, inner sep=0}, "{\mathbf{C}(Y,g)}"', curve={height=24pt}, from=1-1, to=1-4]
	\arrow["{(U :Y \to Z) \mapsto (U, U \eta) \atop \sigma \mapsto \sigma}"', from=1-4, to=1-8]
	\arrow["{(U : Y \to X) \mapsto (U, U \eta) \atop \sigma \mapsto \sigma}"', from=4-1, to=4-4]
	\arrow[""{name=2, anchor=center, inner sep=0}, "{\text{Mnd}(\mathbf{C})((Y,T),\text{Inc}_{\mathbf C}(f))}", curve={height=-24pt}, from=4-4, to=4-9]
	\arrow[""{name=3, anchor=center, inner sep=0}, "{\text{Mnd}(\mathbf{C})((Y,T),\text{Inc}_{\mathbf C}(g))}"', curve={height=24pt}, from=4-4, to=4-9]
	\arrow["{\mathbf{C}(Y,\alpha)}", shorten <=6pt, shorten >=6pt, Rightarrow, from=0, to=1]
	\arrow["{\text{Mnd}(\mathbf{C})((Y,T),\text{Inc}_{\mathbf C}(\alpha))}", shorten <=6pt, shorten >=6pt, Rightarrow, from=2, to=3]
\end{tikzcd}\]

\end{document}
