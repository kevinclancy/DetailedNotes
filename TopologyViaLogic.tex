\documentclass{article}
 
\usepackage{amscd}
\usepackage{graphicx}
\usepackage{amsmath}
\usepackage{amssymb}
\usepackage{amsthm}
\usepackage{stmaryrd}
\usepackage{mathpartir}
\usepackage{multicol}
\usepackage{enumitem}
\usepackage[section]{placeins} % float barriers
\usepackage{natbib}
\usepackage{xcolor} 
\usepackage{bussproofs} 
\usepackage{diagrams}
\usepackage{tikz}
 
\usetikzlibrary{cd}
 
\newtheorem{lemma}{Lemma}

%example: \limit{j \in J}{F_j}
\newcommand{\limit}[2]{\underset{\overset{\longleftarrow}{#1}}{\text{lim}}~#2}
\newcommand{\lims}[1]{\underset{\longleftarrow}{\text{lim}}~#1}
\newcommand{\mbf}{\mathbf}
\newcommand{\sem}[1]{\llbracket #1 \rrbracket}
\newcommand{\defeq}{\overset{\mathit{def}}{=}}


\newcommand{\vrt}[2]{
\pile{
#1 \\
\downarrow \\
#2
}
}

\newcommand{\ddisp}[3]{
\left(
\scriptsize
\begin{tikzcd}
#1 \ar[d, "\footnotesize{#2}"] \\
#3
\end{tikzcd}
\normalsize
\right)
}

\newcommand{\disp}[3]{
\left(
\tiny
\begin{array}{c}
#1 \\
\downarrow\\
#3
\end{array}
\begin{array}{l}
~ \\
#2 \\
~
\end{array}
\normalsize
\right)
}

\newcommand{\dispp}[3]{
\tiny
\begin{tikzcd}
#1 \ar[d, "#2"] \\
#3
\end{tikzcd}
\normalsize
}

\title{Solutions: Topology via Logic}

\begin{document}

\maketitle

\section*{Chpt 3}

\subsection*{3.1}

Prove directly that the subsets of $\mathbb R$ that satisfy the second condition in Proposition 3.9.1 are closed
under finite intersections and arbitrary unions and hence form a frame (a subframe of $\mathcal P \mathbb R$).\\~\\
Seems fairly straightforward. I can think this one through, so I don't feel the need to transcribe it.

\subsection*{3.2}

A subbasic assertion $(x \pm \varepsilon)$ clearly contains more than one point. So the only way to obtain a smaller open
set is to take a finite meet of several subbasic assertions. It isn't hard to see that this either results in $\mathit{false}$ 
or yet another subbasic assertion, and so it isn't possible to obtain a singleton set in this way. Therefore singleton 
sets are not open.

\subsection*{3.4}

\begin{lemma}
For all open sets $U$ in a toplogical space, $\neg U = \text{Int}(U^C)$ 
\label{lemma3.4.1}
\end{lemma}

\begin{proof}
By the definition of $- \to -$, $$V \subseteq U \to \emptyset~(\mathit{aka}~\neg U) \leftrightarrow V \cap U \subseteq \emptyset$$

In words, an open set is contained in $\neg U$ iff it is contained in $U^C$, since $V \cap U \subseteq \emptyset \leftrightarrow V \cap U = \emptyset$. But $\neg U$ is an open set by definition, and so it is the union of all open sets it contains. Hence it is the union of all open sets in $U^C$, i.e. $\neg U = \text{Int}(U^C)$.

\end{proof}

\begin{lemma}
For all open sets $U$ in a toplogical space, $\text{Int}(U^C)^C = \text{Cl}(U)$
\label{lemma3.4.2}
\end{lemma}

\begin{proof}
We show that they have the same complements: $\text{Int}(U^C) = \text{Cl}(U)^C$.
We have $$V \subseteq \mathit{Int}(U^C) \leftrightarrow V \subseteq U^C \leftrightarrow U \subseteq V^C \leftrightarrow \mathit{Cl}(U) \subseteq V^C  \leftrightarrow V \subseteq \mathit{Cl}(U)^C$$

\end{proof}

Finally, we get

$$\neg \neg U = \mathit{Int}( \mathit{Int}(U^C)^C) = \mathit{Int}( \mathit{Cl}(U) ) $$

by applying Lemma \ref{lemma3.4.1} on the left and Lemma \ref{lemma3.4.2} on the right.

\subsection*{3.5}

It's easy to see that $x \wedge \neg x = \mbf{false}$, since the interior of a set's complement is contained
in it its complement we have $\sem{\neg x} \subseteq \sem{x}^C$ and hence $\sem{x \wedge \neg x} = \sem{x} \cap \sem{\neg x} = \emptyset = \mbf{false}$.

In $\Omega \mathbb R$, for any set $x$ of the form $(a,b)$ with $a < b$ we have $x \vee \neg x \neq \mbf{true}$. 
This is because $\sem{\neg x} = \mathit{Int}(\sem{x}^C) = (-\infty,a) \cup (b, \infty)$.

\subsection*{3.6}

This actually isn't an exercise.

\subsection*{3.7}

We need to prove that that in $a \wedge \bigvee S$, $a$ distributes over $\bigvee S$, i.e. that $a \wedge \bigvee S = \bigvee_{b \ in S} (a \wedge b)$. $\mbf{Crucially}$, we need to show that $\bigvee_{b \in S} (a \wedge b)$ exists at all, for if I understand correctly, we are only allowed to assume that $\bigvee S$ exists.

\subsubsection*{Lattices}

If we were allowed to assume that $\bigvee_{b \in S} (a \wedge b)$ exists, this would be true not just for Heyting 
algebras, but all lattices. We first prove $$(a \wedge \bigvee_{s \in S} x) \leq \bigvee_{x \in S} (a \wedge x)$$.

Showing that for all $z$, $z \leq a \wedge \bigvee S \Rightarrow z \leq \bigvee_{x \in S}(a \wedge x)$ 
is sufficient for this. To this end, let $z \leq a \wedge \bigvee S$. Then $z \leq a$ and $z \leq \bigvee S$. Hence $z$ is not an upper bound of $S$, i.e. there is some $x \in S$ with $z \leq x$. Then $z \leq a \wedge x \leq \bigvee_{x \in S} (a \wedge x)$.

Now we prove inequality in the other direction.

$$\bigvee_{x \in S} (a \wedge x) \leq (a \wedge \bigvee_{s \in S} x)$$

For any $x$, $a \wedge x \leq (a \wedge \bigvee_{s \in S} x)$ due to the monotonicity of $- \wedge -$ in its second argument. Since $(a \wedge \bigvee_{s \in S} x)$ is an upper bound of the set $\{ a \wedge x \mid x \in S \}$, it is greater than this set's least upper bound $\bigvee_{x \in S} (a \wedge x)$.   

\subsubsection*{Heyting algebras}

We need only prove that $\bigvee_{b \in S} a \wedge b$ exists. To do that, we show that $a \wedge \bigvee S$ is the least upper 
bound of $X \doteq \{ a \wedge b \mid b \in S \}$.\\~\\
$a \wedge \bigvee S$ is clearly an upper bound of $X$ due to the monoticity of $-\wedge-$ in its second argument.\\~\\
To show that it is the \emph{least} upper bound of $X$, let $z$ be an upper bound of $X$. Then for all $b \in S$, $a \wedge b \leq z \therefore b \leq a \to z \therefore \bigvee S \leq a \to z \therefore (a \wedge \bigvee S) \leq z$. 

\subsection*{3.8}

TODO

\section*{Chpt 4} 

\subsection*{4.1}

The carrier set is the set of elements of the lattice. $- \wedge -$ is interpreted as the greatest lower bound operator.
$- \vee -$ is interpreted as the least upper bound operator. $\mbf{true}$ is interpreted as the top element. $\mbf{false}$ 
is interpreted as the bottom element. TODO: derive the algebraic lattice equations from order theoretic properties of lattices.

\subsection*{4.2}

This isn't an exercise, but okay.

\subsection*{4.3}

\subsubsection*{Order Theoretic $\Rightarrow$ Algebraic}

We first assume the order theoretic properties of a Heyting algebra and derive the two listed equalities from them.

\begin{description}

\item[$\mbf{a \wedge (a \to b) = a \wedge b:}$]~\\

We show $a \wedge (a \to b)$ is the greatest lower bound of $\{ a, b\}$. First, we show it is a lower bound. Clearly, $a \wedge (a \to b) \leq a$. Furthermore, by reflexivity we have $a \to b \leq a \to b$ and hence $a \wedge (a \to b) \leq b$.

Now we show that $a \wedge (a \to b)$ is the \emph{least} upper bound of $\{ a, b \}$. Suppose $z$ is some other lower bound of $\{ a, b \}$, i.e. $z \leq a$ and $z \leq b$. Then

\[
\begin{array}{ll}
 ~ & z \leq a \text{ and } z \wedge a \leq z \leq b \\
 \leftrightarrow & z \leq a \text{ and } z \leq a \to b \\
 \leftrightarrow & z \leq a \wedge (a \to b)
\end{array} 
\]

\item[$\mbf{c = c \wedge (a \to ((a \wedge c) \vee b)):}$]~\\

By the connecting lemma ($a = a \wedge b \Leftrightarrow a \leq b$), we just need to show that
$c \leq a \to ((a \wedge c) \vee b)$. We have
$$c \leq a \to ((a \wedge c) \vee b)~~~\Leftrightarrow~~~a \wedge c \leq (a \wedge c) \vee b$$
The right-hand inequality above is true because $(a \wedge c) \vee b$, being the least upper bound 
of $a \wedge c$ and $b$, is in particular an upper bound of $a \wedge c$.

\end{description}

\subsubsection*{Algebraic $\Rightarrow$ Order Theoretic}

\section*{5}

\subsection*{5.1}

\newcommand{\comp}{\mathit{comp}}

Due to the following equivalences:
$$\mathit{comp}(b_1 b_2 \cdots) \vdash `s_i = 1'~~~\Leftrightarrow~~~(b_1 b_2 \cdots) \vdash \Omega \mathit{comp}(`s_i = 1')$$
$$\mathit{comp}(b_1 b_2 \cdots) \vdash `s_i = 1'~~~\Leftrightarrow~~~(b_1 b_2 \cdots) \vdash \Omega \mathit{comp}(`s_i = 1')$$
We need $\Omega \comp(`s_i = 0') = `s_i = 1'$ and $\Omega \comp(`s_i = 1') = `s_i = 0'$.\\~\\
Of course, by definition $\Omega \comp$ must also be a frame homomorphism from $\Omega 2^{*\omega}$ to $\Omega 2^{* \omega}$, i.e. a model morphism for the presentation\\~\\
\begin{tabular}{ll}
$P = \mathit{Fr} \langle `s_n = 0', `s_n = 1' : n \geq 1 \mid$ & $`s_n = 0' \wedge `s_n = 1' = \mbf{false}$ \\
 & $`s_{n+1} = 0' \vee `s_{n+1}=1' \leq `s_n = 0' \vee `s_n = 1' \rangle$\\
\end{tabular}\\~\\
The term model $T$ of the presentation $P$ is initial ($\mbf{TODO}$: prove lemma?). We have another model $T'$ of $P$ that is similar to the term model. Its carrier set is the set of equivalence classes of expressions built from $- \wedge -$, $\bigvee -$, $\mbf{true}$, $\mbf{false}$, $`s_n = 0'$, and $`s_n = 1'$. Such that $\sem{a \wedge b} = \sem{a} \wedge \sem{b}$, $\sem{\bigvee_{i \in I} a_i} = \bigvee_{i \in I} \sem{a_i}$, $\sem{\mbf{true}} = \sem{\mbf{true}}$, $\sem{\mbf{false}} = \mbf{false}$, $\sem{`s_n = 1'} = `s_n = 0'$, and $\sem{`s_n = 0'} = `s_n = 1'$.

TODO: finish (maybe it we be easier to have a model/algebra that uses sets of finite and infinite bit strings as its carrier set?)  

\subsection*{7}

\subsubsection*{7.1}

\begin{description}

\item[$(\Rightarrow)$:]~\\~\\
Trivially follows from the definition of the specialization preorder.

\item[$(\Leftarrow)$:]~\\~\\
Suppose that for every subbasic open $a$, $x \vDash a$ implies $y \vDash a$.\\~\\
Let $b$ be an arbitrary open. Then $b = \bigvee_{i \in I} (a_{i1} \wedge \cdots \wedge a_{i n_i})$, where
$I$ is an aribtrary index set and the $a_{ij}$'s are subbasic opens.\\~\\
Suppose $x \vDash b$.\\~\\
\begin{tabular}{l}
$\therefore \exists i \in I. x \vDash a_{i1} \wedge \cdots \wedge a_{i n_i}$ \\
$\therefore x \vDash a_{i j}$ for $j \in 1..n_i$ \\
$\therefore y \vDash a_{i j}$ for $j \in 1..n_i$ \\
$\therefore \exists i \in I. y \vDash a_{i1} \wedge \cdots \wedge a_{i n_i}$ \\
$\therefore y \vDash b$ \\
\end{tabular}~\\~\\
Since $x \vDash b$ implies $y \vDash b$ we have $x \sqsubseteq y$.
\end{description}

\subsection*{8}

\subsubsection*{8.1}
We first prove the $(\Leftarrow)$ direction. Consider what it means for a subset $A \subseteq 2^{* \omega}$ to be open: it means that for all $l,s \in \mathit{pt}~2^{* \omega}$ such that $l \sqsubseteq s$
we have $s \in A \Rightarrow l \in A$.\\~\\
Now, let $A \subseteq 2^{\omega}$ be closed. We define:\\~\\

For $x \in A'$ we define $l_x$ as the shortest prefix of $x$ that is not a prefix of any element of $A$. $l_x$ would not be well defined if $A'$ were not open, because then we would have some element of $A'$ that is a prefix of some element of $A$. We define
$$T \doteq \{ l_x \mid x \in A' \}$$
We further define:
$$U_i \doteq \bigvee \{ \mbf{starts}~l \mid \text{length}~l = i \text{ and } l \not \in T \} $$
$$W_i \doteq  \bigvee \{ \mbf{starts}~l \mid \text{length}~l = i \}$$
Clearly, $\bigcap U_i = \bigcap W_i - \uparrow T$.\\~\\
We now prove that $A = \bigcap_{i} U_i$. Let $x \in A$. Let $i \in \mathbb N$ and consider the the length-$i$ prefix $x_i$ of $x$. $x_i \not \in T$ since $T$ does not contain prefixes of elements of $A$ by definition. Hence $x \in U_i$ for all $i \in \mathbb N$. Hence $x \in \bigcap_{i} U_i$.

Conversely, let $x \in \bigcap_i U_i = \bigcap_i W_i - \uparrow T$. Since $x \in \bigcap_i W_i$, it is an infinite string. Since $x \not \in \uparrow T$, each of its prefixes is shared with an element of $A$. Hence $x \in A$.

Each $U_i$ is compact and saturated, and we have $A = \bigcap_{i} U_i$. Since $A$ is the intersection of compact saturated sets,
it is itself compact and saturated. \\~\\
We now prove the $(\Rightarrow)$ direction. Suppose that $A$ is compact.

TODO: finish. But here are some thoughts while I am thinking about this. A compact set of infinite strings must be a finite set due to the ``every open cover has a finite subcover'' formulation of compactness. Is the complement of a finite set of infinite strings open? Yes: for each string we can observe the disjunction of negations of all prefixes. We then take the conjunction of the resulting disjunctions (which is a finite conjunction since the set of strings is finite.)

\subsection*{8.2}

We first show that $Qf(K)$ is a filter.

\begin{description}

\item[$Qf(K)\text{ is upward closed}:$]~\\~\\
Let $b \in Qf(K)$ and $c \in \Omega E$ with $b \leq c$. Then $K \vDash \Omega f(b) \leq \Omega f (b) \vee \Omega f (c) = \Omega f(b \vee c) = \Omega f(c)$.

\item[$Qf(K)\text{ is closed under finite meets}:$]~\\~\\
Let $S \subseteq_{\mathit{fin}} \Omega E$. Suppose that $K \vDash \Omega f(b)$ for all $b \in S$. Then $K \vDash \wedge_{b \in S} \Omega f(b)$.
Hence $K \vDash \Omega f(\wedge_{b \in S} b)$.
  
\end{description}

It's easy to see that $Qf(K)$ is inaccessible by directed joins, for suppose that $K \vDash \Omega f(\bigvee^{\uparrow} B)$. Since $\Omega f$ 
is a frame homomorphism we have $K \vdash \bigvee^{\uparrow} \Omega f(B)$. Since directed joins are disjunctions, we have that 
$K \vdash \Omega f(b)$ for some $b \in B$. Hence $Qf(K)$ is Scott open.\\~\\
For the next part, we have \\~\\
\begin{tabular}{l}
$Qf(K)$  \\
$= Qf( \wedge C )$ \\
$= \{ b \in \Omega E \mid \wedge C \vDash \Omega f(b) \}$ \\
$= \{ b \mid y \vDash \Omega f(b) \text{ for all } y \in C \}$ \\
$= \{ b \mid \mathit{pt} f(y) \vDash b \text{ for all } y \in C \}$  
\end{tabular}\\~\\~\\
We now show that $Qf$ preserves finite meets:~\\~\\
$Qf( \bigcap S )$\\
$= \{ b \in \Omega E \mid \bigcap S \vDash \Omega f(b) \}$ \\
$= \{ b \in \Omega E \mid \mid \forall K \in S. \Omega f(b) \in K \}$\\
$= \bigcap_{K \in S} \{ b \in \Omega E \mid \Omega f(b) \in K \}$\\
$= \bigcap_{K \in S} \{ b \in \Omega E \mid K \vDash \Omega f(b) \}$\\
$= \bigcap_{K \in S} Qf(K)$\\~\\
We now show that $Qf$ preserves directed joins:~\\~\\
$Qf(\bigsqcup^{\uparrow} S)$\\
$= \{ b \mid \bigsqcup^{\uparrow} S \vDash \Omega f(b) \}$\\
$= \{ b \mid \exists K \in S. K \vDash \Omega f (b) \}$\\
$= \bigcup_{K \in S} \{ b \mid K \vDash \Omega f(b) \}$\\
$= \bigcup_{K \in S} Qf(K)$\\
$= \bigsqcup^{\uparrow} Qf(S)$\\~\\
TODO: show that $Q$ is functorial.


\section*{9}

\subsection*{9.1}

For $p \in P$, the sets $\uparrow p$ form a basis for the Alexandrov topology on $P$, so we just need to define our isomorphism on these sets.
The only obvious mapping that I can see is $\uparrow p \mapsto \uparrow (\downarrow p)$, i.e. the principal upset of $p$ is mapped to the set of all ideals containing the principal downset of $p$.

Our isomorphism $h : \mathcal{A}(P) \to \mathit{Scott}(\mathit{Idl}(P))$ is then defined as
$$ h( \bigcup_i  \uparrow a_i ) \defeq \bigcup_i ( \uparrow (\downarrow a_i) )$$

I.e., an upward closed subset $X$ of $P$ is mapped to the set of all ideals that contain at least one element of $X$.
The latter is clearly Scott open.\\~\\
We now prove that the map $h$ is onto. Let $A \in \mathit{Scott}(\mathit{Idl}(P))$. For $I \in A$, since $I$ is an ideal, it can be expressed as a directed join of principal downsets: $I = \bigcup^{\uparrow}_{p \in I}\downarrow p$.
Since $A$ is Scott open, it is inaccessible by directed joins; therefore there exists a $p_I \in I$ such that 
$\downarrow p_I \in A$.  

Then $\bigcup_{I \in A} \uparrow p_I$ is an upward closed set. We now prove $h(\bigcup_{I \in A} \uparrow p_I)=A$. Or, unfolding the definition of $h$, $\bigcup_{I \in A} (\uparrow (\downarrow p_I) ) = A$. 

\begin{description}

\item[$\bigcup_{I \in A} (\uparrow (\downarrow p_I) ) \subseteq A$:]~\\~\\
Let $J \in \bigcup_{I \in A} (\uparrow (\downarrow p_I) )$. Then there exists an $I \in A$ with $p_I \in J$ and hence $\downarrow p_I \subseteq J$. Since $\downarrow p_I \in A$ and $A$ is upward closed we have $J \in A$.

\item[$A \subseteq \bigcup_{I \in A} (\uparrow (\downarrow p_I) )$:]~\\

Let $J \in A$. Then there exists a $p_J \in J$ with $\downarrow p_J \in A$. Since $J$ contains $\downarrow p_J$
we have $J \in \bigcup_{I \in A} (\uparrow (\downarrow p_I) )$.

\end{description}~\\
We now prove that the map $h$ is one-to-one. Let $X,Y \in \mathcal A(P)$ with $h(X) = h(Y)$ or, unfolding the
definition of $h$, $\bigcup_{x \in X} (\uparrow (\downarrow x)) = \bigcup_{y \in Y} (\uparrow (\downarrow y))$.

\begin{description}

\item[$\bigcup_{x \in X} (\uparrow (\downarrow x)) \subseteq \bigcup_{y \in Y} (\uparrow (\downarrow y))$:]~\\

Hello

\end{description}



\end{document}
