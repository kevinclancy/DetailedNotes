\documentclass{article}

\usepackage{amscd}
\usepackage{graphicx}
\usepackage{amsmath}
\usepackage{amssymb}
\usepackage{amsthm}
\usepackage{stmaryrd}
\usepackage{mathpartir}
\usepackage{multicol}
\usepackage{enumitem}
\usepackage[section]{placeins} % float barriers
\usepackage{natbib}
\usepackage{xcolor} 
\usepackage{bussproofs} 
\usepackage{diagrams}
\usepackage{tikz}

\usetikzlibrary{cd}
 
\newtheorem{lemma}{Lemma}

%example: \limit{j \in J}{F_j}
\newcommand{\limit}[2]{\underset{\overset{\longleftarrow}{#1}}{\text{lim}}~#2}
\newcommand{\lims}[1]{\underset{\longleftarrow}{\text{lim}}~#1}
\newcommand{\mbf}{\mathbf}
\newcommand{\sem}[1]{\llbracket #1 \rrbracket}


\newcommand{\vrt}[2]{
\pile{
#1 \\
\downarrow \\
#2
}
}

\newcommand{\ddisp}[3]{
\left(
\scriptsize
\begin{tikzcd}
#1 \ar[d, "\footnotesize{#2}"] \\
#3
\end{tikzcd}
\normalsize
\right)
}

\newcommand{\disp}[3]{
\left(
\tiny
\begin{array}{c}
#1 \\
\downarrow\\
#3
\end{array}
\begin{array}{l}
~ \\
#2 \\
~
\end{array}
\normalsize
\right)
}

\newcommand{\dispp}[3]{
\tiny
\begin{tikzcd}
#1 \ar[d, "#2"] \\
#3
\end{tikzcd}
\normalsize
}

\title{Solutions: Topology via Logic}

\begin{document}

\maketitle

\section*{Chpt 3}

\subsection*{3.1}

Prove directly that the subsets of $\mathbb R$ that satisfy the second condition in Proposition 3.9.1 are closed
under finite intersections and arbitrary unions and hence form a frame (a subframe of $\mathcal P \mathbb R$).\\~\\
Seems fairly straightforward. I can think this one through, so I don't feel the need to transcribe it.

\subsection*{3.2}

A subbasic assertion $(x \pm \varepsilon)$ clearly contains more than one point. So the only way to obtain a smaller open
set is to take a finite meet of several subbasic assertions. It isn't hard to see that this either results in $\mathit{false}$ 
or yet another subbasic assertion, and so it isn't possible to obtain a singleton set in this way. Therefore singleton 
sets are not open.

\subsection*{3.4}

\begin{lemma}
For all open sets $U$ in a toplogical space, $\neg U = \text{Int}(U^C)$ 
\label{lemma3.4.1}
\end{lemma}

\begin{proof}
By the definition of $- \to -$, $$V \subseteq U \to \emptyset~(\mathit{aka}~\neg U) \leftrightarrow V \cap U \subseteq \emptyset$$

In words, an open set is contained in $\neg U$ iff it is contained in $U^C$, since $V \cap U \subseteq \emptyset \leftrightarrow V \cap U = \emptyset$. But $\neg U$ is an open set by definition, and so it is the union of all open sets it contains. Hence it is the union of all open sets in $U^C$, i.e. $\neg U = \text{Int}(U^C)$.

\end{proof}

\begin{lemma}
For all open sets $U$ in a toplogical space, $\text{Int}(U^C)^C = \text{Cl}(U)$
\label{lemma3.4.2}
\end{lemma}

\begin{proof}
We show that they have the same complements: $\text{Int}(U^C) = \text{Cl}(U)^C$.
We have $$V \subseteq \mathit{Int}(U^C) \leftrightarrow V \subseteq U^C \leftrightarrow U \subseteq V^C \leftrightarrow \mathit{Cl}(U) \subseteq V^C  \leftrightarrow V \subseteq \mathit{Cl}(U)^C$$

\end{proof}

Finally, we get

$$\neg \neg U = \mathit{Int}( \mathit{Int}(U^C)^C) = \mathit{Int}( \mathit{Cl}(U) ) $$

by applying Lemma \ref{lemma3.4.1} on the left and Lemma \ref{lemma3.4.2} on the right.

\subsection*{3.5}

It's easy to see that $x \wedge \neg x = \mbf{false}$, since the interior of a set's complement is contained
in it its complement we have $\sem{\neg x} \subseteq \sem{x}^C$ and hence $\sem{x \wedge \neg x} = \sem{x} \cap \sem{\neg x} = \emptyset = \mbf{false}$.

In $\Omega \mathbb R$, for any set $x$ of the form $(a,b)$ with $a < b$ we have $x \vee \neg x \neq \mbf{true}$. 
This is because $\sem{\neg x} = \mathit{Int}(\sem{x}^C) = (-\infty,a) \cup (b, \infty)$.

\subsection*{3.6}

This actually isn't an exercise.

\subsection*{3.7}

We need to prove that that in $a \wedge \bigvee S$, $a$ distributes over $\bigvee S$, i.e. that $a \wedge \bigvee S = \bigvee_{b \ in S} (a \wedge b)$. $\mbf{Crucially}$, we need to show that $\bigvee_{b \in S} (a \wedge b)$ exists at all, for if I understand correctly, we are only allowed to assume that $\bigvee S$ exists.

\subsubsection*{Lattices}

If we were allowed to assume that $\bigvee_{b \in S} (a \wedge b)$ exists, this would be true not just for Heyting 
algebras, but all lattices. We first prove $$(a \wedge \bigvee_{s \in S} x) \leq \bigvee_{x \in S} (a \wedge x)$$.

Showing that for all $z$, $z \leq a \wedge \bigvee S \Rightarrow z \leq \bigvee_{x \in S}(a \wedge x)$ 
is sufficient for this. To this end, let $z \leq a \wedge \bigvee S$. Then $z \leq a$ and $z \leq \bigvee S$. Hence $z$ is not an upper bound of $S$, i.e. there is some $x \in S$ with $z \leq x$. Then $z \leq a \wedge x \leq \bigvee_{x \in S} (a \wedge x)$.

Now we prove inequality in the other direction.

$$\bigvee_{x \in S} (a \wedge x) \leq (a \wedge \bigvee_{s \in S} x)$$

For any $x$, $a \wedge x \leq (a \wedge \bigvee_{s \in S} x)$ due to the monotonicity of $- \wedge -$ in its second argument. Since $(a \wedge \bigvee_{s \in S} x)$ is an upper bound of the set $\{ a \wedge x \mid x \in S \}$, it is greater than this set's least upper bound $\bigvee_{x \in S} (a \wedge x)$.   

\subsubsection*{Heyting algebras}

We need only prove that $\bigvee_{b \in S} a \wedge b$ exists. To do that, we show that $a \wedge \bigvee S$ is the least upper 
bound of $X \doteq \{ a \wedge b \mid b \in S \}$.\\~\\
$a \wedge \bigvee S$ is clearly an upper bound of $X$ due to the monoticity of $-\wedge-$ in its second argument.\\~\\
To show that it is the \emph{least} upper bound of $X$, let $z$ be an upper bound of $X$. Then for all $b \in S$, $a \wedge b \leq z \therefore b \leq a \to z \therefore \bigvee S \leq a \to z \therefore (a \wedge \bigvee S) \leq z$. 

\subsection*{3.8}

TODO

\section*{Chpt 4} 

\subsection*{4.1}

The carrier set is the set of elements of the lattice. $- \wedge -$ is interpreted as the greatest lower bound operator.
$- \vee -$ is interpreted as the least upper bound operator. $\mbf{true}$ is interpreted as the top element. $\mbf{false}$ 
is interpreted as the bottom element. TODO: derive the algebraic lattice equations from order theoretic properties of lattices.

\subsection*{4.2}

This isn't an exercise, but okay.

\subsection*{4.3}

a

\end{document}
