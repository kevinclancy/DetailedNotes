\documentclass{article}

\usepackage{amscd}
\usepackage{graphicx}
\usepackage{amsmath}
\usepackage{amssymb}
\usepackage{amsthm}
\usepackage{stmaryrd}
\usepackage{mathpartir}
\usepackage{multicol}
\usepackage{enumitem}
\usepackage[section]{placeins} % float barriers
\usepackage{natbib}
\usepackage{xcolor} 
\usepackage{bussproofs} 
\usepackage{diagrams}
\usepackage{tikz}

\usetikzlibrary{cd}
 
\newtheorem{lemma}{Lemma}

%example: \limit{j \in J}{F_j}
\newcommand{\limit}[2]{\underset{\overset{\longleftarrow}{#1}}{\text{lim}}~#2}
\newcommand{\lims}[1]{\underset{\longleftarrow}{\text{lim}}~#1}
\newcommand{\mbf}{\mathbf}
\newcommand{\sem}[1]{\llbracket #1 \rrbracket}
\newcommand{\csem}[1]{( \! | #1 | \! )}
\newcommand{\self}{a}
\newcommand{\vrt}[2]{
\pile{
#1 \\
\downarrow \\
#2
}
}

\newcommand{\ddisp}[3]{
\left(
\scriptsize
\begin{tikzcd}
#1 \ar[d, "\footnotesize{#2}"] \\
#3
\end{tikzcd}
\normalsize
\right)
}

\newcommand{\disp}[3]{
\left(
\tiny
\begin{array}{c}
#1 \\
\downarrow\\
#3
\end{array}
\begin{array}{l}
~ \\
#2 \\
~
\end{array}
\normalsize
\right)
}

\newcommand{\dispp}[3]{
\tiny
\begin{tikzcd}
#1 \ar[d, "#2"] \\
#3
\end{tikzcd}
\normalsize
}

\title{My Denotational Semantics for OOP}

\begin{document}

\maketitle

\section{Syntax}
Let $\mathit{FieldNames}$, $\mathit{MethodNames}$, and $\mathit{ClassNames}$ be countable sets of field names,
method names, and class names, respectively. Our language's syntax is defined as follows:\\~\\
\begin{tabular}{llll}
\multicolumn{4}{l}{$x,y,z \in \mathit{Vars}$} \\
\multicolumn{4}{l}{$\self~(\text{self variable}) \in \mathit{Vars}$} \\
\multicolumn{4}{l}{$\alpha \in \mathit{ClassNames}$} \\ 
\multicolumn{4}{l}{$f \in \mathit{FieldNames}$} \\
& & & \\
$\tau,\sigma$ (types) & $::=$ & $\alpha \mid \mathit{Int} \mid \mathit{Bool} \mid 1 \mid \tau \times \sigma \mid \tau \to \sigma$ & \\
$\beta$ (class defn.) & $::=$ & \multicolumn{2}{l}{$\lbrack f_i : \tau_i^{~i \in 1..n},~m_i : \sigma_i \to \sigma_i' = \lambda x~\varsigma z.t_i^{~i \in 1..k},~\mathit{constr_i} : \tau \to \tau_1 \times \cdots \times \tau_n = \lambda x. t \rbrack$} \\
              & & \multicolumn{2}{l}{\emph{where the $f_i$'s are distinct, and the $m_i$'s are also distinct}} \\
 & & & \\
$c,d$ (commands) & $::=$  & $\self.f_{ij} := t$ & (field update) \\
                 & $\mid$ & $!\self.f_{ij}$ & (field access)    \\
                 & $\mid$ & $s.m(t)$ & (method call)  \\
                 & $\mid$ & $\mbf{let}~x \Leftarrow c~\mbf{in}~d$ & (monadic bind) \\
                 & $\mid$ & $\mbf{let}~x = t~\mbf{in}~d$ & (expression bind) \\
                 & $\mid$ & $\mathbf{if}~t~\mathbf{then}~c~\mathbf{else}~d$ & (if command) \\
                 & $\mid$ & $\mathit{constr}_i~t$ & (constructor application) \\

 & & & \\

$t,s$ (terms) & $::=$  & $\mathit{k}$ & (constant occurrence) \\
              & $\mid$ & $\mathit{t~s}$ & (application) \\
              & $\mid$ & $x$ & (variable occurrence) \\ 
              & $\mid$ & $\langle t, s \rangle$ & (pair) \\
              & $\mid$ & $\mbf{inl}~t$ & (left injection) \\
              & $\mid$ & $\mbf{inr}~t$ & (right injection) \\
              & $\mid$ & $\mbf{case}~t~\mbf{with}~\mbf{inl}~x \to t~\mbf{inr}~x \to s$ & (case split) \\
              & $\mid$ & $\pi_i~t$ & (projection) \\
 & & & \\
$p$ (program) & $:=$ & $\alpha_i = \beta_i^{~i \in 1..n}$~~(where the $\alpha_i$'s are distinct) & \\
\if 0
 & & & \\
$\Delta$ (Field environment) & $:=$ & $\Delta,c_i.f_j : \tau \mid \emptyset$ & \\
$\Theta$ (Method environment) & $:=$ & $\Theta,c_i.m_j : \tau \to \sigma \mid \emptyset$ & \\
$\Omega$ (Constructor environment) & $:=$ & $\Omega,c_i.\mathit{constr} : \tau \to \tau_1 \times \cdots \times \tau_n \mid \emptyset$ & \\
\fi
$\Gamma$ (Primary environment) & $:=$ & $\Gamma,x:\tau \mid \emptyset$ & 
\end{tabular}

\section{Type system}

We define our typing judgments with respect to a background program $p = [f_{ij} : \tau_{ij}^{~j \in 1..n_i},~m_{ij} : \sigma_{ij} \to \sigma_{ij}'^{~j \in 1..l_i} = \lambda x. \varsigma z. t, \mathit{constr}_i : \tau \to \tau_{i1} \times \cdots \times \tau_{in_i} = \lambda x.t]^{i \in 1..n}$.

\begin{figure}
\begin{mathpar}
\inferrule[Var]
  {x : \tau \in \Gamma}
  {\self : c_i \mid \Gamma \vdash_p x : \tau}
\and
\inferrule[Self-Var]
  {~}
  {\self : \tau \mid \Gamma \vdash_p \self : \tau}
\and
\inferrule[App]
  {a : c_i \mid \Gamma \vdash_p t : \tau' \to \tau \\ a : c_i \mid \Gamma \vdash t' : \tau'}
  {a : c_i \mid \Gamma \vdash_p t~t' : \tau}
\and
\inferrule[Const]
  {\mathit{ty}(k) = \iota}
  {a : c_i \mid \Gamma \vdash k : \iota}

\end{mathpar}
\caption{Expression typing judgments $\self : c_k \mid \Gamma \vdash_p t : \tau$}
\begin{mathpar}

\and
\inferrule[Constr]
  {\mathit{constr}_i : \tau \to \tau_{i1} \times \cdots \times \tau_{in_i} = \lambda x.t' ~\in~ p \\ c_i : \alpha_k \mid \Gamma \vdash_p t : \tau}
  {\self : \alpha_k \mid \Gamma \vdash_p \mathit{constr}_i~t : T \alpha_i}
\and
\inferrule[Inv]
  {x : \alpha_i \in (\self : \alpha_k \mid \Gamma) \\ m_{ij} : \sigma_{ij} \to \sigma_{ij}' \in p \\ y : \sigma_{ij}' \in (\self : \alpha_k \mid \Gamma)}
  {\self : \alpha_k \mid \Gamma \vdash_p s.m_{ij}(t) : T \sigma_{ij}'}
\and
\inferrule[Get]
  {f_{ij} : \tau_{ij} \in p}
  {\self : c_i \mid \Gamma \vdash_p~!a.f_{ij} : T \tau_{ij}}
\and
\inferrule[Set]
  {f_{ij} : \tau_{ij} \in p \\ \self : c_i \mid \Gamma \vdash_{p} t : \tau_{ij}}
  {\self : c_i \mid \Gamma \vdash_p a.f_{ij} := t : T1}
\and
\inferrule[Term-Bind]
  {a : c_i \mid \Gamma \vdash_p t : \tau \\ a : c_i \mid \Gamma,x:\tau \vdash_p c : T \tau'}
  {a : c_i \mid \Gamma \vdash_p \mbf{let}~x = t~\mbf{in}~c : T \tau'}
\and
\inferrule[Monad-Bind]
  {a : c_i \mid \Gamma \vdash_p c : T \tau \\ a : c_i \mid \Gamma, x:\tau \vdash_p d : T \tau'}
  {a : c_i \mid \Gamma \vdash_{p} \mbf{let}~x \Leftarrow c~\mbf{in}~d : T \tau'}
\end{mathpar}
\caption{Command typing judgments $\self : \alpha_k \mid \Gamma \vdash_p c : T \tau$}
\end{figure}

\begin{figure}
\begin{mathpar}
\inferrule
  {p \doteq [f_{ij} : \tau_{ij}^{~j \in 1..n_i},~m_{ij} : \sigma_{ij} \to \sigma_{ij}'^{~j \in 1..l_i} = \varsigma z.\lambda x. t_{ij}, \mathit{constr}_i : \tau_i \to \tau_{i1} \times \cdots \times \tau_{in_i} = \lambda x.t]^{i \in 1..n} \\
   a:c_i \mid x:\sigma_{ij} \vdash_p t_j : \sigma_{ij}'^{~i \in 1..n, j \in 1..n_i} \\ 
   a:1 \mid x:\tau_i \vdash_p t : \tau_1 \times \cdots \times \tau_n^{~i \in 1 ..n}}
  { p :: \ast}
\end{mathpar}
\caption{Well typed program judgment $p :: \ast$}
\end{figure}

\if 0

\subsection{Field/Method Environment Generation}

\begin{figure}
\begin{mathpar}
\inferrule
  {p' \Uparrow \Delta' \mid \Theta'}
  {c_i = [f_j:\tau_j^{~j \in 1..n}, m_j : \sigma_j \to \sigma_j'^{~j \in 1..l},\cdots],p' \Uparrow \Delta \mid \Theta
 \\~\\
 \text{where } \Delta = \Delta',c_i.f_j:\tau_j,\ldots,c_i.f_n:\tau_n \\~\\~\\
 \text{and } \Theta = \Theta',c_i.m_1 : \sigma_1 \to \sigma_1', \ldots, c_i.m_l : \sigma_n \to \sigma_{n}'}
\and
 \inferrule{~}{\emptyset \Uparrow \emptyset \mid \emptyset}
\end{mathpar}

\label{fig:jgen}
\caption{Type ascription collection}
\end{figure}

Is this judgment really necessary? Can't I just write $c_i.f_j : \tau \Leftrightarrow$ $(f_j : \tau)$ is a field of class $c_i$.
$c_i.constr : \tau \to \tau_1 \times \cdots \times \tau_n$ $\Leftrightarrow$ 
$c_i = [\cdots constr : \tau \to \tau_1 \times \cdots \times \tau_n = \lambda x.t]$

\fi

\section{Preview}

Our semantics will interpret types as sets. 
We will then define a set $S$ that represents our program state. As a first approximation, we might interpret a 
command with domain $A$ and codomain $B$ monadically; defining $T$ as the state monad $T(X) \doteq (S \times -)^S$, 
such a command would be interpreted as a morphism $A \to B$ in the Kleisli category $\mathcal K \ell(T)$. 

However, since this is an 
object-oriented language with open recursion, we must accomodate divergence. 
Furthermore, our language features dynamic allocation. A command such as a method invocation must be allowed to 
proceed from any 
initial state; in particular, if we proceed from a state in which the target of the invocation is a dangling pointer, we must raise an 
exception. We therefore define our computation monad 
$T : \mbf{Sets} \to \mbf{Sets}$ as $T(X) \doteq ((S \times X) + 1 + 1)^S$, where 
$\kappa_2 \ast$ represents a diverging computation and $\kappa_3 \ast$ represents a 
dangling pointer exception. 



\section{State Interpretations $\sem{\cdot}$}

\begin{mathpar}
\begin{tabular}{l}
$\sem{\cdot} : \tau \to \mbf{Sets}$ \\
$\sem{Int} \doteq \mathbb{Z}$ \\
$\sem{Bool} \doteq \mathbb{B}$ \\
$\sem{c} \doteq \mathbb{N}$
\end{tabular}
\and

\begin{tabular}{l}
$\sem{\cdot} : d \to \mbf{Sets}$ \\
$\sem{~\lbrack f_i : \tau_i^{~i \in 1..n} \cdots \rbrack~} \doteq \sem{\tau_1} \times \cdots \times \sem{\tau_n}$
\end{tabular}
\and
\begin{tabular}{l}
$\sem{\cdot} : p \to \mbf{Sets}$\\
$\sem{c_i = d_i^{~i \in 1..n}} \doteq 
  (\mathbb N \overset{\mathit{fin}}{\rightharpoonup} \sem{d_1}) \times \cdots \times (\mathbb N \overset{\mathit{fin}}{\rightharpoonup} \sem{d_n}) $\\~\\
\end{tabular}
\end{mathpar}
% how do I left align the two-dollasign mode?



We refer to the set $\sem{c_i = d_i^{~i \in 1..n}}$ as $S$, because it is used to model our program's state 
\underline{S}tate.


\section{Interface Interpretations $\csem{\cdot}$}

Now that we have defined a set $S$ representing the program state, our computation monad
$T(X) \doteq ((S \times X) + 1 + 1)^S$ is fully defined.
We can now interpret methods, fields, OOP objects, and classes.

As a first approximation, a getter for a field field $f : \tau$ can be interpreted as an arrow of the form 
$\mathbb N \to \sem{\tau}$ in $\mathcal K \ell(T)$, and a setter for $f : \tau$ can be interpreted as an arrow
of the form $\mathbb N \times \sem{\tau} \to 1$ in $\mathcal K \ell(T)$. Likewise, a premethod 
$m : \sigma \to \sigma'$ can be interpreted as a $\mathcal K \ell(T)$-arrow of the form $\mathbb N \times \sem{\sigma} \to \sem{\sigma'}$.




\if 0
However, a field in a class definition doesn't denote just a single 
getter/setter coalgebra, but instead an entire family of them, one for each allocated instance of the class.
Since it may be possible to attempt accessing a field that has not been allocated, a field is interpreted as 
$\mathcal K \ell(T)$-morphism of type 
\begin{center}
\begin{tikzcd}
\mathbb N \ar[r] & T(\sem{\tau} \times S^{\sem{\tau}})
\end{tikzcd} 
\end{center} 

Actually, I don't think we want to dereference fields individually. First we want to dereference the object containing
the field and then we want to project the appropriate coalgebra.

Thus, for a class $c$, 
we interpret a field $f : \tau$ as a function which maps a natural number $i$ to the getter/setter coalgebra 
for field $f$ of the 
$i$th allocated instance of class $c$, i.e. an element of the dcppo $((\sem{\tau} \times S^{\sem{\tau}})^S)^{\mathbb N_{\bot}}$ 
%note that above we're not mentioning that this function also has bottom in its domain; I think we can do this by 
%convention to make things easier to read

Likewise, a method $m : \sigma \to \sigma'$ is also interpreted coalgebraically, as an element of the dcppo
$((\sem{\sigma'} \times S)^{\sem{\sigma}})^S$.

Then, given a class definition 
$\lbrack f_i:\tau_i^{~i \in 1..n}, m_i = \lambda x \zeta z. t_i :\sigma_i \to \sigma_i'^{~i \in 1..l}, \cdots \rbrack$, we interpret its list of fields 
as the product of their coalgebraic interpretations:
$$\csem{f_i : \tau_i^{~i \in 1..n}} \doteq (\sem{\tau_1} \times S^{\sem{\tau_1}})^S \times \cdots \times (\sem{\tau_n} \times S^{\sem{\tau_n}})^S$$
and we interpret its list of methods as the product of their coalgberaic interpretations
$$\csem{m_i = \lambda x \zeta z. t_i : \sigma_i \to \sigma_i'^{~i \in 1..l}} \doteq ((\sem{\sigma_1'} \times S)^{\sem{\sigma_1}})^S \times \cdots \times ((\sem{\sigma_l'} \times S)^{\sem{\sigma_l}})^S$$
Then, a class definition is interpreted as the product of these two components:
$$ \csem{~\lbrack f_i:\tau_i^{~i \in 1..n}, m_i:\sigma_i \to \sigma_i'^{~i \in 1..l}, \cdots \rbrack~} \doteq 
  \csem{f_i : \tau_i^{~i \in 1..n}} \times \csem{m_i : \sigma_i \to \sigma_i'^{~i \in 1..l}} $$

\fi





\if 0
Then, given an object 
$\lbrack f_i:\tau_i^{~i \in 1..n}, m_i:\sigma_i \to \sigma_i'^{~i \in 1..l}, \cdots \rbrack$, we interpret its set of fields
as a function which maps each $f_i$ to an element of $(\sem{\tau_i} \times S^{\sem{\tau_i}})^S$; i.e., collectively its fields are interpreted as an element of the dcppo
$$\prod_{f_i \in \{ f_1 \ldots, f_n \}} (\sem{\tau_i} \times S^{\sem{\tau_i}})^S$$
and we interpret its set of methods $m_i:\sigma_i\to\sigma_i'^{~i \in 1..l}$ as an element of the dcppo
$$\prod_{m_i \in \{ m_1 \ldots, m_l \}} ((\sem{\sigma_i'} \times S)^{\sem{\sigma_i}})^S$$
We are now ready to define the interface interpretation for class definitions:\\~\\
\begin{tabular}{ll}
\multicolumn{2}{l}{$\csem{\cdot} : d \to \mbf{dcppo}$} \\
$\csem{~\lbrack f_i:\tau_i^{~i \in 1..n}, m_i:\sigma_i \to \sigma_i'^{~i \in 1..l}, \cdots \rbrack~} \doteq$ & 
  $\{ F \uplus M \mid$ \\ 
  & $~~F \in \prod_{f_i \in \{ f_1, \ldots, f_n \}} (\sem{\tau_i} \times S^{\sem{\tau_i}})^S~~\wedge$ \\
  & $~~M \in \prod_{m_i \in \{ m_1, \ldots, m_l \}} ((\sem{\sigma_i'} \times S)^{\sem{\sigma_i}})^S \}$
\end{tabular}\\~\\
Here $F \uplus M$ means $F \cup M$, but with the plus symbol emphasizing that the two components $F$ and $M$ are disjoint.

We also need to take a closure as an argument though, so the above interpreation is wrong. Also I can't use common 
isomorphisms to pull the S exponents out.

\fi



%$\csem{f : \tau} \doteq ((\sem{\tau} \times S^{\sem{\tau}})^S)^{\mathbb N_{\bot}}$\\
%$\csem{m : \sigma \to \sigma'} \doteq (((\sem{\sigma'} \times S)^{\sem{\sigma}})^S)^{\mathbb N_{\bot}}$\\~\\
%$\csem{c_i = [f_{ij} : \tau_{ij}^{~j \in 1..n}, m_{ij} : \sigma_{ij} \to \sigma_{ij}'^{~j \in 1..l}, \cdots]^{i \in 1..k}} \doteq$\\
%$~~~~~((\sem{\tau_1} \times S^{\sem{\tau_1}} \times \cdots \times \sem{\tau_n} \times S^{\sem{\tau_n}}
%\times (\sem{\sigma_1} \times S)^{\sem{\sigma_1'}} \times \cdots \times (\sem{\sigma_l} \times S)^{\sem{\sigma_l'}})^S)^{\mathbb N_{\bot}}$


% $~~~~~((\sem{\tau_1} \times S^{\sem{\tau_1}} \times \cdots \times \sem{\tau_l} \times S^{\sem{\tau_l}}
% \times (\sem{\sigma_1} \times S)^{\sem{\sigma_1'}} \times \cdots \times (\sem{\sigma_l} \times S)^{\sem{\sigma_l'}})^S)^{\mathbb N_{\bot}}$





\end{document}
