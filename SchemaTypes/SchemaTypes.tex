\documentclass{article}
 
\usepackage{amscd}
\usepackage{graphicx}
\usepackage{amsmath}
\usepackage{amssymb}
\usepackage{amsthm}
\usepackage{stmaryrd}
\usepackage{mathpartir}
\usepackage{multicol}
\usepackage{enumitem}
\usepackage[section]{placeins} % float barriers
\usepackage{natbib}
\usepackage{xcolor} 
\usepackage{bussproofs} 
\usepackage{diagrams}
\usepackage{tikz}
\usepackage{tabu}
\usepackage{quiver} 

\usetikzlibrary{cd}
 
\newtheorem{lemma}{Lemma}

%example: \limit{j \in J}{F_j}
\newcommand{\limit}[2]{\underset{\overset{\longleftarrow}{#1}}{\text{lim}}~#2}
\newcommand{\lims}[1]{\underset{\longleftarrow}{\text{lim}}~#1}
\newcommand{\mbf}{\mathbf}
\newcommand{\sem}[1]{\llbracket #1 \rrbracket}
\newcommand{\defeq}{\overset{\mathit{def}}{=}}


\newcommand{\vrt}[2]{
\pile{
#1 \\
\downarrow \\
#2
}
}

\newcommand{\sdisp}[1]{
\left( #1 \right)
}

\newcommand{\ddisp}[3]{
\left(
\scriptsize
\begin{tikzcd}
#1 \ar[d, "\footnotesize{#2}"] \\
#3
\end{tikzcd}
\normalsize
\right)
}

\newcommand{\disp}[3]{
\left(
\tiny
\begin{array}{c}
#1 \\
\downarrow\\
#3
\end{array}
\begin{array}{l}
~ \\
#2 \\
~
\end{array}
\normalsize
\right)
}

\newcommand{\dispp}[3]{
\tiny
\begin{tikzcd}
#1 \ar[d, "#2"] \\
#3
\end{tikzcd}
\normalsize
}

% give tables some extra space between rows and columns
\renewcommand{\arraystretch}{1.4}

\title{Schema Type Calculus}

\begin{document}

\maketitle

\section*{Syntax}

\begin{tabular}{llll}
$\mathit{Chars}$ & $\doteq$ & the set of all characters \\
$\mathit{TypeVars}$ & $\doteq$ & the set of all type variables \\
$\mathit{StringVars}$ & $\doteq$ & the set of all string variables \\
$\alpha$ & $\in$ & $\mathit{TypeVars}$ & ~ \\
$\zeta$ & $\in$ & $\mathit{StringVars}$ & ~ \\
$c$ & $\in$ & $\mathit{Chars}$ & (characters) \\
$\phi$ & $\in$ & $\mathcal P(\mathit{Chars})$ & (subsets of the set of characters) \\
$s,t$ & $\in$ & $\mathit{Strings}$ & (where $\mathit{Strings} = \mathit{Chars}^{\star}$) \\~\\
$S$ (string index) & $::=$ & $s$ & (literal string) \\
                   & $\mid$ & $\zeta$ & (string variable) \\~\\
$d$ (discriminator) & $::=$  & $\mathit{Prefix}~S$ & (values that begin with $S$) \\
    & $\mid$ & $\mathit{Literal}~S$ & (values that are equal to $S$) \\
 & & \\
$m$ (multiplicity) & $::=$ & $\mathit{one} \mid \mathit{set} \mid \mathit{opt}$\\~\\
$\tau$ (type) & $::=$ & $\mathit{tuple} \{ s_i : \tau_i : c_i^{~i \in 1..n} \}$ & (pieced string) \\
       & $\mid$ & $s$ & (string literal type) \\
       & $\mid$ & $(S_1,\ldots,S_n).[\phi]$ & (path type) \\
       & $\mid$ & $\biguplus \tau_i^{~i \in 1..n}$ & (union of disjoint types) \\ 
       & $\mid$ & $\tau.\{[\tau_i] :^{m_i} \tau_i ^{~i \in 1..n}\}$ & (record type - $\mathit{root}.\{ fields \}$) \\ 
       & $\mid$ & $\lambda (\alpha :: \kappa). \tau$ & (type abstraction) \\
       & $\mid$ & $\lambda \zeta. \tau$ & (index abstraction) \\
       & $\mid$ & $\tau~[S]$ & (index application) \\
       & $\mid$ & $\tau~\tau$ & (type application) \\
       & $\mid$ & $\alpha$ & (type variable) \\~\\
$\kappa, \iota$ (kind) & $::=$ & $\mathit{KString}(d,\phi)$ & (Kind of string types, chars drawn from $\phi$) \\
                         & $\mid$ & $\mathit{KTree}(d)$ & (Kind of tree types w/ root discriminator $d$) \\
                         & $\mid$ & $\kappa \overset{A}{\Rightarrow} \kappa$ & (Kind of type abstractions) \\
                         & $\mid$ & $\forall \zeta. \kappa$ & (Kind of string abstractions) \\~\\
\end{tabular}\\~\\~\\
\begin{tabular}{llll}
$\Sigma$ (index context) & $::=$ & $\Sigma,\zeta$ & (context extension) \\
                         & $\mid$ & $\emptyset$ & (empty context) \\~\\
$\Gamma$ (kind context) & $::=$ & $\Gamma,x :: \kappa$ & (context extension) \\
                           & $\mid$ & $\emptyset$ & (empty context) \\~\\
$A$ (coeffect) & $\subseteq$ & $\mathit{String}^\star \times \mathcal P(\mathit{Chars})$ & ~ 
\end{tabular}

\section*{Index well-formedness}

\begin{mathpar}
\inferrule[IWF-Var]
  {\zeta \in \Sigma}
  {\Sigma \vdash \zeta}
\and
\inferrule[IWF-Lit]
  {~}
  {\Sigma \vdash s}
\end{mathpar}
The interpretation $\sem{\Sigma}$ of an index context $\Sigma = \zeta_1, \ldots, \zeta_n$ is the set of all functions from $\{ \zeta_1, \ldots, \zeta_n \}$ to the set of strings. The interpretation $\sem{\Sigma \vdash S}$ of a well-formed index judgment is a $\sem{\Sigma}$-indexed family of strings, i.e., viewing the set $\mathit{Strings}$ as a discrete category, it is an object of $\mathit{Fam}(\mathit{Strings})_{\sem{\Sigma}}$ defined as follows:
\begin{mathpar}
\sem{\Sigma \vdash \zeta} \doteq (\sigma(\zeta))_{\sigma \in \sem{\Sigma}}
\and
\sem{\Sigma \vdash s} \doteq (s)_{\sigma \in \sem{\Sigma}}
\end{mathpar}

\section*{Discriminator well-formedness}

\begin{mathpar}
\inferrule[DWF-Prefix]
  {\Sigma \vdash S}
  {\Sigma \vdash \mathit{Prefix}~S}
\and
\inferrule[DWF-Lit]
  {\Sigma \vdash S}
  {\Sigma \vdash \mathit{Literal}~S}
\end{mathpar}

The interpretation $\sem{\Sigma \vdash d}$ of a discriminator well-formedness judgment is a $\sem{\Sigma}$-indexed family of sets of strings, i.e., viewing $\mathcal P(\mathit{Strings})$ as a thin category ordered by set inclusion, $\sem{\Sigma \vdash d}$ is an object of $\mathit{Fam}(\mathcal P(\mathit{Strings}))_{\sem{\Sigma}}$. 
\begin{mathpar}
\sem{\Sigma \vdash \mathit{Prefix}~S} \doteq (\{ t \mid \sem{\Sigma \vdash S}_\sigma \text{ is a prefix of } t \})_{\sigma \in \sem{\Sigma}}
\and
\sem{\Sigma \vdash \mathit{Literal}~S} \doteq (\{ \sem{\Sigma \vdash S}_\sigma \})_{\sigma \in \sem{\Sigma}}
\end{mathpar}

\section*{Discriminator subsumption}

\begin{mathpar}
\inferrule[SD-PrefPref]
  {s' \sqsubseteq s}
  {\Sigma \vdash \mathit{Prefix}~s \leq \mathit{Prefix}~s'}
\and
\inferrule[SD-LitLit]
  {\Sigma \vdash S}
  {\Sigma \vdash \mathit{Literal}~S \leq \mathit{Literal}~S}
\and
\inferrule[SD-LitPref]
  {s' \sqsubseteq s}
  {\Sigma \vdash \mathit{Literal}~s \leq \mathit{Prefix}~s'}
\and
\end{mathpar}

Above, $s' \sqsubseteq s$ means ``$s'$ is a prefix of $s$''. 
The interpretation $\sem{\Sigma \vdash d \leq d'}$ of a subsumption judgment is an arrow of $\mathit{Fam}(\mathcal P(\mathit{Strings}))_{\sem{\Sigma}}$ with domain $\sem{\Sigma \vdash d}$ and codomain $\sem{\Sigma \vdash d'}$.

\section*{Multiplicities}

A multiplicity is intepreted as a subset of $\mathbb N$.\\~\\
\begin{tabular}{lll}
$\sem{one}$ & $\doteq$ & $\{ 1 \}$ \\
$\sem{opt}$ & $\doteq$ & $\{ 0, 1 \}$ \\
$\sem{set}$ & $\doteq$ & $\mathbb N$
\end{tabular}
\section*{Database instances}

Given lists of strings $\rho$ and $\theta$, we write $\rho \cdot \theta$ for the list obtained by concatenating $\theta$ to the right of $\rho$. A \emph{database instance} is a partial function $f : \mathit{Strings}^{\star} \rightharpoonup \mathit{Strings}$ from lists of strings to strings. We write $\mathit{Inst}$ for the set of database instances. For a database instance $f$ and a list of strings $\rho$, we write $f \! \mid_\rho$ for the database instance defined such that for all lists of strings $\theta$ we have $$f \! \mid_\rho \! (\theta) \doteq f(\rho \cdot \theta)$$ We write the list of strings consisting of $s_1, \ldots, s_n$ as $[s_1,\ldots,s_n]$. Given a set of strings $X$ and a database instance $f$ we define the notation $$f \# X \doteq | \{ x \in X \mid f([x]) \text{ is defined } \} |$$ Furthermore, we define $$\mathit{keys}(f) \doteq \{ s \in \mathit{Strings} \mid f([s]) \text{ is defined} \}$$

\section*{Coeffect Well-formedness}

The judgment $\Sigma \vdash A$ means that the coeffect $A$ is well-formed under index context $\Sigma$,
in the sense that for each $(S_1,\ldots,S_n).[\phi] \in A$ we have $\Sigma \vdash S_i^{~i \in 1..n}$.

The interpretation $\sem{\Sigma \vdash A}$ is a $\sem{\Sigma}$-indexed family of posets of database instances, defined as $$\sem{\Sigma \vdash A} \doteq (I_\sigma, \leq_\sigma)_{\sigma \in \sem{\Sigma}}$$ where $$I_\sigma \doteq \sdisp{\bigcap_{(S_1,\ldots,S_n).[\phi] \in A} \{ f \in \mathit{Inst} \mid \forall s \in \mathit{keys}(f \! \mid_{\sigma S_1, \ldots, \sigma S_n}).~\mathit{chars}(s) \subseteq \phi \}}_{\sigma \in \sem{\Sigma}}$$ and for $f,g \in I_{\sigma}$ we have
$$f \leq_\sigma g \overset{\cdot}{\Leftrightarrow} (\forall \rho \in \mathit{Strings}^\star.~f(\rho) \text{ is defined} \Rightarrow g(\rho) \text{ is defined})$$


\section*{Fibred graded comonads}

For each $\Sigma$, we define a graded comonad $D_{\Sigma}$ on the category $\mathit{Fam}(\mbf{Posets})_{\sem{\Sigma}}$ of $\sem{\Sigma}$-indexed families of posets. The scalar coeffect structure is $$( \{ A \mid \Sigma \vdash A \}, \cup, \cup, \emptyset, \emptyset, \subseteq)$$ For objects $(X_\sigma)_{\sigma \in \sem{\Sigma}}$ of the fibre $\mathit{Fam}(\mbf{Posets})_{\sem{\Sigma}}$, we define $$D_{\Sigma} A(~(X_\sigma)_{\sigma \in \sem{\Sigma}}~) \doteq (X_\sigma \times \sem{\Sigma \vdash A}_\sigma)_{\sigma \in \sem{\Sigma}}$$ For arrows $(f_\sigma : X_\sigma \to Y_\sigma)_{\sigma \in \sem{\Sigma}}$ of the fibre $\mathit{Fam}(\mbf{Posets})_{\sem{\Sigma}}$ we define $$D_{\Sigma} A(~(f_\sigma)_{\sigma \in \sem{\Sigma}}~) \doteq (~(x_\sigma, a_\sigma) \mapsto (f_\sigma(x_\sigma),a_\sigma)~)_{\sigma \in \sem{\Sigma}}$$ Its counit $\epsilon$ is defined as $$\epsilon_X(x) \doteq (x,\mathit{Inst})$$ Its comultiplication $\delta$, for coeffect scalars $A$ and $B$, is defined as $$\delta_{A,B}((x,a),b) \doteq (x, a \cap b)$$ For string contexts $\Sigma$ and coeffect scalars $\Sigma \vdash A$ and $\Sigma \vdash B$ with $A \subseteq B$ we have a natural transformation $\mathit{sub}_{A,B} : D_{\Sigma} B \to D_{\Sigma } A$, defined as the family of inclusions $(\sem{\Sigma \vdash B}_\sigma \subseteq \sem{\Sigma \vdash A}_\sigma)_{\sigma \in \sem{\Sigma}}$.\\~\\
For coeffect scalars $\Sigma \vdash A$ and kinding contexts $\Sigma \vdash \Gamma,x :: \kappa$, we have a natural isomorphism $$m_{\Sigma,A,\Gamma,\kappa} : D_{\Sigma} \emptyset(\sem{\Sigma \vdash \Gamma}) \times D_{\Sigma} A( \sem{\Sigma \vdash \kappa} ) \overset{\sim}{\to} D_{\Sigma} A \sem{\Sigma \vdash \Gamma,x :: \kappa}$$ defined to discard the copy of $\sem{\Sigma \vdash \emptyset}$ produced by applying $D_{\Sigma} \emptyset$ and then reassociate.

\section*{Substitution Functors}

Given a function $u : I \to J$, we write $u^* : \mathit{Fam}(\mbf{Posets})_J \to \mathit{Fam}(\mbf{Posets})_I$ for the substitution functor induced by $u$; i.e., for an $J$-indexed family of posets $\sdisp{P_j}_{j \in J}$ we have $$u^* (\sdisp{P_j}_{j \in J}) = \sdisp{ P_{u(i)} }_{i \in I}$$ and for a $J$-indexed family of monotone functions $(f_j : P_j \to Q_j)_{j \in J}$ we have $$u^*(\sdisp{f_j}_{j \in J}) \doteq \sdisp{f_{u(i)}}_{i \in I}$$

We overload the notation $u^*$ to other forms of set-indexed families. For example, we can also use $u^* : \mathit{Fam}(\mathit{Strings})_J \to \mathit{Fam}(\mathit{Strings})_I$ to reindex $J$-indexed families of strings, considering the set $\mathit{Strings}$ as a discrete category. 

\section*{Simple Products}

In what follows, let $\pi_{\Sigma,\zeta} : \sem{\Sigma,\zeta} \to \sem{\Sigma}$ denote the projection function which discards the mapping entry corresponding to $\zeta$. The category $\mathit{Fam}(\mbf{Posets})$ has \emph{simple products}, meaning that the substitution functor $\pi_{\Sigma,\zeta}^* : \mathit{Fam}(\mbf{Posets})_{\sem{\Sigma}} \to \mathit{Fam}(\mbf{Posets})_{\sem{\Sigma,\zeta}}$ has a right adjoint $\Pi_{\Sigma,\zeta} : \mathit{Fam}(\mbf{Posets})_{\sem{\Sigma,\zeta}} \to \mathit{Fam}(\mbf{Posets})_{\sem{\Sigma}}$ such that for functions $f : \sem{\Sigma'} \to \sem{\Sigma}$, in the diagram

\[\begin{tikzcd}
	{\mathit{Fam}(\mathbf{Posets})_{\sem{\Sigma}}} && {\mathit{Fam}(\mathbf{Posets})_{\sem{\Sigma'}}} \\
	\\
	{\mathit{Fam}(\mathbf{Posets})_{\sem{\Sigma,\zeta}}} && {\mathit{Fam}(\mathbf{Posets})_{\sem{\Sigma',\zeta}}}
	\arrow["{\pi_{\sem{\Sigma,\zeta}}^* }"', curve={height=18pt}, from=1-1, to=3-1]
	\arrow["{\Pi_{\sem{\Sigma,\zeta}}}"', curve={height=18pt}, from=3-1, to=1-1]
	\arrow["{f^*}", from=1-1, to=1-3]
	\arrow["{(f \times \mathit{id})^*}" below, from=3-1, to=3-3]
	\arrow["{\pi^*_{\sem{\Sigma',\zeta}}}"', curve={height=18pt}, from=1-3, to=3-3]
	\arrow["{\Pi_{\sem{\Sigma',\zeta}}}"', curve={height=18pt}, from=3-3, to=1-3]
\end{tikzcd}\]
the canonical transformation $f^* \Pi_{\sem{\Sigma,\zeta}} \Rightarrow \Pi_{\sem{\Sigma',\zeta}} (f \times \mathit{id})^*$ is an isomorphism.

The \emph{canonical transformation} is obtained as follows. First, from the well known fact that for all fibrations we have $u^*v^* \cong (u;v)^*$ we obtain $$(f \times \mathit{id})^*\pi^*_{\sem{\Sigma,\zeta}} \cong ((f \times \mathit{id});\pi_{\sem{\Sigma,\zeta}})^* = (\langle \pi_{\Sigma',\zeta} f, \pi'_{\Sigma',\zeta} \mathit{id} \rangle;\pi_{\sem{\Sigma,\zeta}})^* = (\pi_{\Sigma',\zeta};f)^* = \pi_{\sem{\Sigma',\zeta}}^* f^*$$
Then, the canonical transformation is the transpose of 
$$\pi^* f^* \Pi \overset{\cong}{\longrightarrow} (f \times \mathit{id})^* \pi^* \Pi \overset{(f \times \mathit{id})^* \epsilon}{\longrightarrow} (f \times \mathit{id})^*$$

In the specific case of $\mathit{Fam}(\mathbf{Posets})$, for sets $I$ and $J$ we define $\Pi_{(I,J)} : \mathit{Fam}(\mathbf{Posets})_{I \times J} \to \mathit{Fam}(\mathbf{Posets})_I$ by
$$(P_{(i,j)})_{(i,j) \in I \times J} \mapsto (\Pi_{j \in J} Y_{(i,j)})_{i \in I}$$  
where for $i \in I$, $\Pi_{j \in J} Y_{(i,j)}$ is a $J$-indexed product of posets.

TODO: finish this... some of it can be found in Jacobs Lemma 1.9.2.

\section*{Kind well-formedness}

\begin{mathpar}
\inferrule[KWF-Str]
  {\Sigma \vdash d}
  {\Sigma \vdash \mathit{KString(d,\phi)}}
\and
\inferrule[KWF-Tree]
  {\Sigma \vdash d}
  {\Sigma \vdash \mathit{KTree(d)}}
\and
\inferrule[KWF-TyFun]
  {\Sigma \vdash \kappa \\ \Sigma \vdash \kappa' \\ \Sigma \vdash A}
  {\Sigma \vdash \kappa \overset{A}{\Rightarrow} \kappa'}
\and
\inferrule[KWF-StrFun]
  {\Sigma,\zeta \vdash \kappa}
  {\Sigma \vdash \forall \zeta. \kappa}
\end{mathpar}

The interpretation $\sem{\Sigma \vdash \kappa}$ of a kind well-formedness judgment is a $\sem{\Sigma}$-indexed family of posets, defined as follows.

\begin{tabular}{lll}
$\sem{\Sigma \vdash \mathit{KString}(d,\phi)}$ & $\doteq \sdisp{~(\mathcal P(X_\sigma), \subseteq)~}_{\sigma \in \sem{\Sigma}}$ & $\text{where } X_\sigma \doteq \{ x \in \sem{\Sigma \vdash d}_\sigma \mid \mathit{chars}(x) \subseteq \phi \}$ \\
$\sem{\Sigma \vdash \mathit{KTree}(d)}$ & $\doteq \sdisp{~(\mathcal P(Y_\sigma), \subseteq)~}_{\sigma \in \sem{\Sigma}}$ & $\text{where } Y_\sigma \doteq \{ f \in \mathit{Inst} \mid f(\epsilon) \in \sem{\Sigma \vdash d}_{\sigma} \}$\\
$\sem{\Sigma \vdash \kappa \overset{A}{\Rightarrow} \kappa'}$ & $\doteq D_{\Sigma} A(\sem{\Sigma \vdash \kappa}) \Rightarrow \sem{\Sigma \vdash \kappa'}$\\
$\sem{\Sigma \vdash \forall \zeta. \kappa}$ & $\doteq \Pi_{\sem{\Sigma},\sem{\zeta}} \sem{\Sigma,\zeta \vdash \kappa} $ & ~
\end{tabular}

\section*{Subkinding}

The interpretation $\sem{\Sigma \vdash \kappa <:: \kappa'}$ is a mono in the fibre $\mathit{Fam}(\mbf{Posets})_{\sem{\Sigma}}$ from $\sem{\Sigma \vdash \kappa}$ to $\sem{\Sigma \vdash \kappa'}$.

\begin{mathpar}
\inferrule[SK-StrTree]
  {\Sigma \vdash d \leq d'}
  {\Sigma \vdash \mathit{KString}(d,\phi) <:: \mathit{KTree}(d')}
\and
\inferrule[SK-StrStr]
  {\Sigma \vdash d \leq d' \\ \phi \subseteq \phi'}
  {\Sigma \vdash \mathit{KString}(d,\phi) <:: \mathit{KString}(d',\phi')}
\and
\inferrule[SK-TreeTree]
  {\Sigma \vdash d \leq d'}
  {\Sigma \vdash \mathit{KTree}(d) <:: \mathit{KTree}(d')}
\end{mathpar}

\section*{Kinding}

In what follows, we define $\ast \doteq \mathit{KTree}(\mathit{Prefix}~\epsilon)$, where $\epsilon$ is the empty string, so that all proper types have kind $\ast$. We define $\star \doteq \mathit{KString}(\mathit{Prefix}~\epsilon, \mathit{Chars})$, so that all string types have kind $\star$.

\begin{mathpar}
\inferrule[Tuple]
  {\Sigma \mid \Gamma@A_i \vdash \tau_i :: \kappa_i^{~i \in 1..n} \\ 
   \Sigma \vdash \kappa_i <:: \mathit{KString}(d_i,\phi_i)^{~i \in 1..n} \\
   c_i \not \in \phi_i^{~i \in 1..n} }
  {\Sigma \mid \Gamma@(\bigcup_{i \in 1..n} A_i) \vdash \mathit{tuple} \{ s_i : \tau_i : c_i^{~i \in 1..n}\} :: \mathit{KString}(d_1, (\bigcup_{i \in 1..n} \phi_i) \cup \{ c_1, \ldots, c_n \})}
\and
\inferrule[DisjUnion]
  {\Sigma \mid \Gamma@A_i \vdash \tau_i :: \mathit{KTree}(d_i)^{~i \in 1..n} \\ \mathit{disjoint} \{ d_i^{~i \in 1..n} \} }
  {\Sigma \mid \Gamma@(\bigcup_{i \in 1..n} A_i) \vdash \biguplus_{i \in 1..n} \tau_i :: \mathit{KTree}(\mathit{merge}~\{ d_i^{~i \in 1..n} \})}
\and
\inferrule[Record]
  {\Sigma \mid \Gamma@A \vdash \tau :: \mathit{KString}(d,\phi) \\ 
    \Sigma \mid \Gamma@B_i \vdash \tau_i :: \iota_i^{~i \in 1..n} \\ \Sigma \vdash \iota_i <:: \star^{~i \in 1..n} \\ 
    \Sigma \mid \Gamma@A_i \vdash \tau_i' :: \kappa_i^{~i \in 1..n} \\ \Sigma \vdash \kappa_i <:: \ast^{~i \in 1..n}}
  {\Sigma \mid \Gamma@(A \cup \bigcup_{i \in 1..n} A_i \cup B_i) \vdash \tau.\{[\tau_i] :^{m_i} \tau_i'^{~i \in 1..n}\} :: \mathit{KTree}(d)}
\and
\inferrule[Path]
  {\Sigma \vdash S_i^{~i \in 1..n}}
  {\Sigma \mid \Gamma@(\{ (S_1,\ldots,S_n).[\phi] \} \cup A) \vdash (S_1,\ldots,S_n).[\phi] :: \mathit{KString}(\mathit{Prefix}~\epsilon,\phi)}
\and
\inferrule[StrAbs]
  {\Sigma, \zeta \mid \Gamma @ \emptyset \vdash \tau :: \kappa}
  {\Sigma \mid \Gamma @ \emptyset \vdash \lambda \zeta. \tau :: \forall \zeta. \kappa}
\and
\inferrule[TyAbs]
  {\Sigma \mid \Gamma,\alpha :: \kappa @ A \vdash \tau :: \kappa'}
  {\Sigma \mid \Gamma @ \emptyset \vdash \lambda (\alpha :: \kappa).\tau :: \kappa \overset{A}{\Rightarrow} \kappa'}
\and
\inferrule[TyApp]
  {\Sigma \mid \Gamma @ A \vdash \tau :: \kappa \overset{C}{\Rightarrow} \iota \\ 
   \Sigma \mid \Gamma @ B \vdash \tau' :: \kappa}
  {\Sigma \mid \Gamma @ (A \cup B \cup C) \vdash \tau~\tau' :: \iota}
\and
\inferrule[StrApp]
  {\Sigma \mid \Gamma @ A \vdash \tau :: \forall \zeta. \kappa \\ \Sigma \vdash S}
  {\Sigma \mid \Gamma @ A \vdash \tau~[S] :: \kappa[S/\zeta]}

\end{mathpar}
We write $\Sigma \vdash \Gamma$ to mean that $\Sigma \vdash \kappa$ for all $(x : \kappa) \in \Gamma$.
When $\Gamma = x_1 : \kappa_1, \ldots, x_n : \kappa_n$, we define the interpretation $\sem{\Sigma \vdash \Gamma} \doteq \sem{\Sigma \vdash \kappa_1} \times \cdots \times \sem{\Sigma \vdash \kappa_n}$, where products in $\mathit{Fam}(\mbf{Posets})$ are taken pointwise.
We will prove that $\Sigma \mid \Gamma @ A \vdash \tau :: \kappa$ implies that $\Sigma \vdash \Gamma$, $\Sigma \vdash \kappa$, and $\Sigma \vdash A$. The interpretation $$\sem{\Sigma \mid \Gamma @ A \vdash \tau :: \kappa}$$ of a kinding judgment is a morphism of the fibre category $\mathit{Fam}(\mbf{Posets})_{\sem{\Sigma}}$ from $D_{\Sigma} A(\sem{\Sigma \vdash \Gamma})$ to $\sem{\Sigma \vdash \kappa}$.

\begin{mathpar}
%\inferrule[Tuple]
%  {\sem{\Sigma \mid \Gamma@A_i \vdash \tau_i :: \kappa_i} = f_i : D_{\Sigma} A_i(\sem{\Sigma \vdash \Gamma}) \to \sem{\Sigma \vdash \kappa_i}^{~i \in 1..n} \\ 
%   \sem{\Sigma \vdash \kappa_i <:: \mathit{KString}(d_i,\phi_i)} = g_i : \sem{\Sigma \vdash \kappa_i} \to \sem{\Sigma \vdash \mathit{KString}(d_i,\phi_i)}^{~i \in 1..n} \\
%   c_i \not \in \phi_i^{~i \in 1..n} }
%  {\Sigma \mid \Gamma@(\bigcup_{i \in 1..n} A_i) \vdash \mathit{tuple} \{ s_i : \tau_i : c_i^{~i \in 1..n}\} :: \mathit{KString}(d_1, (\bigcup_{i \in 1..n} \phi_i) \cup \{ c_1, \ldots, c_n \})}
%\and
%\inferrule[DisjUnion]
%  {\Sigma \mid \Gamma@A_i \vdash \tau_i :: \mathit{KTree}(d_i)^{~i \in 1..n} \\ \mathit{disjoint} \{ d_i^{~i \in 1..n} \} }
%  {\Sigma \mid \Gamma@(\bigcup_{i \in 1..n} A_i) \vdash \biguplus_{i \in 1..n} \tau_i :: \mathit{KTree}(\mathit{merge}~\{ d_i^{~i \in 1..n} \})}
%\and
\inferrule[Record]
  {\sem{\Sigma \mid \Gamma@A \vdash \tau :: \mathit{KString}(d,\phi)} = f \\ 
   \sem{\Sigma \mid \Gamma@B_i \vdash \tau_i :: \iota_i} = g_i^{~i \in 1..n} \\ \sem{\Sigma \vdash \iota_i <:: \star} = m_i^{~i \in 1..n} \\ 
    \sem{\Sigma \mid \Gamma@A_i \vdash \tau_i' :: \kappa_i} = h_i^{~i \in 1..n} \\ \sem{\Sigma \vdash \kappa_i <:: \ast} = n_i^{~i \in 1..n}}
  {\sem{\Sigma \mid \Gamma@(A \cup \bigcup_{i \in 1..n} A_i \cup B_i) \vdash \tau.\{[\tau_i] :^{m_i} \tau_i'^{~i \in 1..n}\} :: \mathit{KTree}(d)}_\sigma (\gamma,a) \doteq \\ \{ g \in \mathit{Inst} \mid (g(\epsilon) \in f(\gamma,a)) \wedge \\ (\forall s \in (g_i;m_i)(\gamma,a).~g \! \mid_{[s]} \in (h_i;n_i)(\gamma,a))^{i \in 1 ..n} \wedge \\ (g \# (g_i;m_i)(\gamma,a) \in \sem{m_i})^{i \in 1..n} \} }
\and
\inferrule[Path]
  {\sem{\Sigma \vdash S_i} = f_i^{~i \in 1..n}}
  {\sem{\Sigma \mid \Gamma@(\{ (S_1,\ldots,S_n).[\phi] \} \cup A) \vdash (S_1,\ldots,S_n).[\phi] :: \mathit{KString}(\mathit{Prefix}~\epsilon,\phi)}(\gamma,a) \doteq \\ \{ s \in \mathit{Strings} \mid a([f_1(\gamma),\ldots,f_n(\gamma),s]) \text{ is defined } \}}
\and
\inferrule[StrAbs]
  {\sem{\Sigma, \zeta \mid \Gamma @ \emptyset \vdash \tau :: \kappa} = f : D_{\Sigma,\zeta}\emptyset(\sem{\Sigma,\zeta \vdash \Gamma}) \to \sem{\Sigma,\zeta \vdash \kappa}}
  {\sem{\Sigma \mid \Gamma @ \emptyset \vdash \lambda \zeta. \tau :: \forall \zeta. \kappa} \doteq f^\flat : D_{\Sigma} \emptyset(\sem{\Sigma \vdash \Gamma}) \to \sem{\Sigma \vdash \forall \zeta. \kappa)}}
\and
\inferrule[TyAbs]
  {\sem{\Sigma \mid \Gamma,\alpha :: \kappa @ A \vdash \tau :: \kappa'} = f : D_{\Sigma} A(\sem{\Sigma \vdash \Gamma,\alpha :: \kappa}) \to \sem{\Sigma \vdash \kappa'}}
  {\sem{\Sigma \mid \Gamma @ \emptyset \vdash \lambda (\alpha :: \kappa).\tau :: \kappa \overset{A}{\Rightarrow} \kappa'} \doteq \Lambda(m_{\Sigma,A,\Gamma,\kappa};f) : D_{\Sigma} \emptyset(\sem{\Sigma \vdash \Gamma}) \to \sem{\Sigma \vdash \kappa \overset{A}{\Rightarrow} \kappa'}}
\and
\inferrule[TyApp]
  {\Sigma \mid \Gamma @ A \vdash \tau :: \kappa \overset{C}{\Rightarrow} \iota \\ 
   \Sigma \mid \Gamma @ B \vdash \tau' :: \kappa}
  {\Sigma \mid \Gamma @ (A \cup B \cup C) \vdash \tau~\tau' :: \iota}
\and
\inferrule[StrApp]
  {\sem{\Sigma \mid \Gamma @ A \vdash \tau :: \forall \zeta. \kappa} = f : D_{\Sigma}A(\sem{\Sigma \vdash \Gamma}) \to \sem{\Sigma \vdash \forall \zeta. \kappa} \\ \sem{\Sigma \vdash S} = g : \sem{\Sigma} \to \mathit{Strings}}
  {\sem{\Sigma \mid \Gamma @ A \vdash \tau~[S] :: \kappa[S/\zeta]} \doteq \langle id_{\sem{\Sigma}}, g \rangle^*(f^\sharp) : D_{\Sigma} A(\sem{\Sigma \vdash \Gamma}) \to \langle id_{\Sigma}, g \rangle^* \sem{\Sigma,\zeta \vdash \kappa}}

\end{mathpar}

\subsection*{Explanation}

\subsubsection*{\sc{StrAbs}}
For the string abstraction rule (second-to-last) is essentially mapping across an adjunction, but we need the following lemma to see this.
\begin{lemma}
$\pi^*_{\Sigma,\zeta}(D_{\Sigma} \emptyset(\sem{\Sigma \vdash \Gamma})) = D_{\Sigma,\zeta} \emptyset(\sem{\Sigma,\zeta \vdash \Gamma})$
\end{lemma}

\subsubsection*{\sc{StrApp}}

The domain of the conclusion should be $D_{\Sigma}A(\sem{\Sigma \vdash \Gamma})$. The domain of $\langle \mathit{id}_{\Sigma}, g \rangle^*(f^\sharp)$ is $\langle \mathit{id}_{\sem{\Sigma}}, g \rangle^* \pi_{\Sigma,\zeta}^*(D_{\Sigma} A \sem{\Sigma \vdash \Gamma})$, which is equal to $D_{\Sigma}A(\sem{\Sigma \vdash \Gamma})$ due to the following calculation.\\~\\
\begin{tabu}{l}
$\langle \mathit{id}_{\sem{\Sigma}}, g \rangle^* \pi_{\Sigma,\zeta}^* D_{\Sigma} A (\sem{\Sigma \vdash \Gamma})$ \\
$= (\langle \mathit{id}_{\sem{\Sigma}}, g \rangle; \pi_{\Sigma,\zeta})^* D_{\Sigma} A (\sem{\Sigma \vdash \Gamma})$ \\
$= id_{\sem{\Sigma}}^* D_{\Sigma} A (\sem{\Sigma \vdash \Gamma})$ \\
$= D_{\Sigma} A (\sem{\Sigma \vdash \Gamma})$
\end{tabu}

\subsection*{Index substitution lemmas}

\subsubsection*{Notation}

Given sets $I$ and $I$ and a function $u : I \to J$, we write $u^* : \mathit{Fam}(\mbf{Posets})_J \to \mathit{Fam}(\mbf{Posets})_I$ for the substitution functor induced by $u$, e.g. for an $J$-indexed family of posets $\sdisp{P_j}_{j \in J}$ we have $u^* \sdisp{P_j}_{j \in J} = \sdisp{ P_{u(i)} }_{i \in I}$.

We overload the notation $u^*$ to other forms of set-indexed families. For example, we can also use $u^* : \mathit{Fam}(\mathit{Strings})_J \to \mathit{Fam}(\mathit{Strings})_I$ to reindex $J$-indexed families of strings, considering the set $\mathit{Strings}$ as a discrete category. 

\begin{lemma}
If $\Sigma,\zeta \vdash S$ and $\Sigma \vdash S'$ then $\Sigma \vdash S[S'/\zeta]$. Furthermore, $$\sem{\Sigma \vdash S[S'/\zeta]} = \langle \mathit{id}, \sem{\Sigma \vdash S'} \rangle^* \sem{\Sigma,\zeta \vdash S}$$
\label{SWF-Subst}
\end{lemma}
\begin{proof}
By cases on $S$.

\begin{description}
\item[Case $S = s$]:\\
We have $S[S'/\zeta] = s[S'/\zeta] = s$. By \begin{sc}IWF-Lit\end{sc}, $s$ is a well-formed index term in any index environment, so in particular we have $\Sigma \vdash s$ i.e. $\Sigma \vdash s[S'/\zeta]$.

Semantically, we have $\sem{\Sigma \vdash s} = \langle \mathit{id}, \sem{\Sigma \vdash S'} \rangle^* \sem{\Sigma,\zeta \vdash s}$, since the left-hand side is defined as $$( s )_{\sigma \in \sem{\Sigma}}$$ and the right-hand side is $$\langle \mathit{id}, \sem{\Sigma \vdash S'} \rangle^* ( s )_{\sigma \in \sem{\Sigma,\zeta}}$$ which simplifies to $$ ( s )_{\sigma \in \sem{\Sigma}}$$

\item[Case ]$S = \zeta'$:\\
We have $\Sigma,\zeta \vdash \zeta'$. So either $\zeta' \in \Sigma$ or $\zeta' = \zeta$.

\begin{description}
\item[Case $\zeta' \in \Sigma$]:\\
We have $S[S'/\zeta] = \zeta'$, and since $\zeta' \in \Sigma$, \begin{sc}IWF-Var\end{sc} gives $\Sigma \vdash \zeta'$, i.e. $\Sigma \vdash S[S'/\zeta]$.

Semantically, we have $\sem{\Sigma \vdash \zeta'} = \langle \mathit{id}, \sem{\Sigma \vdash S'} \rangle^* \sem{\Sigma,\zeta \vdash S}$, because the left-hand side is defined as $$\sdisp{\sigma(\zeta')}_{\sigma \in \sem{\Sigma}}$$ while the right-hand side is equal to $$\langle \mathit{id}, \sem{\Sigma \vdash \zeta'} \rangle^* (\sigma(\zeta'))_{\sigma \in \sem{\Sigma,\zeta}}$$ which simplifies to $$(\sigma(\zeta'))_{\sigma \in \sem{\Sigma}}$$
 
\item[Case $\zeta' = \zeta$]:\\
We have $S[S'/\zeta] = S'$. By assumption, $\Sigma \vdash S'$, i.e. $\Sigma \vdash S[S'/\zeta]$.

Semantically, we have $\sem{\Sigma \vdash S'} = \langle \mathit{id}, \sem{\Sigma \vdash S'} \rangle^* \sem{\Sigma,\zeta \vdash \zeta}$ since the right-hand is equal to $$\langle \mathit{id}, \sem{\Sigma \vdash S'} \rangle^* (\sigma(\zeta))_{\sigma \in \sem{\Sigma,\zeta}}$$
which is equal to
$$( \sem{\Sigma,\zeta \vdash \zeta}_{\langle \mathit{id}, \sem{\Sigma \vdash S'} \rangle(\sigma)} )_{\sigma \in \sem{\Sigma}}$$
which is equal to
$$ $$
\end{description}

\end{description}

\end{proof}

\begin{lemma}
If $\Sigma,\zeta \vdash d$ and $\Sigma \vdash S$ then $\Sigma \vdash d[S/\zeta]$
\label{DWF-Subst}
\end{lemma}

\begin{proof}
By cases on $d$

\item[Case $d = \mathit{Prefix}~S'$]:\\
Since $\Sigma,\zeta \vdash \mathit{Prefix}~S'$ we get $\Sigma,\zeta \vdash S'$ from the premise of 
\begin{sc}DWF-Prefix\end{sc}. By Lemma \ref{SWF-Subst} we get $\Sigma \vdash S'[S/\zeta]$. Applying
\begin{sc}DWF-Prefix\end{sc} using the premise $\Sigma \vdash S'[S/\zeta]$ then gives $\Sigma \vdash \mathit{Prefix}~S'[S/\zeta]$, i.e. $\Sigma \vdash d[S/\zeta]$.

\item[Case $d = \mathit{Literal}~S'$]:\\
Similar to the previous case.

\end{proof}

\begin{lemma}
If $\Sigma,\zeta \vdash \kappa$ and $\Sigma \vdash S$ then $\Sigma \vdash \kappa[S/\zeta]$.
Furthermore, if $\Sigma,\zeta \vdash A$ and $\Sigma \vdash S$ then $\Sigma \vdash A[S/\zeta]$.
\end{lemma}

\begin{proof}
By simultaneous induction on the proofs of $\Sigma,\zeta \vdash \kappa$ and $\Sigma,\zeta \vdash A$. 

\begin{description}
\item[Case $\Sigma,\zeta \vdash A$]:~\\
If $\Sigma,\zeta \vdash A$ then for each $(S_1,\ldots,S_n).[\phi] \in A$ we 
have $\Sigma,\zeta \vdash S_i^{~i \in 1..n}$ and $\Sigma,\zeta \vdash \kappa$. By Lemma \ref{SWF-Subst} we have
$\Sigma \vdash S_i[S/\zeta]$. By the inductive hypothesis, we have $\Sigma \vdash \kappa[S/\zeta]$. From this,
we deduce that $\Sigma \vdash (S_1[S/\zeta],\ldots,S_n[S/\zeta]).[\phi[S/\zeta]]$; hence $\Sigma \vdash A[S/\zeta]$. (TODO: should we name the ``rule'' for concluding coeffect well-formedness judgments?)
\item[Case \sc{KWF-Str}]:~\\
We apply Lemma \ref{DWF-Subst} to the premise to get $\Sigma \vdash d[S/\zeta]$. Then applying 
\begin{sc}KWF-Str\end{sc} using $\Sigma \vdash d[S/\zeta]$ as the premise gives 
$\Sigma \vdash \mathit{KString}(d[S/\zeta],\phi)$, i.e. $\Sigma \vdash \kappa[S/\zeta]$.

\item[Case \sc{KWF-Tree}]:~\\
Similar to the above case.

\item[Case \sc{KWF-TyFun}]:~\\
Our inductive hypothesis gives $\Sigma \vdash \kappa[S/\zeta]$, $\Sigma \vdash \kappa'[S/\zeta]$, and
$\Sigma \vdash A[S/\zeta]$. Applying \begin{sc}KWF-TyFun\end{sc} gives $\Sigma \vdash \kappa[S/\zeta] \overset{A[S/\zeta]}{\Rightarrow} \kappa'[S/\zeta]$ i.e. $$\Sigma \vdash (\kappa \overset{A}{\Rightarrow} \kappa')[S/\zeta]$$

\item[Case \sc{KWF-StrFun}]:~\\
The premise for proving $\Sigma,\zeta \vdash \forall \zeta'.\kappa$ is $\Sigma,\zeta,\zeta' \vdash \kappa$. Exchanging $\zeta$ and $\zeta'$ gives $\Sigma,\zeta',\zeta \vdash \kappa$. (The proof should be of the same height.) Applying the IH gives $\Sigma,\zeta' \vdash \kappa[S/\zeta]$. Applying \begin{sc}KWF-StrFun\end{sc} gives $\Sigma \vdash \forall \zeta'. \kappa[S/\zeta]$ i.e.
$$\Sigma \vdash (\forall \zeta'. \kappa)[S/\zeta]$$.
\end{description}
\end{proof}

\begin{lemma}
If $\Sigma,\zeta \vdash d \leq d'$ and $\Sigma \vdash S$ then $\Sigma \vdash d[S/\zeta] \leq d[S/\zeta']$.
\label{SD-Subst}
\end{lemma}

\begin{proof}
One of the following conditions holds. 
\begin{enumerate}
\item Both $d$ and $d'$ are closed, or
\item $d = d' = \mathit{Literal}~\zeta'$ where $\zeta' \in \Sigma$, or
\item $d = d' = \mathit{Literal}~\zeta$
\end{enumerate}

If 1 or 2 then $d[S/\zeta] = d$, $d'[S/\zeta] = d'$, and $\Sigma \vdash d \leq d'$, from which we conclude 
$$\Sigma \vdash d[S/\zeta] \leq d'[S/\zeta]$$

If 3 holds then $d[S/\zeta] = d'[S/\zeta] = \mathit{Literal}~S$. Applying \begin{sc}SD-LitLit\end{sc} gives
$\Sigma \vdash d[S/\zeta] \leq d'[S/\zeta]$. 
\end{proof}

\begin{lemma}
If $\Sigma,\zeta \vdash \kappa <:: \kappa'$ and $\Sigma \vdash S$ then $\Sigma \vdash \kappa[S/\zeta] <:: \kappa'[S/\zeta]$.
\label{SK-Subst}
\end{lemma}

\begin{proof}
By induction on $\Sigma,\zeta \vdash \kappa <:: \kappa'$.

\begin{description}
\item[Case \begin{sc}SK-StrTree\end{sc}]:~\\
Applying lemma \ref{SD-Subst} to the premise $\Sigma,\zeta \vdash d \leq d'$ gives $$\Sigma \vdash d[S/\zeta] \leq d'[S/\zeta]$$ Applying \begin{sc}SK-StrTree\end{sc} gives $$\Sigma \vdash \mathit{KString(d[S/\zeta],\phi) <:: \mathit{KTree}(d'[S/\zeta])}$$
\item[Case \begin{sc}SK-StrStr\end{sc}]:~\\
Applying lemma \ref{SD-Subst} to the premise $\Sigma,\zeta \vdash d \leq d'$ gives $$\Sigma \vdash d[S/\zeta] \leq d'[S/\zeta]$$ Applying \begin{sc}SK-StrStr\end{sc} gives $$\Sigma \vdash \mathit{KString}(d[S/\zeta],\phi) <:: \mathit{KString}(d'[S/\zeta], \phi')$$
\item[Case \begin{sc}SK-TreeTree\end{sc}]:~\\
Applying lemma \ref{SD-Subst} to the premise $\Sigma,\zeta \vdash d \leq d'$ gives $$\Sigma \vdash d[S/\zeta] \leq d'[S/\zeta]$$ Applying \begin{sc}SK-TreeTree\end{sc} gives $$\Sigma \vdash \mathit{KTree}(d[S/\zeta]) <:: \mathit{KTree}(d'[S/\zeta])$$
\end{description}
\end{proof}

\begin{lemma}
If $\Sigma,\zeta \vdash \kappa <:: \star$ and $\Sigma \vdash S$ then
$\Sigma \vdash \kappa[S/\zeta] <:: \star$.
\label{SK-Subst2}
\end{lemma}

\begin{proof}
This lemma is a simple corollary of lemma \ref{SK-Subst}.
\end{proof}


\begin{lemma}
If $\Sigma,\zeta \vdash \kappa <:: \ast$ and $\Sigma \vdash S$ then
$\Sigma \vdash \kappa[S/\zeta] <:: \ast^{~i \in 1..n}$.
\label{SK-Subst3}
\end{lemma}

\begin{proof}
Another simple corollary of lemma \ref{SK-Subst}.
\end{proof}

\subsection*{Kinding index substitution Lemma}

\begin{lemma}
If $$\Sigma,\zeta \mid \Gamma @ A \vdash \tau :: \kappa$$ and $$\Sigma \vdash S$$ then $$\Sigma \mid \Gamma[S/\zeta] @ A[S/\zeta] \vdash \tau[S/\zeta ] :: \kappa [S/\zeta ]$$ and furthermore $$\sem{\Sigma \mid \Gamma[S/\zeta] @ A[S/\zeta] \vdash \tau[S/\zeta] :: \kappa[S/\zeta]} = \langle \mathit{id}, \sem{\Sigma \vdash S} \rangle^* \sem{\Sigma,\zeta \mid \Gamma @ A \vdash \tau :: \kappa}$$
\end{lemma}

\begin{proof}
By induction on the proof $\Sigma,\zeta \mid \Gamma @ A \vdash \tau :: \kappa$.

\begin{description}
\item[Case \sc{Record}]:~\\
The inductive hypothesis gives~\\~\\ 1.) $$\Sigma \mid \Gamma[S/\zeta] @ A[S/\zeta] \vdash \tau[S/\zeta] :: \mathit{KString}([S/\zeta]d,\phi)$$ 2.)
\begin{center}
$\sem{\Sigma \mid \Gamma[S/\zeta]@A[S/\zeta] @ A[S/\zeta] \vdash \tau[S/\zeta] :: KString([S/\zeta]d,\phi)}$\\$= \langle \mathit{id}, \sem{\Sigma \vdash S} \rangle^* \sem{\Sigma,\zeta \mid \Gamma @ A \vdash \tau :: \mathit{KString}(d,\phi)}$
\end{center}
3.)\\
$$\Sigma \mid \Gamma[S/\zeta] @ B_i[S/\zeta] \vdash \tau_i[S/\zeta] :: \iota_i[S/\zeta]^{~i \in 1..n}$$
4.)
\begin{center}
$\sem{\Sigma \mid \Gamma[S/\zeta] @ B_i[S/\zeta] \vdash \tau_i[S/\zeta] :: \iota_i[S/\zeta]}$\\
$= \langle \mathit{id}, \sem{\Sigma \vdash S} \rangle^* \sem{\Sigma,\zeta \mid \Gamma @ B_i \vdash \tau_i :: \iota_i}$
\end{center}
5.)
$$\Sigma \mid \Gamma[S/\zeta ]@ A_i[S/\zeta] \vdash \tau_i'[S/\zeta] :: \kappa_i[S/\zeta]^{~i \in 1..n}$$
Applying Lemma \ref{SK-Subst2} gives
$$\Sigma \vdash \iota_i[S/\zeta] <:: \star^{~i \in 1..n}$$
Applying Lemma \ref{SK-Subst3} gives
$$\Sigma \vdash \kappa_i[S/\zeta] <:: \ast^{~i \in 1..n}$$ 
Applying \begin{sc}Record\end{sc} then gives 
$$\Sigma \mid \Gamma[S/\zeta] @ (A[S/\zeta] \cup \bigcup_{i \in 1.. n} A_i[S/\zeta] \cup B_i[S/\zeta]) \vdash \tau[S/\zeta].\{ [\tau_i[S/\zeta] :^{m_i} \tau_i'[S/\zeta]^{~i \in 1..n} \} :: \mathit{KTree}(d[S/\zeta])$$
and furthermore,
\begin{center}
\begin{tabular}{l}
$(\langle \mathit{id}, \sem{\Gamma \vdash S} \rangle^* \sem{\Sigma,\zeta \mid \Gamma @ A \cup \bigcup_{i \in 1..n} A_i \cup B_i \vdash \tau.\{ [\tau_i] :^{m_i} \tau_i' \} :: \mathit{KTree(d)} })_\sigma (\gamma,a)$ \\
$= \sem{\Sigma,\zeta \mid \Gamma @ A \cup \bigcup_{i \in 1..n} A_i \cup B_i \vdash \tau.\{ [\tau_i] :^{m_i} \tau_i' \} :: \mathit{KTree(d)} }_{\langle \mathit{id}, \sem{\Sigma \vdash S} \rangle(\sigma)}(\gamma,a)$ \\
$= \{ g \in \mathit{Inst} \mid g(\epsilon) \in \sem{\Sigma,\zeta \mid \Gamma @ A \vdash \tau :: \mathit{KString(d,\phi)}}~\wedge$ \\
$~~~~~\forall s \in (\sem{\Sigma,\zeta \mid \Gamma @ B_i \vdash \tau_i :: \iota_i};\sem{\Sigma,\zeta \vdash \iota_i <:: \star})(\gamma,a).$\\
$~~~~~~~~~~~g \! \mid_{[s]} \in (\sem{\Sigma,\zeta \mid \Gamma @ A_i \vdash \tau_i' :: \kappa};\sem{\Sigma,\zeta \vdash \kappa_i <:: \ast})(\gamma,a)^{~i \in 1..n} \}$
\end{tabular}
\end{center}
that is,
$$\Sigma \mid \Gamma[S/\zeta] @ (A \cup \bigcup_{i \in 1..n} A_i \cup B_i)[S/\zeta] \vdash (\tau.\{ [\tau_i] :^{m_i} \tau'_i \})[S/\zeta] :: (\mathit{KTree}(d))[S/\zeta]$$

\item[Case \sc{Path}]:~\\
Applying lemma \ref{SWF-Subst} to $\Sigma,\zeta \vdash S_i^{~i \in 1..n}$ gives $\Sigma \vdash S_i[S/\zeta]^{~i \in 1..n}$. The IH gives $\Sigma \mid \Gamma[S/\zeta] @ A[S/\zeta] \vdash \tau[S/\zeta] :: \kappa[S/\zeta]$. Applying lemma \ref{SK-Subst2} gives $\Sigma \vdash \kappa[S/\zeta] <:: \star$.

\item[Case \begin{sc}StrAbs\end{sc}]:~\\

TODO.

\item[Case \sc{TyAbs}]:~\\

TODO.

\item[Case \sc{TyApp}]:~\\

TODO.

\item[Case \sc{StrApp}]:~\\

TODO.

\end{description}

\end{proof}

\subsection*{Semantic Substitution}

\subsubsection*{Index substitution}

\begin{lemma}
If $\Sigma,\zeta \vdash S$ and $\Sigma \vdash S'$ then $\Sigma \vdash S[S'/\zeta]$ and $$\sem{\Sigma \vdash S[S'/\zeta]} = \langle \mathit{id}, \sem{\Sigma \vdash S'} \rangle^* \sem{\Sigma,\zeta \vdash S}$$
\end{lemma}

\begin{proof}

\end{proof}

\subsubsection*{Kind well-formedness semantic substitution}

\begin{lemma}
Let $\Sigma,\zeta \vdash \kappa$ and $\Sigma \vdash S$. Then $$\sem{\Sigma \vdash \kappa[S/\zeta]} = \langle \mathit{id}, \sem{\Sigma \vdash S} \rangle^* \sem{\Sigma,\zeta \vdash \kappa}$$
\end{lemma}

\begin{proof}
By induction on the proof of $\Sigma \vdash \kappa$.
\begin{description}
\item[Case \begin{sc}KWF-Str\end{sc}]:~\\
Hello
\end{description}
\end{proof}

\subsubsection*{Kinding semantic substitution}

\begin{lemma}
Let $g \doteq \sem{\Sigma \vdash S}$. Then $\sem{\Sigma \mid \Gamma[S/\zeta] @ A[S/\zeta] \vdash \tau[S/\zeta] :: \kappa[S/\zeta]} = \langle \mathit{id}, g \rangle^* \sem{\Sigma,\zeta \mid \Gamma @ A \vdash \tau :: \kappa}$. 
\end{lemma}

\begin{proof}
By induction on the proof of $\Sigma,\zeta \mid \Gamma @ A \vdash \tau :: \kappa$.

\begin{description}
\item[\sc{Record}]:~\\
The IH gives $$\sem{\Sigma \mid \Gamma[S/\zeta] @ A[S/\zeta] \vdash \tau[S/\zeta] :: \kappa[S/\zeta]} = \langle \mathit{id}, g \rangle^* \sem{\Sigma,\zeta \mid \Gamma @ A \vdash \tau :: \kappa}$$ and $$\sem{\Sigma \mid \Gamma[S/\zeta] @ B_i[S/\zeta] \vdash \tau_i[S/\zeta] :: \kappa_i[S/\zeta]} = \langle \mathit{id}, g \rangle^* \sem{\Sigma,\zeta \mid \Gamma @ B_i \vdash \tau_i :: \iota_i}^{~i \in 1..n}$$ and $$\sem{\Sigma \mid \Gamma[S/\zeta] @ A_i[S/\zeta] \vdash \tau_i'[S/\zeta] :: \kappa_i[S/\zeta]} = \langle \mathit{id}, g \rangle^* \sem{\Sigma,\zeta \mid \Gamma @ A_i \vdash \tau_i' :: \kappa_i}^{~i \in 1..n}$$
\end{description}

\end{proof}

\subsection*{Soundness}

\subsubsection*{String abstraction applications}

We must prove that $\Sigma \mid \Gamma @ A \vdash (\lambda \zeta.\tau)~[S] :: \kappa[S/\zeta]$ implies 
$$\Sigma \mid \Gamma @ A \vdash \tau[S/\zeta] :: \kappa[S/\zeta]$$
We must further prove that $$\sem{\Sigma \mid \Gamma @ A \vdash (\lambda \zeta.\tau)~[S] :: \kappa[S/\zeta]}$$
is equal to $$\sem{\Sigma \mid \Gamma @ A \vdash \tau[S/\zeta] :: \kappa[S/\zeta]}$$
We do so by induction on the structure of $\tau$.

\section*{Additional ideas}

Rather than just using a list of variables for the string index context, we should classify each
string index variable by the set of possible characters that may occur in the string. 
 
\end{document}
  
