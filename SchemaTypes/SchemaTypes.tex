\documentclass{article}
 
\usepackage{amscd}
\usepackage{graphicx}
\usepackage{amsmath}
\usepackage{amssymb}
\usepackage{amsthm}
\usepackage{stmaryrd}
\usepackage{mathpartir}
\usepackage{multicol}
\usepackage{enumitem}
\usepackage[section]{placeins} % float barriers
\usepackage{natbib}
\usepackage{xcolor} 
\usepackage{bussproofs} 
\usepackage{diagrams}
\usepackage{tikz}
\usepackage{tabu}
 
\usetikzlibrary{cd}
 
\newtheorem{lemma}{Lemma}

%example: \limit{j \in J}{F_j}
\newcommand{\limit}[2]{\underset{\overset{\longleftarrow}{#1}}{\text{lim}}~#2}
\newcommand{\lims}[1]{\underset{\longleftarrow}{\text{lim}}~#1}
\newcommand{\mbf}{\mathbf}
\newcommand{\sem}[1]{\llbracket #1 \rrbracket}
\newcommand{\defeq}{\overset{\mathit{def}}{=}}


\newcommand{\vrt}[2]{
\pile{
#1 \\
\downarrow \\
#2
}
}

\newcommand{\ddisp}[3]{
\left(
\scriptsize
\begin{tikzcd}
#1 \ar[d, "\footnotesize{#2}"] \\
#3
\end{tikzcd}
\normalsize
\right)
}

\newcommand{\disp}[3]{
\left(
\tiny
\begin{array}{c}
#1 \\
\downarrow\\
#3
\end{array}
\begin{array}{l}
~ \\
#2 \\
~
\end{array}
\normalsize
\right)
}

\newcommand{\dispp}[3]{
\tiny
\begin{tikzcd}
#1 \ar[d, "#2"] \\
#3
\end{tikzcd}
\normalsize
}

% give tables some extra space between rows and columns
\renewcommand{\arraystretch}{1.4}

\title{Schema Type Calculus}

\begin{document}

\maketitle

\section*{Syntax}

\begin{tabular}{llll}
$\mathit{Chars}$ & $\doteq$ & the set of all characters \\
$\mathit{TypeVars}$ & $\doteq$ & the set of all type variables \\
$\mathit{StringVars}$ & $\doteq$ & the set of all string variables \\
$\alpha$ & $\in$ & $\mathit{TypeVars}$ & ~ \\
$\zeta$ & $\in$ & $\mathit{StringVars}$ & ~ \\
$c$ & $\in$ & $\mathit{Chars}$ & (character) \\
$\phi$ & $\in$ & $\mathcal P(\mathit{Chars})$ & (a subset of the set of characters) \\
$s,t$ & $\in$ & $\mathit{Strings}$ & (where $\mathit{Strings} = \mathit{Chars}^{\star}$) \\~\\
$S$ (string index) & $::=$ & $s$ & (literal string) \\
                   & $\mid$ & $\zeta$ & (string variable) \\~\\
$d$ (discriminator) & $::=$  & $\mathit{Prefix}~s$ & (values begin with $s$) \\
    & $\mid$ & $\mathit{Literal}~s$ & (values are equal to $s$) \\
 & & \\
$m$ (multiplicity) & $::=$ & $1 \mid \mathit{set} \mid \mathit{opt}$\\~\\
$\tau$ (type) & $::=$ & $\mathit{tuple} \{ s_i : \tau_i : c_i^{~i \in 1..n} \}$ & (pieced string) \\
       & $\mid$ & $S$ & (string index) \\
       & $\mid$ & $(S_1,\ldots,S_n).[\kappa]$ & (path type) \\
       & $\mid$ & $\biguplus \tau_i^{~i \in 1..n}$ & (union of disjoint types) \\ 
       & $\mid$ & $\tau.\{[\tau_i] :^{m_i} \tau_i ^{~i \in 1..n}\}$ & (record type - $\mathit{root}.\{ fields \}$) \\ 
       & $\mid$ & $\lambda (\alpha :: \kappa). \tau$ & (type abstraction) \\
       & $\mid$ & $\lambda \zeta. \tau$ & (string abstraction) \\
       & $\mid$ & $\tau~[S]$ & (index application) \\
       & $\mid$ & $\tau~\tau$ & (type application) \\
       & $\mid$ & $\alpha$ & (type variable) \\~\\
$\kappa, \iota$ (kind) & $::=$ & $\mathit{KString}(d,\phi)$ & (Kind of string types, chars drawn from $\phi$) \\
                         & $\mid$ & $\mathit{KTree}(d)$ & (Kind of tree types w/ root discriminator $d$) \\
                         & $\mid$ & $\kappa \overset{A}{\Rightarrow} \kappa$ & (Kind of type abstractions) \\
                         & $\mid$ & $\forall \zeta. \kappa$ & (Kind of string abstractions) \\~\\
$\Sigma$ (index context) & $::=$ & $\Sigma,\zeta$ & (context extension) \\
                         & $\mid$ & $\emptyset$ & (empty context) \\~\\
$\Gamma$ (kind context) & $::=$ & $\Gamma,x :: \kappa$ & (context extension) \\
                           & $\mid$ & $\emptyset$ & (empty context) \\~\\

\end{tabular}\\~\\~\\
\begin{tabular}{llll}

\end{tabular}

\section*{Index well-formedness}

\begin{mathpar}
\inferrule[IWF-Var]
  {\zeta \in \Sigma}
  {\Sigma \vdash \zeta}
\and
\inferrule[IWF-Lit]
  {~}
  {\Sigma \vdash s}
\end{mathpar}
The interpretation $\sem{\Sigma}$ of an index context $\Sigma = \zeta_1, \ldots, \zeta_n$ is the set of all functions from $\{ \zeta_1, \ldots, \zeta_n \}$ to the set of strings. The interpretation $\sem{\Sigma \vdash S}$ of a well-formed index judgment is a function from $\sem{\Sigma}$ to the set of all strings, i.e. $\sem{\Gamma \vdash S} : \sem{\Gamma} \to \mathit{Strings}$, defined as follows:
\begin{mathpar}
\sem{\Sigma \vdash \zeta} \sigma \doteq \sigma(\zeta)
\and
\sem{\Sigma \vdash s} \sigma \doteq s
\end{mathpar}

\section*{Discriminator well-formedness}

\begin{mathpar}
\inferrule[DWF-Prefix]
  {\Sigma \vdash S}
  {\Sigma \vdash \mathit{Prefix}~S}
\and
\inferrule[DWF-Lit]
  {\Sigma \vdash S}
  {\Sigma \vdash \mathit{Literal}~S}
\end{mathpar}

The interpretation $\sem{\Sigma \vdash d}$ of a discriminator well-formedness judgment is a function from $\sem{\Sigma}$ into $\mathcal P(\mathit{Strings})$. 
\begin{mathpar}
\sem{\Sigma \vdash \mathit{Prefix}~S}\sigma \doteq \{ t \mid \sem{\Sigma \vdash S} \sigma \text{ is a prefix of } t \}
\and
\sem{\Sigma \vdash \mathit{Literal}~S} \sigma \doteq \{ \sem{\Sigma \vdash S}\sigma \}
\end{mathpar}

\section*{Discriminator subsumption}

\begin{mathpar}
\inferrule[SD-PrefPref]
  {s' \sqsubseteq s}
  {\Sigma \vdash \mathit{Prefix}~s \leq \mathit{Prefix}~s'}
\and
\inferrule[SD-LitLit]
  {\Sigma \vdash S}
  {\Sigma \vdash \mathit{Literal}~S \leq \mathit{Literal}~S}
\and
\inferrule[SD-LitPref]
  {s' \sqsubseteq s}
  {\Sigma \vdash \mathit{Literal}~s \leq \mathit{Prefix}~s'}
\and
\end{mathpar}

Above, $s' \sqsubseteq s$ means ``$s'$ is a prefix of $s$''.
Ordering $\mathcal P(\mathit{Strings})$ by inclusion and the function space $\sem{\Sigma} \to \mathcal P(\mathit{Strings})$ pointwise, 
The interpretation $\sem{\Sigma \vdash d \leq d'}$ of a subsumption judgment is a proof that that $\sem{\Sigma \vdash d} \leq \sem{\Sigma \vdash d'}$.

\section*{Kind well-formedness}

\begin{mathpar}
\inferrule[KWF-Str]
  {\Sigma \vdash d}
  {\Sigma \vdash \mathit{KString(d,\phi)}}
\and
\inferrule[KWF-Tree]
  {\Sigma \vdash d}
  {\Sigma \vdash \mathit{KTree(d)}}
\and
\inferrule[KWF-TyFun]
  {\Sigma \vdash \kappa \\ \Sigma \vdash \kappa' \\ \Sigma \vdash A}
  {\Sigma \vdash \kappa \overset{A}{\Rightarrow} \kappa'}
\and
\inferrule[KWF-StrFun]
  {\Sigma,\zeta \vdash \kappa}
  {\Sigma \vdash \forall \zeta. \kappa}
\end{mathpar}

The interpretation $\sem{\Sigma \vdash \kappa}$ of a kind well-formedness judgment is a $\sem{\Sigma}$-indexed family of posets, defined, for $\sigma \in \sem{\Sigma}$ as follows.

$$\sem{\Sigma \vdash \mathit{KString}(d,\phi)} \sigma \doteq (\mathcal P(X), \subseteq),~~\text{where } X \doteq \{ x \in \sem{\Sigma \vdash d} \sigma \mid \mathit{chars}(x) \subseteq \phi \}$$

\section*{Subkinding}

The interpretation $\sem{\Sigma \vdash \kappa <:: \kappa'}$ is a mono in the fibre $\mathit{Fam}(\mbf{Posets})_{\sem{\Sigma}}$ from $\sem{\Sigma \vdash \kappa}$ to $\sem{\Sigma \vdash \kappa'}$.

\begin{mathpar}
\inferrule[SK-StrTree]
  {\Sigma \vdash d \leq d'}
  {\Sigma \vdash \mathit{KString}(d,\phi) <:: \mathit{KTree}(d')}
\and
\inferrule[SK-StrStr]
  {\Sigma \vdash d \leq d' \\ \Sigma \vdash \phi \subseteq \phi'}
  {\Sigma \vdash \mathit{KString}(d,\phi) <:: \mathit{KString}(d',\phi')}
\and
\inferrule[SK-TreeTree]
  {\Sigma \vdash d \leq d'}
  {\Sigma \vdash \mathit{KTree}(d) <:: \mathit{KTree}(d')}
\end{mathpar}

\section*{Database instances}

Given lists $\rho$ and $\theta$ of strings, we write $\rho \cdot \theta$ for the list obtained by concatenating $\theta$ to the right of $\rho$.
A \emph{database instance} is a partial function $f : \mathit{Strings}^{\star} \rightharpoonup \mathit{Strings}$ from lists of strings to strings. We write $\mathit{Inst}$ for the set of database instances. For a database instance $f$ and a list of strings $\rho$, we write $f \! \mid_\rho$ for the database instance defined such that for all lists of strings $\theta$ we have $$f \! \mid_\rho \! (\theta) \doteq f(\rho \cdot \theta)$$
Furthermore, we define $\mathit{keys}(f) \doteq \{ s \mid f(s) \text{ is defined } \}$.

\section*{Coeffect Well-formedness}

The judgment $\Sigma \vdash A$ means that the coeffect $A$ is well-formed under index context $\Sigma$,
in the sense that for each $(S_1,\ldots,S_n).[\kappa] \in A$ we have $\Sigma \vdash S_i^{~i \in 1..n}$ and $\Sigma \vdash \kappa$.

The interpretation $\sem{\Sigma \vdash A}$ is a function of type $\sem{\Sigma} \to \mathcal{P}(\mathit{Inst})$, defined as $$\sem{\Sigma \vdash A}\sigma \doteq \bigcap_{(S_1,\ldots,S_n).[\kappa] \in A} \{ f \in \mathit{Inst} \mid \mathit{keys}(f \! \mid_{\sigma S_1, \ldots, \sigma S_n}) \subseteq \sem{\Sigma \vdash \kappa} \sigma \}$$

\section*{Fibred graded comonads}

For each $\Sigma$, we define a graded comonad $D_{\Sigma}$ on the category $\mathit{Fam}(\mbf{Posets})_{\sem{\Sigma}}$ of
$\sem{\Sigma}$-indexed families of posets. The scalar coeffect structure is 
$$( \{ A \mid \Sigma \vdash A \}, \cup, \cup, \emptyset, \emptyset, \subseteq)$$
For objects $(X_\sigma)_{\sigma \in \sem{\Sigma}}$ of the fibre $\mathit{Fam}(\mbf{Posets})_{\sem{\Sigma}}$, we define $$D_{\Sigma} A(~(X_\sigma)_{\sigma \in \sem{\Sigma}}~) \doteq (X_\sigma \times \sem{\Sigma \vdash A}\sigma)_{\sigma \in \sem{\Sigma}}$$.
For arrows $(f_\sigma : X_\sigma \to Y_\sigma)_{\sigma \in \sem{\Sigma}}$ of the fibre $\mathit{Fam}(\mbf{Posets})_{\sem{\Sigma}}$ we define $$D_{\Sigma} A(~(f_\sigma)_{\sigma \in \sem{\Sigma}}~) \doteq (~(x_\sigma, a_\sigma) \mapsto (f_\sigma(x_\sigma),a_\sigma)~)_{\sigma \in \sem{\Sigma}}$$
Its counit $\epsilon$ is defined as $$\epsilon_X(x) \doteq (x,\mathit{Inst})$$
Its comultiplication $\delta$, for coeffect scalars $A$ and $B$, is defined as $$\delta_{A,B}((x,a),b) \doteq (x, a \cap b)$$
For all string contexts $\Sigma$ and coeffect scalars $\Sigma \vdash A$ and $\Sigma \vdash B$ with $A \subseteq B$ we have a natural transformation $\mathit{sub}_{A,B} : D_{\Sigma} B \to D_{\Sigma } A$. Since for posets $X$, $\sem{\Sigma \vdash B} \subseteq \sem{\Sigma \vdash A}$, we have that $\mathit{sub}_{A,B}$ is the identity function.\\~\\
For all coeffect scalars $\Sigma \vdash A$ and kinding contexts $\Sigma \vdash \Gamma,x :: \kappa$, we have a natural isomorphism $$m_{\Sigma,A,\Gamma,\kappa} : D_{\Sigma} \emptyset(\sem{\Sigma \vdash \Gamma}) \times D_{\Sigma} A( \sem{\Sigma \vdash \kappa} ) \overset{\sim}{\to} D_{\Sigma} A \sem{\Sigma \vdash \Gamma,x :: \kappa}$$ defined to discard the copy of $\sem{\Sigma \vdash \emptyset}$ produced by applying $D_{\Sigma} \emptyset$, and to reassociate.
\section*{Kinding}

In what follows, we define $\ast \doteq \mathit{KTree}(\mathit{Prefix}~\epsilon)$, where $\epsilon$ is the empty string, so that all proper types have kind $\ast$. We define $\star \doteq \mathit{KString}(\mathit{Prefix}~\epsilon, \mathit{Chars})$, so that all string types have kind $\star$.

\begin{mathpar}
\inferrule[Tuple]
  {\Sigma \mid \Gamma@A_i \vdash \tau_i :: \kappa_i^{~i \in 1..n} \\ 
   \Sigma \vdash \kappa_i <:: \mathit{KString}(d_i,\phi_i)^{~i \in 1..n} \\
   c_i \not \in \phi_i^{~i \in 1..n} }
  {\Sigma \mid \Gamma@(\bigcup_{i \in 1..n} A_i) \vdash \mathit{tuple} \{ s_i : \tau_i : c_i^{~i \in 1..n}\} :: \mathit{KString}(d_1, (\bigcup_{i \in 1..n} \phi_i) \cup \{ c_1, \ldots, c_n \})}
\and
\inferrule[DisjUnion]
  {\Sigma \mid \Gamma@A_i \vdash \tau_i :: \mathit{KTree}(d_i)^{~i \in 1..n} \\ \mathit{disjoint} \{ d_i^{~i \in 1..n} \} }
  {\Sigma \mid \Gamma@(\bigcup_{i \in 1..n} A_i) \vdash \biguplus_{i \in 1..n} \tau_i :: \mathit{KTree}(\mathit{merge}~\{ d_i^{~i \in 1..n} \})}
\and
\inferrule[Record]
  {\Sigma \mid \Gamma@A \vdash \tau :: \mathit{KString}(d,\phi) \\ 
    \Sigma \mid \Gamma@B_i \vdash \tau_i :: \iota_i^{~i \in 1..n} \\ \Sigma \vdash \iota_i <:: \star^{~i \in 1..n} \\ 
    \Sigma \mid \Gamma@A_i \vdash \tau_i' :: \kappa_i^{~i \in 1..n} \\ \Sigma \vdash \kappa_i <:: \ast^{~i \in 1..n}}
  {\Sigma \mid \Gamma@(A \cup \bigcup_{i \in 1..n} A_i \cup B_i) \vdash \tau.\{[\tau_i] :^{m_i} \tau_i'^{~i \in 1..n}\} :: \mathit{KTree}(d)}
\and
\inferrule[Path]
  {\Sigma \vdash S_i^{~i \in 1..n} \\ 
   \Sigma \mid \Gamma@A \vdash \tau :: \kappa \\ \Sigma \vdash \kappa <:: \star}
  {\Sigma \mid \Gamma@(\{ (S_1,\ldots,S_n).[\tau] \} \cup A) \vdash (S_1,\ldots,S_n).[\tau] :: \kappa}
\and
\inferrule[StrAbs]
  {\Sigma, \zeta \mid \Gamma @ \emptyset \vdash \tau :: \kappa}
  {\Sigma \mid \Gamma @ \emptyset \vdash \lambda \zeta. \tau :: \forall \zeta. \kappa}
\and
\inferrule[TyAbs]
  {\Sigma \mid \Gamma,\alpha :: \kappa @ A \vdash \tau :: \kappa'}
  {\Sigma \mid \Gamma @ \emptyset \vdash \lambda (\alpha :: \kappa).\tau :: \kappa \overset{A}{\Rightarrow} \kappa'}
\and
\inferrule[TyApp]
  {\Sigma \mid \Gamma @ A \vdash \tau :: \kappa \overset{C}{\Rightarrow} \iota \\ 
   \Sigma \mid \Gamma @ B \vdash \tau' :: \kappa}
  {\Sigma \mid \Gamma @ (A \cup B \cup C) \vdash \tau~\tau' :: \iota}
\and
\inferrule[StrApp]
  {\Sigma \mid \Gamma @ A \vdash \tau :: \forall \zeta. \kappa \\ \Sigma \vdash S}
  {\Sigma \mid \Gamma @ A \vdash \tau~[S] :: \kappa[S/\zeta]}

\end{mathpar}
We write $\Sigma \vdash \Gamma$ to mean that $\Sigma \vdash \kappa$ for all $(x : \kappa) \in \Gamma$.
When $\Gamma = x_1 : \kappa_1, \ldots, x_n : \kappa_n$, we define the interpretation $\sem{\Sigma \vdash \Gamma} \doteq \sem{\Sigma \vdash \kappa_1} \times \cdots \times \sem{\Sigma \vdash \kappa_n}$, where products in $\mathit{Fam}(\mbf{Posets})$ are taken pointwise.
We will prove that $\Sigma \mid \Gamma @ A \vdash \tau :: \kappa$ implies that $\Sigma \vdash \Gamma$, $\Sigma \vdash \kappa$, and $\Sigma \vdash A$. The interpretation $$\sem{\Sigma \mid \Gamma @ A \vdash \tau :: \kappa}$$ of a kinding judgment is a morphism of the fibre category $\mathit{Fam}(\mbf{Posets})_{\sem{\Sigma}}$ from $D_{\Sigma} A(\sem{\Sigma \vdash \Gamma})$ to $\sem{\Sigma \vdash \kappa}$.

\begin{mathpar}
\inferrule[Tuple]
  {\sem{\Sigma \mid \Gamma@A_i \vdash \tau_i :: \kappa_i} = f_i : D_{\Sigma} A_i(\sem{\Sigma \vdash \Gamma}) \to \sem{\Sigma \vdash \kappa_i}^{~i \in 1..n} \\ 
   \sem{\Sigma \vdash \kappa_i <:: \mathit{KString}(d_i,\phi_i)} = g_i : \sem{\Sigma \vdash \kappa_i} \to \sem{\Sigma \vdash \mathit{KString}(d_i,\phi_i)}^{~i \in 1..n} \\
   c_i \not \in \phi_i^{~i \in 1..n} }
  {\Sigma \mid \Gamma@(\bigcup_{i \in 1..n} A_i) \vdash \mathit{tuple} \{ s_i : \tau_i : c_i^{~i \in 1..n}\} :: \mathit{KString}(d_1, (\bigcup_{i \in 1..n} \phi_i) \cup \{ c_1, \ldots, c_n \})}
\and
\inferrule[DisjUnion]
  {\Sigma \mid \Gamma@A_i \vdash \tau_i :: \mathit{KTree}(d_i)^{~i \in 1..n} \\ \mathit{disjoint} \{ d_i^{~i \in 1..n} \} }
  {\Sigma \mid \Gamma@(\bigcup_{i \in 1..n} A_i) \vdash \biguplus_{i \in 1..n} \tau_i :: \mathit{KTree}(\mathit{merge}~\{ d_i^{~i \in 1..n} \})}
\and
\inferrule[Record]
  {\Sigma \mid \Gamma@A \vdash \tau :: \mathit{KString}(d,\phi) \\ 
    \Sigma \mid \Gamma@B_i \vdash \tau_i :: \iota_i^{~i \in 1..n} \\ \Sigma \vdash \iota_i <:: \star^{~i \in 1..n} \\ 
    \Sigma \mid \Gamma@A_i \vdash \tau_i' :: \kappa_i^{~i \in 1..n} \\ \Sigma \vdash \kappa_i <:: \ast^{~i \in 1..n}}
  {\Sigma \mid \Gamma@(A \cup \bigcup_{i \in 1..n} A_i \cup B_i) \vdash \tau.\{[\tau_i] :^{m_i} \tau_i'^{~i \in 1..n}\} :: \mathit{KTree}(d)}
\and
\inferrule[Path]
  {\Sigma \vdash S_i^{~i \in 1..n} \\ 
   \Sigma \mid \Gamma@A \vdash \tau :: \kappa \\ \Sigma \vdash \kappa <:: \star}
  {\Sigma \mid \Gamma@(\{ (S_1,\ldots,S_n).[\tau] \} \cup A) \vdash (S_1,\ldots,S_n).[\tau] :: \kappa}
\and
\inferrule[StrAbs]
  {\sem{\Sigma, \zeta \mid \Gamma @ \emptyset \vdash \tau :: \kappa} = f : D_{\Sigma,\zeta}\emptyset(\sem{\Sigma,\zeta \vdash \Gamma}) \to \sem{\Sigma,\zeta \vdash \kappa}}
  {\sem{\Sigma \mid \Gamma @ \emptyset \vdash \lambda \zeta. \tau :: \forall \zeta. \kappa} \doteq f^\flat : D_{\Sigma} \emptyset(\sem{\Sigma \vdash \Gamma}) \to \sem{\Sigma \vdash \forall \zeta. \kappa)}}
\and
\inferrule[TyAbs]
  {\sem{\Sigma \mid \Gamma,\alpha :: \kappa @ A \vdash \tau :: \kappa'} = f : D_{\Sigma} A(\sem{\Sigma \vdash \Gamma,\alpha :: \kappa}) \to \sem{\Sigma \vdash \kappa'}}
  {\sem{\Sigma \mid \Gamma @ \emptyset \vdash \lambda (\alpha :: \kappa).\tau :: \kappa \overset{A}{\Rightarrow} \kappa'} \doteq \Lambda(m_{\Sigma,A,\Gamma,\kappa};f) : D_{\Sigma} \emptyset(\sem{\Sigma \vdash \Gamma}) \to \sem{\Sigma \vdash \kappa \overset{A}{\Rightarrow} \kappa'}}
\and
\inferrule[TyApp]
  {\Sigma \mid \Gamma @ A \vdash \tau :: \kappa \overset{C}{\Rightarrow} \iota \\ 
   \Sigma \mid \Gamma @ B \vdash \tau' :: \kappa}
  {\Sigma \mid \Gamma @ (A \cup B \cup C) \vdash \tau~\tau' :: \iota}
\and
\inferrule[StrApp]
  {\sem{\Sigma \mid \Gamma @ A \vdash \tau :: \forall \zeta. \kappa} = f : D_{\Sigma}A(\sem{\Sigma \vdash \Gamma}) \to \sem{\Sigma \vdash \forall \zeta. \kappa} \\ \sem{\Sigma \vdash S} = g : \sem{\Sigma} \to \mathit{Strings}}
  {\sem{\Sigma \mid \Gamma @ A \vdash \tau~[S] :: \kappa[S/\zeta]} \doteq \langle id_{\sem{\Sigma}}, g \rangle^*(f^\sharp) : D_{\Sigma} A(\sem{\Sigma \vdash \Gamma}) \to \langle id_{\Sigma}, g \rangle^* \sem{\Sigma,\zeta \vdash \kappa}}

\end{mathpar}

\subsection*{Explanation}

\subsubsection*{\sc{StrAbs}}
For the string abstraction rule (second-to-last) is essentially mapping across an adjunction, but we need the following lemma to see this.
\begin{lemma}
$\pi^*_{\Sigma,\zeta}(D_{\Sigma} \emptyset(\sem{\Sigma \vdash \Gamma})) = D_{\Sigma,\zeta} \emptyset(\sem{\Sigma,\zeta \vdash \Gamma})$
\end{lemma}

\subsubsection*{\sc{StrApp}}

The domain of the conclusion should be $D_{\Sigma}A(\sem{\Sigma \vdash \Gamma})$. The domain of $\langle \mathit{id}_{\Sigma}, g \rangle^*(f^\sharp)$ is $\langle \mathit{id}_{\sem{\Sigma}}, g \rangle^* \pi_{\Sigma,\zeta}^*(D_{\Sigma} A \sem{\Sigma \vdash \Gamma})$, which is equal to $D_{\Sigma}A(\sem{\Sigma \vdash \Gamma})$ due to the following calculation.\\~\\
\begin{tabu}{l}
$\langle \mathit{id}_{\sem{\Sigma}}, g \rangle^* \pi_{\Sigma,\zeta}^* D_{\Sigma} A (\sem{\Sigma \vdash \Gamma})$ \\
$= (\langle \mathit{id}_{\sem{\Sigma}}, g \rangle; \pi_{\Sigma,\zeta})^* D_{\Sigma} A (\sem{\Sigma \vdash \Gamma})$ \\
$= id_{\sem{\Sigma}}^* D_{\Sigma} A (\sem{\Sigma \vdash \Gamma})$ \\
$= D_{\Sigma} A (\sem{\Sigma \vdash \Gamma})$
\end{tabu}

\subsection*{Index substitution lemmas}

\begin{lemma}
If $\Sigma,\zeta \vdash S$ and $\Sigma \vdash S'$ then $\Sigma \vdash S[S'/\zeta]$.
\label{SWF-Subst}
\end{lemma}

\begin{proof}
By cases on $S$.

\begin{description}
\item[Case $S = s$]:\\
We have $S[S'/\zeta] = s[S'/\zeta] = s$. By \begin{sc}IWF-Lit\end{sc}, $s$ is a well-formed index term in 
any index environment,
so in particular we have $\Sigma \vdash s$ i.e. $\Sigma \vdash s[S'/\zeta]$.

\item[Case ]$S = \zeta'$:\\
We have $\Sigma,\zeta \vdash \zeta'$. So either $\zeta' \in \Sigma$ or $\zeta' = \zeta$.

\begin{description}
\item[Case $\zeta' \in \Sigma$]:\\
We have $S[S'/\zeta] = \zeta'$, and since $\zeta' \in \Sigma$, \begin{sc}IWF-Var\end{sc} gives $\Sigma \vdash \zeta'$, i.e.
$\Sigma \vdash S[S'/\zeta]$.
\item[Case $\zeta' = \zeta$]:\\
We have $S[S'/\zeta] = S'$. By assumption, $\Sigma \vdash S'$, i.e. $\Sigma \vdash S[S'/\zeta]$.
%By \begin{sc}IWF-Lit\end{sc} we have $\Sigma \vdash s$, i.e. $\Sigma \vdash S[S'/\zeta]$.

\end{description}

\end{description}

\end{proof}

\begin{lemma}
If $\Sigma,\zeta \vdash d$ and $\Sigma \vdash S$ then $\Sigma \vdash d[S/\zeta]$
\label{DWF-Subst}
\end{lemma}

\begin{proof}
By cases on $d$

\item[Case $d = \mathit{Prefix}~S'$]:\\
Since $\Sigma,\zeta \vdash \mathit{Prefix}~S'$ we get $\Sigma,\zeta \vdash S'$ from the premise of 
\begin{sc}DWF-Prefix\end{sc}. By Lemma \ref{SWF-Subst} we get $\Sigma \vdash S'[S/\zeta]$. Applying
\begin{sc}DWF-Prefix\end{sc} using the premise $\Sigma \vdash S'[S/\zeta]$ then gives $\Sigma \vdash \mathit{Prefix}~S'[S/\zeta]$, i.e. $\Sigma \vdash d[S/\zeta]$.

\item[Case $d = \mathit{Literal}~S'$]:\\
Similar to the previous case.

\end{proof}

\begin{lemma}
If $\Sigma,\zeta \vdash \kappa$ and $\Sigma \vdash S$ then $\Sigma \vdash \kappa[S/\zeta]$.
Furthermore, if $\Sigma,\zeta \vdash A$ and $\Sigma \vdash S$ then $\Sigma \vdash A[S/\zeta]$.
\end{lemma}

\begin{proof}
By simultaneous induction on the proofs of $\Sigma,\zeta \vdash \kappa$ and $\Sigma,\zeta \vdash A$. 

\begin{description}
\item[Case $\Sigma,\zeta \vdash A$]:~\\
If $\Sigma,\zeta \vdash A$ then for each $(S_1,\ldots,S_n).[\kappa] \in A$ we 
have $\Sigma,\zeta \vdash S_i^{~i \in 1..n}$ and $\Sigma,\zeta \vdash \kappa$. By Lemma \ref{SWF-Subst} we have
$\Sigma \vdash S_i[S/\zeta]$. By the inductive hypothesis, we have $\Sigma \vdash \kappa[S/\zeta]$. From this,
we deduce that $\Sigma \vdash (S_1[S/\zeta],\ldots,S_n[S/\zeta]).[\kappa[S/\zeta]]$; hence $\Sigma \vdash A[S/\zeta]$. (TODO: should we name the ``rule'' for concluding coeffect well-formedness judgments?)
\item[Case \sc{KWF-Str}]:~\\
We apply Lemma \ref{DWF-Subst} to the premise to get $\Sigma \vdash d[S/\zeta]$. Then applying 
\begin{sc}KWF-Str\end{sc} using $\Sigma \vdash d[S/\zeta]$ as the premise gives 
$\Sigma \vdash \mathit{KString}(d[S/\zeta],\phi)$, i.e. $\Sigma \vdash \kappa[S/\zeta]$.

\item[Case \sc{KWF-Tree}]:~\\
Similar to the above case.

\item[Case \sc{KWF-TyFun}]:~\\
Our inductive hypothesis gives $\Sigma \vdash \kappa[S/\zeta]$, $\Sigma \vdash \kappa'[S/\zeta]$, and
$\Sigma \vdash A[S/\zeta]$. Applying \begin{sc}KWF-TyFun\end{sc} gives $\Sigma \vdash \kappa[S/\zeta] \overset{A[S/\zeta]}{\Rightarrow} \kappa'[S/\zeta]$ i.e. $$\Sigma \vdash (\kappa \overset{A}{\Rightarrow} \kappa')[S/\zeta]$$

\item[Case \sc{KWF-StrFun}]:~\\
The premise for proving $\Sigma,\zeta \vdash \forall \zeta'.\kappa$ is $\Sigma,\zeta,\zeta' \vdash \kappa$. Exchanging $\zeta$ and $\zeta'$ gives $\Sigma,\zeta',\zeta \vdash \kappa$. (The proof should be of the same height.) Applying the IH gives $\Sigma,\zeta' \vdash \kappa[S/\zeta]$. Applying \begin{sc}KWF-StrFun\end{sc} gives $\Sigma \vdash \forall \zeta'. \kappa[S/\zeta]$ i.e.
$$\Sigma \vdash (\forall \zeta'. \kappa)[S/\zeta]$$.
\end{description}
\end{proof}

\begin{lemma}
If $\Sigma,\zeta \vdash \kappa <:: \kappa'$ and $\Sigma \vdash S$ then $\Sigma \vdash \kappa[S/\zeta] <:: \kappa'[S/\zeta]$.
\end{lemma}

\begin{proof}
By induction on $\Sigma,\zeta \vdash \kappa <:: \kappa'$.

\begin{description}
\item[Case \begin{sc}SK-StrTree\end{sc}]:~\\
The premise $\Sigma,\zeta \vdash d \leq d'$ means that either 
\begin{enumerate}
\item Both $d$ and $d'$ are closed, or
\item $d = d' = \mathit{Literal}~\zeta'$ where $\zeta' \in \Sigma$, or
\item $d = d' = \mathit{Literal}~\zeta$
\end{enumerate}

If 1 or 2 then $d[S/\zeta] = d$, $d'[S/\zeta] = d'$, and $\Sigma \vdash d \leq d'$, from which we conclude 
$$\Sigma \vdash d[S/\zeta] \leq d'[S/\zeta]$$

If 3 holds then $d[S/\zeta] = d'[S/\zeta] = \mathit{Literal}~S$. Applying \begin{sc}SD-LitLit\end{sc} gives
$\Sigma \vdash d[S/\zeta] \leq d'[S/\zeta]$. 

TODO: The above discriminator substitution should be moved to a different lemma.

Applying \begin{sc}SK-StrTree\end{sc} gives 
$$\Sigma \vdash \mathit{KString(d[S/\zeta],\phi) <:: \mathit{KTree}(d'[S/\zeta])}$$

\end{description}
\end{proof}

\begin{lemma}
If $\Sigma,\zeta \vdash \kappa <:: \star$ and $\Sigma \vdash S$ then
$\Sigma \vdash \kappa[S/\zeta] <:: \star$.
\end{lemma}

\begin{proof}
By induction on the structure of $\Sigma \vdash \kappa$.

\begin{description}
\item[Case a]~\\
hi
\end{description}

\end{proof}

\subsection*{Kinding index substitution Lemma}

\begin{lemma}
If $$\Sigma,\zeta \mid \Gamma @ A \vdash \tau :: \kappa$$ and $$\Sigma \vdash S$$ then
$$\Sigma \mid \Gamma @ A[S/\zeta] \vdash \tau[S/\zeta ] :: \kappa [S/\zeta ]$$
\end{lemma}

\begin{proof}
By induction on the proof $\Sigma,\zeta \mid \Gamma @ A \vdash \tau :: \kappa$.

\begin{description}
\item[Case \sc{Record}]:~\\
The inductive hypothesis gives $$\Sigma \mid \Gamma @ A[S/\zeta] \vdash \tau[S/\zeta] :: \mathit{KString}([S/\zeta]d,\phi)$$
and
$$\Sigma \mid \Gamma @ B_i[S/\zeta] \vdash \tau_i[S/\zeta] :: \iota_i[S/\zeta]^{~i \in 1..n}$$
and
$$\Sigma \mid \Gamma@ A_i[S/\zeta] \vdash \tau_i'[S/\zeta] :: \kappa_i[S/\zeta]^{~i \in 1..n}$$

\end{description}

\end{proof}

\subsection*{Soundness}

\subsubsection*{String abstraction applications}

We must prove that $\Sigma \mid \Gamma @ A \vdash (\lambda \zeta.\tau)~[S] :: \kappa[S/\zeta]$ implies 
$$\Sigma \mid \Gamma @ A \vdash \tau[S/\zeta] :: \kappa[S/\zeta]$$
We must further prove that $$\sem{\Sigma \mid \Gamma @ A \vdash (\lambda \zeta.\tau)~[S] :: \kappa[S/\zeta]}$$
is equal to $$\sem{\Sigma \mid \Gamma @ A \vdash \tau[S/\zeta] :: \kappa[S/\zeta]}$$
We do so by induction on the structure of $\tau$.

\section*{Additional ideas}

Rather than just using a list of variables for the string index context, we should classify each
string index variable by the set of possible characters that may occur in the string. 

\end{document}
  
