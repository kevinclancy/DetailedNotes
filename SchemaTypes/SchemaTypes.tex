\documentclass{article}
 
\usepackage{amscd}
\usepackage{graphicx}
\usepackage{amsmath}
\usepackage{amssymb}
\usepackage{amsthm}
\usepackage{stmaryrd}
\usepackage{mathpartir}
\usepackage{multicol}
\usepackage{enumitem}
\usepackage[section]{placeins} % float barriers
\usepackage{natbib}
\usepackage{xcolor} 
\usepackage{bussproofs} 
\usepackage{diagrams}
\usepackage{tikz}
 
\usetikzlibrary{cd}
 
\newtheorem{lemma}{Lemma}

%example: \limit{j \in J}{F_j}
\newcommand{\limit}[2]{\underset{\overset{\longleftarrow}{#1}}{\text{lim}}~#2}
\newcommand{\lims}[1]{\underset{\longleftarrow}{\text{lim}}~#1}
\newcommand{\mbf}{\mathbf}
\newcommand{\sem}[1]{\llbracket #1 \rrbracket}
\newcommand{\defeq}{\overset{\mathit{def}}{=}}


\newcommand{\vrt}[2]{
\pile{
#1 \\
\downarrow \\
#2
}
}

\newcommand{\ddisp}[3]{
\left(
\scriptsize
\begin{tikzcd}
#1 \ar[d, "\footnotesize{#2}"] \\
#3
\end{tikzcd}
\normalsize
\right)
}

\newcommand{\disp}[3]{
\left(
\tiny
\begin{array}{c}
#1 \\
\downarrow\\
#3
\end{array}
\begin{array}{l}
~ \\
#2 \\
~
\end{array}
\normalsize
\right)
}

\newcommand{\dispp}[3]{
\tiny
\begin{tikzcd}
#1 \ar[d, "#2"] \\
#3
\end{tikzcd}
\normalsize
}

\title{Schema Type Calculus}

\begin{document}

\maketitle

\section*{Syntax}

\begin{tabular}{llll}
$\mathit{Chars}$ & $=$ & the set of all characters \\
$\mathit{Vars}$ & $=$ & the set of all variables \\
$\alpha$ & $\in$ & $\mathit{Vars}$ & ~ \\
$c$ & $\in$ & $\mathit{Chars}$ & (the finite set of all characters) \\
$\phi$ & $\in$ & $\mathcal P(\mathit{Chars})$ & (a subset of the set of characters) \\
$s,t$ & $\in$ & $\mathit{Strings}$ & (where $\mathit{Strings} = \mathit{Chars}^{\star}$) \\~\\
$S$ (string index) & $::=$ & $s$ & (literal string) \\
                   & $\mid$ & $\zeta$ & (string variable) \\~\\
$d$ (discriminator) & $::=$  & $\mathit{Prefix}~s$ & (values begin with $s$) \\
    & $\mid$ & $\mathit{Literal}~s$ & (values are equal to $s$) \\
 & & \\
$m$ (multiplicity) & $::=$ & $1 \mid \mathit{set} \mid \mathit{opt}$\\~\\
$\tau,\sigma$ (type) & $::=$ & $\mathit{tuple} \{ s_i : \tau_i : c_i^{~i \in 1..n} \}$ & (pieced string) \\
       & $\mid$ & $S$ & (string index) \\
       & $\mid$ & $(S_1,\ldots,S_n).[\tau]$ & (path type) \\
       & $\mid$ & $\biguplus \tau_i^{~i \in 1..n}$ & (union of disjoint types) \\ 
       & $\mid$ & $\tau.\{[\sigma_i] :^{m_i} \tau_i ^{~i \in 1..n}\}$ & (record type - $\mathit{root}.\{ fields \}$) \\ 
       & $\mid$ & $\forall (\alpha :: \kappa). \tau$ & (type abstraction) \\
       & $\mid$ & $\forall \zeta. \tau$ & (string abstraction) \\
       & $\mid$ & $\tau~[S]$ & (index application) \\
       & $\mid$ & $\tau~\tau$ & (type application) \\
       & $\mid$ & $\alpha$ & (type variable) \\~\\
$\kappa, \iota$ (kind) & $::=$ & $\mathit{KString}(d,\phi)$ & (Kind of string types, chars in vals drawn from $\phi$) \\
                         & $\mid$ & $\mathit{KTree}(d)$ & (Kind of tree types w/ root discriminator $d$) \\
                         & $\mid$ & $\kappa \overset{A}{\Rightarrow} \kappa$ & (Kind of type abstractions) \\
                         & $\mid$ & $\zeta \Rightarrow \kappa$ & (Kind of string abstractions) \\~\\
$\Sigma$ (index context) & $::=$ & $\Sigma,\zeta$ & (context extension) \\
                         & $\mid$ & $\emptyset$ & (empty context) \\~\\
$\Gamma$ (kind context) & $::=$ & $\Gamma,x :: \kappa$ & (context extension) \\
                           & $\mid$ & $\emptyset$ & (empty context) \\~\\

\end{tabular}\\~\\~\\
\begin{tabular}{llll}

\end{tabular}

\section*{Index well-formedness}

\begin{mathpar}
\inferrule
  {\zeta \in \Sigma}
  {\Sigma \vdash \zeta}
\and
\inferrule
  {~}
  {\Sigma \vdash s}
\end{mathpar}
The interpretation $\sem{\Sigma}$ of an index context $\Sigma = \zeta_1, \ldots, \zeta_n$ is the set of all functions from $\{ \zeta_1, \ldots, \zeta_n \}$ to strings. The interpretation $\sem{\Sigma \vdash S}$ of a well-formed index judgment is a function from $\sem{\Sigma}$ to the set of all strings, i.e. $\sem{\Gamma \vdash S} : \sem{\Gamma} \to \mathit{Strings}$, defined as follows:

\begin{mathpar}
\sem{\Sigma \vdash \zeta} \doteq \sem{\Sigma} \overset{\pi_{\zeta}}{\longrightarrow} \mathit{Strings}
\and
\sem{\Sigma \vdash s} \doteq \sem{\Sigma} \overset{!}{\longrightarrow} 1 \overset{s}{\longrightarrow} \mathit{Strings}
\end{mathpar}

\section*{Discriminator well-formedness}

\begin{mathpar}
\inferrule
  {\Sigma \vdash S}
  {\Sigma \vdash \mathit{Prefix}~S}
\and
\inferrule
  {\Sigma \vdash S}
  {\Sigma \vdash \mathit{Literal}~S}
\end{mathpar}

The interpretation $\sem{\Sigma \vdash d}$ of a discriminator well-formedness judgment is a function from $\sem{\Sigma}$ into $\mathcal P(\mathit{Strings})$. Defining $\mathit{pref},\mathit{lit} : \mathit{Strings} \to \mathcal P(\mathit{Strings})$ such that
\begin{mathpar}
\mathit{pref}(s) \doteq \{ t \mid t~\text{is a prefix of}~s \}
\and
\mathit{lit}(s) \doteq \{ s \}
\end{mathpar}
we have
\begin{mathpar}
\inferrule 
  {\sem{\Sigma \vdash S} = f}
  {\sem{\Sigma \vdash \mathit{Prefix}~S} \doteq f;\mathit{pref}}
\and
\inferrule
  {\sem{\Sigma \vdash S} = f}
  {\sem{\Sigma \vdash \mathit{Literal}~S} \doteq f;\mathit{lit}}
\end{mathpar}

\section*{Discriminator subsumption}

\begin{mathpar}
\inferrule
  {s' \sqsubseteq s}
  {\Sigma \vdash \mathit{Prefix}~s \leq \mathit{Prefix}~s'}
\and
\inferrule
  {\Sigma \vdash S}
  {\Sigma \vdash \mathit{Literal}~S \leq \mathit{Literal}~S}
\and
\inferrule
  {s' \sqsubseteq s}
  {\Sigma \vdash \mathit{Literal}~s \leq \mathit{Prefix}~s'}
\and
\end{mathpar}

Above, $s' \sqsubseteq s$ means ``$s'$ is a prefix of $s$''.
Ordering $\mathcal P(\mathit{Strings})$ by inclusion and the function space $\sem{\Sigma} \to \mathcal P(\mathit{Strings})$ pointwise, 
The interpretation $\sem{\Sigma \vdash d \leq d'}$ of a subsumption judgment is a proof that that $\sem{\Sigma \vdash d} \leq \sem{\Sigma \vdash d'}$.

\section*{Kind well-formedness}

\begin{mathpar}
\inferrule
  {\Sigma \vdash d}
  {\Sigma \vdash \mathit{KString(d,\phi)}}
\and
\inferrule
  {\Sigma \vdash d}
  {\Sigma \vdash \mathit{KTree(d)}}
\and
\inferrule
  {\Sigma \vdash \kappa \\ \Sigma \vdash \kappa' \\ \Sigma \vdash A}
  {\Sigma \vdash \kappa \overset{A}{\Rightarrow} \kappa'}
\and
\inferrule
  {\Sigma,\zeta \vdash \kappa}
  {\Sigma \vdash \zeta \Rightarrow \kappa}
\end{mathpar}

The interpretation $\sem{\Sigma \vdash \kappa}$ of a kind well-formedness judgment is an object of the total 
category $\mathit{Fam(\mbf{Posets})}$ in the fibre over $\sem{\Sigma}$, i.e. it is a $\sem{\Sigma}$-indexed family of posets, defined as follows.

\begin{mathpar}
\inferrule
  {\sem{\Sigma \vdash d} = f}
  {\sem{\Sigma \vdash \mathit{KString}(d,\phi)} \doteq UHHH}
\end{mathpar}

\section*{Subkinding}

\begin{mathpar}
\inferrule
  {\Sigma \vdash d \leq d'}
  {\Sigma \vdash \mathit{KString}(d,\phi) <:: \mathit{KTree}(d)}
\and
\inferrule
  {\Sigma \vdash d \leq d' \\ \Sigma \vdash \phi \subseteq \phi'}
  {\Sigma \vdash \mathit{KString}(d,\phi) <:: \mathit{KString}(d',\phi')}
\and
\inferrule
  {\Sigma \vdash d \leq d'}
  {\Sigma \vdash \mathit{KTree}(d) <:: \mathit{KTree}(d')}
\end{mathpar}

\section*{Kinding}

In what follows, we define $\ast \doteq \mathit{KTree}(\mathit{Prefix}~\epsilon)$, where $\epsilon$ is the empty string, so that all proper types have kind $\ast$. We define $\star \doteq \mathit{KString}(\mathit{Prefix}~\epsilon, \mathit{Chars})$, so that all string types have kind $\star$.

\begin{mathpar}
\inferrule
  {\Sigma \mid \Gamma@A_i \vdash \tau_i :: \kappa_i^{~i \in 1..n} \\ 
   \kappa_i <:: \mathit{KString}(d_i,\phi_i)^{~i \in 1..n} \\
   c_i \not \in \phi_i^{~i \in 1..n} }
  {\Sigma \mid \Gamma@(\bigcup_{i \in 1..n} A_i) \vdash \mathit{tuple} \{ s_i : \tau_i : c_i^{~i \in 1..n}\} :: \mathit{KString}(d_1, (\bigcup_{i \in 1..n} \phi_i) \cup \{ c_1, \ldots, c_n \})}
\and
\inferrule
  {\Sigma \mid \Gamma@A_i \vdash \tau_i :: \mathit{KTree}(d_i)^{~i \in 1..n} \\ \mathit{disjoint} \{ d_i^{~i \in 1..n} \} }
  {\Sigma \mid \Gamma@(\bigcup_{i \in 1..n} A_i) \vdash \biguplus_{i \in 1..n} \tau_i :: \mathit{KTree}(\mathit{merge}~\{ d_i^{~i \in 1..n} \})}
\and
\inferrule
  {\Sigma \mid \Gamma@A \vdash \tau :: \mathit{KString}(d,\phi) \\ 
    \Sigma \mid \Gamma@B_i \vdash \sigma_i :: \iota_i^{~i \in 1..n} \\ \iota_i <:: \star^{~i \in 1..n} \\ 
    \Sigma \mid \Gamma@A_i \vdash \tau_i :: \kappa_i^{~i \in 1..n} \\ \kappa_i <:: \ast^{~i \in 1..n}}
  {\Sigma \mid \Gamma@(A \cup \bigcup_{i \in 1..n} A_i \cup B_i) \vdash \tau.\{[\sigma_i] :^{m_i} \tau_i ^{~i \in 1..n}\} :: \mathit{KTree}(d)}
\and
\inferrule
  {\Sigma \vdash S_i^{~i \in 1..n} \\ 
   \Sigma \mid \Gamma@A \vdash \tau :: \kappa \\ \kappa <:: \star}
  {\Sigma \mid \Gamma@(\{ (S_1,\ldots,S_n).[\tau] \} \cup A) \vdash (S_1,\ldots,S_n).[\tau] :: \kappa}
\and
\inferrule
  {\Sigma, \zeta \mid \Gamma @ \emptyset \vdash \tau :: \kappa}
  {\Sigma \mid \Gamma @ \emptyset \vdash \forall \zeta. \tau :: \zeta \Rightarrow \kappa}
\and
\inferrule
  {\Sigma \mid \Gamma,\alpha :: \kappa @ A \vdash \tau :: \kappa'}
  {\Sigma \mid \Gamma @ \emptyset \vdash \forall (\alpha :: \kappa).\tau :: \kappa \overset{A}{\Rightarrow} \kappa'}
\and
\inferrule
  {\Sigma \mid \Gamma @ A \vdash \tau :: \kappa \overset{C}{\Rightarrow} \iota \\ 
   \Sigma \mid \Gamma @ B \vdash \sigma :: \kappa}
  {\Sigma \mid \Gamma @ (A \cup B \cup C) \vdash \tau~\sigma :: \iota}
\and
\inferrule
  {\Sigma \mid \Gamma @ A \vdash \tau :: \zeta \Rightarrow \tau \\ \Sigma \vdash S}
  {\Sigma \mid \Gamma @ A \vdash \tau~[S] :: \iota[S/\zeta]}

\end{mathpar}

\end{document}
  
