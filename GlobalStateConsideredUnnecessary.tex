\documentclass{article}
 
\usepackage{amscd}
\usepackage{graphicx}
\usepackage{amsmath}
\usepackage{amssymb}
\usepackage{amsthm}
\usepackage{stmaryrd}
\usepackage{mathpartir}
\usepackage{multicol}
\usepackage{enumitem}
\usepackage[section]{placeins} % float barriers
\usepackage{natbib}
\usepackage{xcolor} 
\usepackage{bussproofs} 
\usepackage{diagrams}
\usepackage{tikz}

\usetikzlibrary{cd}
 
\newtheorem{lemma}{Lemma}

%example: \limit{j \in J}{F_j}
\newcommand{\limit}[2]{\underset{\overset{\longleftarrow}{#1}}{\text{lim}}~#2}
\newcommand{\lims}[1]{\underset{\longleftarrow}{\text{lim}}~#1}
\newcommand{\mbf}{\mathbf}

\newcommand{\inconsist}{\mathrel{\substack{\smile \\ \frown}}}
\newcommand{\consist}{\mathrel{\substack{\frown \\ \smile}}}

\newcommand{\vrt}[2]{
\pile{
#1 \\
\downarrow \\
#2
}
}

\newcommand{\ddisp}[3]{
\left(
\scriptsize
\begin{tikzcd}
#1 \ar[d, "\footnotesize{#2}"] \\
#3
\end{tikzcd}
\normalsize
\right)
}

\newcommand{\disp}[3]{
\left(
\tiny
\begin{array}{c}
#1 \\
\downarrow\\
#3
\end{array}
\begin{array}{l}
~ \\
#2 \\
~
\end{array}
\normalsize
\right)
}

\newcommand{\dispp}[3]{
\tiny
\begin{tikzcd}
#1 \ar[d, "#2"] \\
#3
\end{tikzcd}
\normalsize
}

\title{Notes for: \emph{Global State Considered Unnecessary} by Uday Reddy}

\begin{document}

\maketitle

\section*{Coherence Spaces}

\begin{lemma}
Let $\mbf{CohL}$ be the category of coherence spaces and linear maps. We define $- \otimes- : \mbf{CohL} \times \mbf{CohL} \to \mbf{CohL}$ for coherence spaces $A$ and $B$ such that $|A \otimes B| \doteq \{ (a,b) \mid a \in |A|, b \in |B| \}$ and $(a,b) \consist (a',b')~\dot{\Leftrightarrow}~(a \consist a') \wedge (b \consist b')$. For linear maps $f : A \to_L C$ and $g : B \to_L D$ we define 
$f \otimes g : A \otimes B \to_L C \otimes D \doteq \{ (a,b) \mapsto (c,d) \mid (a,c) \in f, (b,d) \in g \}$. 
Then $- \otimes - : \mbf{CohL} \times \mbf{CohL} \to \mbf{CohL}$ is a functor.
\end{lemma}

\begin{proof}
First, for coherence spaces $A$ and $B$ we have 
$$\mathit{id}_A \otimes \mathit{id}_B = \{ (a,b) \mid a \in |A|, b \in |B| \} = \{ (a,b) \mid (a,b) \in |A \otimes B| \} = \mathit{id}_{A \otimes B}$$ 
Second, for $(f,g) : (A,B) \to (C,D)$ and $(h,k) : (C,D) \to (X,Y)$ we have \\~\\
\begin{tabular}{ll}
$(f;h \otimes g;k)$ & $= \{ (a,b) \mapsto (x,y) \mid (a,x) \in f;h \wedge (b,y) \in g;k \} $ \\
~ & $= \{ (a,b) \mapsto (x,y) \mid (\exists c \in C. (a \mapsto c) \in f \wedge (c \mapsto x) \in h) \wedge (\exists d \in D. (b \mapsto d) \in g \wedge (d \mapsto y) \in k) \}$ \\ 
~ & $= \{ (a,b) \mapsto (x,y) \mid \exists (c,d) \in C \times D.~(a,b) \mapsto (c,d) \in (f \otimes g) \wedge 
  (c,d) \mapsto (x,y) \in (h \otimes k)\} $ \\
~ & $= (f \otimes g);(h \otimes k)$
\end{tabular}

\end{proof}

\section*{Appendix A: Categorical Details}

In the definition of the $\dagger-$ comonad, the first thing to note is that he is specifying these natural transformations as points in coherence spaces. The notation for the definition of $\mathit{dup}_A$ needs clarification. (At least, I needed clarification when I first saw it.) Each $s_i$ is an element of $| \dagger A |$, and $s_1 \cdots s_n$ is their concatenation, which is also an element of $| \dagger A|$. So he is quantifying over all natural numbers $n$ and all partitions $s_1 \cdots s_n$ of an arbitrary string in $| \dagger A|$.

The function corresponding to the point $\mathit{read}_A$ of 
$\dagger A \multimap A$ is:
$$\mathit{read_A}(S) \doteq \{ a \mid \langle a \rangle \in S \}$$
(It's instructive to consider the above definition in the special case that $S$ is an \emph{active} point of $\dagger A$.)

The function corresponding to the point $\mathit{dup}_A$ of $\dagger A \multimap \dagger \dagger A$ is:
$$\mathit{dup}_A(S) \doteq \{ \langle s_1,\ldots,s_n \rangle \mid n \in \mathbb N, s_1 s_2 \cdots s_n \in S \} $$

I don't think $\dagger -$'s functorial mapping on functions was defined anywhere. Here's my guess: letting $f : A \to B$ be a linear function from coherence space $A$ to coherence space $B$, we obtain a linear function $\dagger f : \dagger A \to \dagger B$ defined as:
$$\dagger f \in (\dagger A \multimap \dagger B) \doteq \{ ( \langle a_1,\ldots,a_n \rangle, \langle b_1, \ldots, b_n \rangle)  \mid (a_1,b_1),\ldots,(a_n,b_n) \in f \}$$
 
\begin{lemma}
$\mathit{read_A} : \dagger A \to A$ is a natural transformation.
\end{lemma}

\begin{proof}
Let $f : A \multimap B$ be a stable linear function between coherence spaces $A$ and $B$. Then,\\~\\
\begin{tabular}{ll}
$(z,b) \in (\mathit{read}_A;f)$ & $\mathit{iff}$ \\
$\exists a \in A.~((z,a) \in \mathit{read}_A \wedge (a,b) \in f) $ & iff \\
$\exists a \in A.~(z = \langle a \rangle \wedge (a,b) \in f)$ & iff \\

$\exists b \in B.~(z,\langle b \rangle) \in \dagger f$ & iff \\
$(z,b) \in (\dagger f;\mathit{read}_b)$ & ~ 
\end{tabular} 
\end{proof}

\begin{lemma}
$\mathit{dup}_A : \dagger A \to \dagger \dagger A$ is a natural transformation. 
\end{lemma}

\begin{proof}
Intuitively we can see that this is true since partitioning and then mapping through $f$ gives us the same results as mapping through $f$ and then partitioning.\\~\\

Let $f : A \to B$ be a stable linear function between coherence spaces $A$ and $B$. Then,\\~\\
\begin{tabular}{ll}
$(s,z) \in (\mathit{dup}_A; \dagger \dagger f)$ & iff \\~\\
 ~ & ~ \\
$s = t_1 \cdots t_m$ and $(\langle t_1, \ldots, t_m \rangle,z) \in \dagger \dagger f$ & iff \\~\\
 ~ & ~ \\
$s = t_1 \cdots t_m$ and $\forall i \in 1..m.~t_i = a_{i1} \cdots a_{i k_i}$ and \\ 
$z = \langle r_i, \ldots, r_m \rangle$ and $\forall i \in 1..m.~r_i = b_{i1} \cdots b_{i k_i}$ and $\forall i j.~(a_{ij}, b_{ij}) \in f$ & iff \\~\\
$(s,z) \in (\dagger f; \mathit{dup}_{B})$
\end{tabular}

\end{proof}

Next we show that the comonad laws hold.

\begin{lemma}
For all coherence spaces $A$, $\mathit{dup}_A;\dagger(\mathit{read}_A) = \mathit{id}_{\dagger A}$.
\end{lemma}

\begin{proof}~\\~\\
I'm going to elide the proof of this, but it's easy to see since the partition of a sequence is only in the domain of the point $\dagger (read_A)$ if it consists solely of singletons.
\end{proof}

\begin{lemma}
For all coherence spaces $A$, $\mathit{dup}_A;\mathit{read}_{\dagger A} = \mathit{id}_{\dagger A}$.
\end{lemma}

\begin{proof}
The only partition of $\langle a_1, \ldots, a_k \rangle$ in the domain of the point $\mathit{read}_{\dagger A}$ is trivial ``non-partition'' with $n=1$: $\langle \langle a_1, \ldots, a_k \rangle \rangle$. 
\end{proof}

\begin{lemma}
For all coherence spaces $A$, $\mathit{dup}_A;\mathit{dup}_{\dagger A} = \mathit{dup}_A;\dagger(\mathit{dup}_A)$.
\end{lemma}

\begin{proof}
Intuitively, this says that any token of $\dagger \dagger A$ obtained by partitioning an element $s \in \dagger A$
twice can also be obtained by paritioning $s$ and then partitioning each element of the resulting list. This 
seems obvious enough to warrant eliding the proof.
\end{proof}

We now show that our friends $\mathit{merge}_{A,B}$ are natural transformations.\\~\\
Recall that for coherence spaces $A$ and $B$, $\mathit{merge}_{A,B}$ is defined as the following point of $\dagger A \otimes \dagger B \multimap \dagger(A \otimes B)$:
 
$$\mathit{merge}_{A,B} \doteq \{ (\langle a_1, \ldots, a_n \rangle, \langle b_1, \ldots, b_n \rangle) \mapsto 
 \langle (a_1, b_1),\ldots,(a_n,b_n) \rangle \mid a_1, \ldots, a_n \in |A| \text{ and } b_1, \ldots, b_n \in |B|  \}$$

\begin{lemma}
$\mathit{merge}_{A,B} : (\dagger - \otimes \dagger -) \to \dagger(- \otimes -)$ is a natural transformation. (Between functors of type $\mbf{CohL} \times \mbf{CohL} \to \mbf{CohL}$.)
\end{lemma}

\begin{proof}
We must show that it is natural in each component separately, i.e. that for $\alpha : A \to C$ and $\beta : B \to D$ we have $$\dagger A \otimes \dagger \beta;\mathit{merge} = \mathit{merge};\dagger(A \otimes \beta)$$
and $$\dagger \alpha \otimes \dagger B;\mathit{merge} = \mathit{merge};\dagger(\alpha \otimes B)$$
for then by the bifunctor lemma (see MiscStudy pg 2817) we have~\\

\begin{tabular}{ll}
$(\dagger \alpha \otimes \dagger \beta);\mathit{merge}$ & $= (\dagger \alpha \otimes \dagger B);(\dagger A \otimes \dagger \beta);\mathit{merge}$ \\
 ~ & $= (\dagger \alpha \otimes \dagger B);\mathit{merge};\dagger(A \otimes \beta)$ \\
 ~ & $= \mathit{merge};\dagger(\alpha \otimes B);\dagger(A \otimes \beta)$ \\
 ~ & $= \mathit{merge};\dagger(\alpha \otimes \beta)$
\end{tabular}

a
\end{proof}


\end{document}



