\documentclass{article}
 
\usepackage{amscd}
\usepackage{graphicx}
\usepackage{amsmath}
\usepackage{amssymb}
\usepackage{amsthm}
\usepackage{stmaryrd}
\usepackage{mathpartir}
\usepackage{multicol}
\usepackage{enumitem}
\usepackage[section]{placeins} % float barriers
\usepackage{natbib}
\usepackage{xcolor} 
\usepackage{bussproofs} 
\usepackage{diagrams}
\usepackage{tikz}

\usetikzlibrary{cd}
 
\newtheorem{lemma}{Lemma}

%example: \limit{j \in J}{F_j}
\newcommand{\limit}[2]{\underset{\overset{\longleftarrow}{#1}}{\text{lim}}~#2}
\newcommand{\lims}[1]{\underset{\longleftarrow}{\text{lim}}~#1}
\newcommand{\mbf}{\mathbf}


\newcommand{\vrt}[2]{
\pile{
#1 \\
\downarrow \\
#2
}
}

\newcommand{\ddisp}[3]{
\left(
\scriptsize
\begin{tikzcd}
#1 \ar[d, "\footnotesize{#2}"] \\
#3
\end{tikzcd}
\normalsize
\right)
}

\newcommand{\disp}[3]{
\left(
\tiny
\begin{array}{c}
#1 \\
\downarrow\\
#3
\end{array}
\begin{array}{l}
~ \\
#2 \\
~
\end{array}
\normalsize
\right)
}

\newcommand{\dispp}[3]{
\tiny
\begin{tikzcd}
#1 \ar[d, "#2"] \\
#3
\end{tikzcd}
\normalsize
}

\title{Notes for: \emph{Global State Considered Unnecessary} by Uday Reddy}

\begin{document}

\maketitle


\section*{Appendix A: Categorical Details}

In the definition of the $\dagger-$ comonad, the first thing to note is that he is specifying these natural transformations as points in coherence spaces. The notation for the definition of $\mathit{dup}_A$ needs clarification. After careful consideration of the needs of the coKlesli category for $\dagger -$, the domain of $\mathit{dup}$ needs to contain arbitrary sequences of sequences. Intuitively, a token of $\dagger A$ is a sequences of usages (e.g. method calls) of an object. In Reddy's notation, each $s_i$ is an element of $| \dagger A |$, and $s_1 \cdots s_n$ is their concatenation, which is also an element of $| \dagger A|$. So he is quantifying over all natural numbers $n$ and all partitions $s_1 \cdots s_n$ of an arbitrary string in $| \dagger A|$.  $\langle a_1, \ldots, a_n \rangle \mapsto \langle \langle a_1 \rangle, \ldots, \langle a_n \rangle \rangle$. Below I used the wrong definition of $\mathit{dup}$, which is why I couldn't prove all of the comonad laws. Will fix soon.

The function corresponding to the point $\mathit{read}_A$ of 
$\dagger A \multimap A$ is:
$$\mathit{read_A}(S) \doteq \{ a \mid \langle a \rangle \in S \}$$
(It's instructive to consider the above definition in the special case that $S$ is an \emph{active} point of $\dagger A$.)

The function corresponding to the point $\mathit{dup}_A$ of $\dagger A \multimap \dagger \dagger A$ is:
$$\mathit{dup}_A(S) \doteq \{ \langle \langle a_1 \rangle, \ldots, \langle a_n \rangle \rangle \mid \langle a_1, \ldots, a_n \rangle \in S \} $$

I don't think $\dagger -$'s functorial mapping on functions was defined anywhere. Here's my guess: letting $f : A \to B$ be a linear function from coherence space $A$ to coherence space $B$, we obtain a linear function $\dagger f : \dagger A \to \dagger B$ defined as:
$$\dagger f \in (\dagger A \multimap \dagger B) \doteq \{ ( \langle a_1,\ldots,a_n \rangle, \langle b_1, \ldots, b_n \rangle)  \mid (a_1,b_1),\ldots,(a_n,b_n) \in f \}$$
 
\begin{lemma}
$\mathit{read_A} : \dagger A \to A$ is a natural transformation.
\end{lemma}

\begin{proof}
Let $f : A \multimap B$ be a stable linear function between coherence spaces $A$ and $B$. Then,\\~\\
\begin{tabular}{ll}
$(z,b) \in (\mathit{read}_A;f)$ & $\mathit{iff}$ \\
$\exists a \in A.~((z,a) \in \mathit{read}_A \wedge (a,b) \in f) $ & iff \\
$\exists a \in A.~(z = \langle a \rangle \wedge (a,b) \in f)$ & iff \\

$\exists b \in B.~(z,\langle b \rangle) \in \dagger f$ & iff \\
$(z,b) \in (\dagger f;\mathit{read}_b)$ & ~ 
\end{tabular} 
\end{proof}

\begin{lemma}
$\mathit{dup}_A : \dagger A \to \dagger \dagger A$ is a natural transformation. 
\end{lemma}

\begin{proof}

Let $f : A \to B$ be a stable linear function between coherence spaces $A$ and $B$. Then,\\~\\
\begin{tabular}{ll}
$(\langle a_1,\ldots,a_n \rangle,t) \in (\mathit{dup}_A; \dagger \dagger f)$ & iff \\
$t = \langle \langle b_1 \rangle,\ldots, \langle b_n \rangle \rangle$ with $(a_1,b_1),\ldots,(a_n,b_n) \in f$ & iff \\
$(\langle a_1,\ldots,a_n \rangle,t)  \in (\dagger f; dup_B)$
\end{tabular}~\\

\end{proof}

Next we show that the comonad laws hold.

\begin{lemma}
For all coherence spaces $A$, $\mathit{dup}_A;\dagger(\mathit{read}_A) = \mathit{id}_{\dagger A}$.
\end{lemma}

\begin{proof}~\\~\\
\begin{tabular}{ll}
$(\langle a_1, \ldots, a_n \rangle, \langle a_1', \ldots, a_n' \rangle) \in \mathit{dup}_A;\dagger(\mathit{read}_A)$ & iff \\
$(\langle \langle a_1 \rangle, \ldots, \langle a_n \rangle \rangle, \langle a_1', \ldots, a_n' \rangle) \in \dagger(\mathit{read}_A)$ & iff \\
$(\langle a_1 \rangle, a_1'),\ldots,(\langle a_n \rangle, a_n') \in \mathit{read_A}$ & iff \\
$a_1 = a_1', \ldots, a_n = a_n'$ & iff \\
$(\langle a_1, \ldots, a_n \rangle, \langle a_1', \ldots, a_n' \rangle) \in \mathit{id}_{\dagger A}$
\end{tabular}~\\
\end{proof}

\begin{lemma}

\end{lemma}

\end{document}



