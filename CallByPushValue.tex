\documentclass{article}

\usepackage{amscd}
\usepackage{graphicx}
\usepackage{amsmath}
\usepackage{amssymb}
\usepackage{amsthm}
\usepackage{stmaryrd}
\usepackage{mathpartir}
\usepackage{multicol}
\usepackage{enumitem}
\usepackage[section]{placeins} % float barriers
\usepackage{natbib}
\usepackage{xcolor} 
\usepackage{bussproofs} 
\usepackage{diagrams}
\usepackage{tikz}

\usetikzlibrary{cd}

\newtheorem{lemma}{Lemma}

%example: \limit{j \in J}{F_j}
\newcommand{\limit}[2]{\underset{\overset{\longleftarrow}{#1}}{\text{lim}}~#2}
\newcommand{\lims}[1]{\underset{\longleftarrow}{\text{lim}}~#1}
\newcommand{\mbf}{\mathbf}
\newcommand{\sem}[1]{\llbracket #1 \rrbracket}
\newcommand{\gr}[1]{\textcolor{blue}{#1}}
\newcommand{\bk}[1]{\textcolor{black}{#1}}

\newcommand{\vrt}[2]{
\pile{
#1 \\
\downarrow \\
#2
}
}

\newcommand{\ddisp}[3]{
\left(
\scriptsize
\begin{tikzcd}
#1 \ar[d, "\footnotesize{#2}"] \\
#3
\end{tikzcd}
\normalsize
\right)
}

\newcommand{\disp}[3]{
\left(
\tiny
\begin{array}{c}
#1 \\
\downarrow\\
#3
\end{array}
\begin{array}{l}
~ \\
#2 \\
~
\end{array}
\normalsize
\right)
}

\newcommand{\dispp}[3]{
\tiny
\begin{tikzcd}
#1 \ar[d, "#2"] \\
#3
\end{tikzcd}
\normalsize
}

\title{Notes: \emph{Call-By-Push-Value: A Functional/Imperative Synthesis} }

\begin{document}

\maketitle
 
\section*{2}

\subsection*{2.6}

Levy gives a full categorical semantics in chapter 9. I have not yet read chapter 9, but I'd like to gain a complete understanding of chapter 2 before moving forward. To do this, I'm going to to try to follow the bread crumb trail of categorical details in chapter 2 and try to fill in the blanks. 

Here we give categorical semantics for all CBPV types and judgments. Categorical constructs are written in \gr{gray} to distinguish them from the syntax of cbpv. It is given with respect to a distributive category $\gr{\mathcal C}$ and a strong monad $\gr{T}$ such that the exponential $\gr{A \Rightarrow B}$ exists whenever $\gr{B}$ is the carrier of a $\gr{T}$-algebra. Additionally we have a pair of adjoint functors $\gr{F : \mathcal C \leftrightarrows \mathcal \mathit{TAlg} : U}$, whose resulting monad $\gr{UF}$ is has a strength $\gr{s_{A,B} : A \times UFB \to UF(A \times B)}$. We write $\gr{t}$ for the strength of $\gr{T}$.

A value $\Gamma \vdash^v V : A$ is interpreted as a morphism of type $\gr{\sem{\bk{\Gamma}} \to \sem{\bk{A}}}$. A 
computation $\Gamma \vdash^c M : \underline{B}$ is interpreted as a morphism of type $\gr{\sem{\bk{\Gamma}} \to U\sem{\bk{\underline B}}}$ 

\subsubsection*{Value Types}

Value types, written as $\gr{A,B,C}$ are interpreted as $C$ objects.

\begin{mathpar}
\inferrule
  {\sem{\underline{B} : \mathit{CompTy}} = (\gr{A}, \gr{f})}
  {\sem{U\underline{B} : \mathit{ValTy}} \doteq \gr{A}}
\and
\inferrule
  {\sem{B_i : \mathit{ValTy}} = \gr{A_i}^{i \in I} \\ \text{I is a finite set}}
  {\sem{\Sigma_{i \in I} A_i : \mathit{ValTy}} \doteq \gr{\Sigma}_{i \in I} \gr{A_i}}
\and
\inferrule
  {~}
  {\sem{1 : \mathit{ValTy}} = \gr{1}}
\and
\inferrule
  {\sem{A : \mathit{ValTy}} = \gr{A} \\ \sem{A' : \mathit{ValTy}} = \gr{A'}}
  {\sem{A \times A' : \mathit{ValTy}} = \gr{A \times A'}}
\end{mathpar}

\subsubsection*{Computation Types}

\begin{mathpar}
\inferrule
  {\sem{A : \mathit{ValTy}} = \gr{A}}
  {\sem{FA : \mathit{CompTy}} = \gr{FA}}
\and
\inferrule
  {\sem{A : \mathit{ValTy}} = \gr{A} \\ \sem{B : \mathit{CompTy}} = (\gr{B},\gr{f})}
  {\sem{A \to B : \mathit{CompTy}} = (\gr{A \Rightarrow B}, \gr{\Lambda(\mathit{str}_{A,A \Rightarrow B};T(\mathit{ev}_{A,A \Rightarrow B});f)})}
\and
\inferrule
  {\sem{B_i : \mathit{CompTy}} = (\gr{B_i}, \gr{f_i})^{i \in 1..n}}
  {\sem{\Pi_{i \in I} B_i : \mathit{CompTy}} = \gr{\langle \cdots T(\pi_i);f_i~^{i \in 1..n} \cdots \rangle} }
(\text{where } \gr{\pi_i : (\Pi_{i \in I} B_i) \to B_i} \text{ is the obvious projection.})
\end{mathpar}

\subsection*{Typing Judgments}

\begin{mathpar}
\inferrule
  {\sem{\Gamma \vdash^c M : FA} = \gr{f : \sem{\bk{\Gamma}} \to U\sem{\bk{FA}}} \\ \sem{\Gamma,x:A \vdash^c N : \underline{C}} = \gr{g : \sem{\bk{\Gamma}} \times \sem{\bk{A}} \to \sem{\bk{\underline{C}}}}}
  {\sem{\Gamma \vdash M~\mbf{to}~x. N : \underline{C}} = \gr{\sem{\bk{\Gamma}} \overset{\langle \sem{\bk{\Gamma}}, f \rangle}{\longrightarrow} \sem{\Gamma} \times UF\sem{\bk{A}}  \overset{\sem{\bk{\Gamma}} \times s}{\longrightarrow} UF(\sem{\bk{\Gamma}} \times \sem{\bk{A}}) \overset{UFg}{\longrightarrow} UFU \sem{\bk{\underline{C}}} \overset{U \epsilon}{\longrightarrow} U\sem{\bk{\underline{C}}} }  }
\and
\inferrule
  {~}
  {\sem{\Gamma, x:A, \Gamma' \vdash^v x : A} = \gr{\pi_x} : \sem{\Gamma,x:A,\Gamma'} \to \sem{A}}
\and
\inferrule
  {\sem{\Gamma \vdash^v V : A} = \gr{f} : \sem{\Gamma} \to \sem{A}}
  {\sem{\Gamma \vdash^c \mbf{return}~A : FA} = \gr{f}}
\and
\inferrule
  {\sem{\Gamma \vdash^c M : \underline{B}} = \gr{f} : \sem{\Gamma} \to \mathit{Carrier}(\sem{\underline{B}})}
  {\sem{\Gamma \vdash^v \mbf{thunk}~M : U\underline{B}} = \gr{f}}
\and
\inferrule
  {\sem{\Gamma \vdash^V V : U \underline{B}} = \gr{f} : \sem{\Gamma} \to \sem{U\underline{B}}}
  {\sem{\Gamma \vdash^C \mbf{force}~V : \underline{B}} = \gr{f}}
\and
\inferrule
  {\sem{\Gamma \vdash^v V : A_i} = \gr{f} : \sem{\Gamma} \to \sem{A_i}}
  {\sem{\Gamma \vdash^v (\hat{\iota},V) : \Sigma_{i \in I} A_i} = \gr{f;\kappa_i}}
\and
\inferrule
  {\sem{\Gamma \vdash^v V : \Sigma_{i \in I}A_i} = \gr{f} \\ \cdots \sem{\Gamma,x:A_i \vdash^c M_i : \underline{B}} = \gr{g_i} \cdots_{i\in I}}
  {\sem{\Gamma \vdash^c \mbf{pm}~V~\mbf{as}~\{ \ldots, (i,x).M_i,\ldots \} : \underline{B}} = \gr{\langle id_{\sem{\Gamma}}, f \rangle;[\cdots g_i \cdots]_{\sem{\Gamma}}}}
\and
\inferrule
  {\sem{\Gamma \vdash^v V : A} = \gr{f} \\ \sem{\Gamma \vdash^v V' : A'} = \gr{g}}
  {\sem{\Gamma \vdash^v (V,V') : A \times A'} = \gr{\langle f, g \rangle}}
\and
\inferrule
  {\sem{\Gamma \vdash^v V : A \times A'} = \gr{f} \\ \sem{\Gamma,x:A,y:A' \vdash^c M : \underline{B}} = \gr{g}}
  {\sem{\Gamma \vdash^c \mbf{pm}~V~\mbf{as}~(x,y).M : \underline{B}} = \gr{\langle \mathit{id}_{\sem{\Gamma}},f \rangle; g}}
\and
\inferrule
  {\cdots \sem{\Gamma \vdash^c M_i : \underline{B_i}} = \gr{f_i} \cdots _{i \in I}}
  {\sem{\Gamma \vdash^c \lambda \{ \ldots, i. M_i, \ldots \} : \Pi_{i \in I} \underline{B_i}} = \gr{\langle \cdots f_i \cdots \rangle}}
\and
\inferrule
  {\sem{\Gamma \vdash^c M : \Pi_{i \in I} \underline{B_i}} = \gr{f}}
  {\sem{\Gamma \vdash^c \hat{\iota}`M : \underline{B_i}} = \gr{f;\pi_i}}
\and
\inferrule
  {\sem{\Gamma,x:A \vdash^c M : \underline{B}} = \gr{f}}
  {\sem{\Gamma \vdash^c \lambda x. M : A \to B} = \gr{\Lambda(f)}}
\and
\inferrule
  {\sem{\Gamma \vdash^v V : A} = \gr{f} \\ \sem{\Gamma \vdash^c M : A \to \underline{B}} = \gr{g}}
  {\sem{\Gamma \vdash^c V ` M : \underline{B}} = \gr{\langle \mathit{id}, f \rangle;\Lambda^{-1}(g)}}
\end{mathpar}

\subsection*{Stack Typing Judgments}

\begin{mathpar}
\inferrule
  {~}
  {\sem{\Gamma \mid \underline{C} \vdash^k \mathit{nil} : \underline{C}} = \gr{\pi'}}
\and
\inferrule
  {\sem{\Gamma, x:A \vdash^c M : \underline{B}} = \gr{f} \\ \sem{\Gamma \mid \underline{B} \vdash^k K : \underline{C}} = \gr{g}}
  {\sem{\Gamma \mid FA \vdash^k [\cdot]~\mbf{to}~x. M :: K : \underline{C}} = \gr{\langle \pi, f \rangle; g}}
\and
\inferrule
  {\sem{\Gamma \mid \underline{B_{\hat{\iota}}} \vdash^k K : \underline{C}} = \gr{f}}
  {\sem{\Gamma \mid \Pi_{i \in I} \underline{B_i} \vdash^k \hat{\iota} :: K : \underline{C}} = \gr{(\mathit{id} \times \pi_i);f}}
\and
\inferrule
  {\sem{\Gamma \vdash^v V : A} = \gr{f} \\ \sem{\Gamma \mid \underline{B} \vdash^k K : \underline{C}} = \gr{g}}
  {\sem{\Gamma \mid A \to \underline{B} \vdash^k V :: K : \underline{C}} = \gr{\langle \mathit{id}, (f \times \mathit{id});\mathit{ev} \rangle;g}}
\end{mathpar}

We must prove that each of the above stack interpretations are $T$-homomorphisms over $\sem{\Gamma}$.




\end{document}
