\documentclass{article}

\usepackage{amscd}
\usepackage{graphicx}
\usepackage{amsmath}
\usepackage{amssymb}
\usepackage{amsthm}
\usepackage{stmaryrd}
\usepackage{mathpartir}
\usepackage{multicol}
\usepackage{enumitem}
\usepackage[section]{placeins} % float barriers
\usepackage{natbib}
\usepackage{xcolor} 
\usepackage{bussproofs} 
\usepackage{diagrams}
\usepackage{tikz}

\usetikzlibrary{cd}

\newtheorem{lemma}{Lemma}

%example: \limit{j \in J}{F_j}
\newcommand{\limit}[2]{\underset{\overset{\longleftarrow}{#1}}{\text{lim}}~#2}
\newcommand{\lims}[1]{\underset{\longleftarrow}{\text{lim}}~#1}
\newcommand{\mbf}{\mathbf}

\newcommand{\vrt}[2]{
\pile{
#1 \\
\downarrow \\
#2
}
}

\newcommand{\ddisp}[3]{
\left(
\scriptsize
\begin{tikzcd}
#1 \ar[d, "\footnotesize{#2}"] \\
#3
\end{tikzcd}
\normalsize
\right)
}

\newcommand{\disp}[3]{
\left(
\tiny
\begin{array}{c}
#1 \\
\downarrow\\
#3
\end{array}
\begin{array}{l}
~ \\
#2 \\
~
\end{array}
\normalsize
\right)
}

\newcommand{\dispp}[3]{
\tiny
\begin{tikzcd}
#1 \ar[d, "#2"] \\
#3
\end{tikzcd}
\normalsize
}

\title{Solutions: Jacobs \emph{Cateogorical Type Theory} }

\begin{document}

\maketitle

\section*{1.1}

\subsection*{1.1.1}

\subsubsection*{i}
\begin{tikzcd}
X \arrow[dd, dotted, "\varphi"] \arrow[rrd, "f"] & & \\ % \arrow[dd, "\phi"] \arrow[dr, "\psi"] & &   \\
  & & Y \\
X' \ar[rru, "f'"] \ar[uu, dotted] & &  \\
 & & \\
I \ar[rr, "u = pf = pf'"] \ar[loop left, "I = p \varphi"] & & J
\end{tikzcd}
\\~\\~\\
Since $f' : X' \to Y$ is Cartesian over $u$ and $pf = I;pf = I;u$, we have a
unique $\varphi : X \to X'$ such that $p \varphi = I$ and $\varphi;f' = f$.
Likewise, we have a unique $\varphi^{-1} : X' \to X$ such that $p \varphi^{-1} = I$ 
and $\varphi^{-1};f = f'$. Thus $\varphi$ and $\varphi^{-1}$ are vertical isos.

\subsubsection*{ii} 

Not sure I fully understand this. I have the following diagram.

\begin{tikzcd}
Z \ar[r, shift left, "g" above] \ar[r, shift right, "h" below] \ar[rr, bend left = 50, "k=g;f=h;f" above] & X \ar[r, "f" above] & Y \\
   &   &   \\
pZ \ar[r, shift left, "pg" above] \ar[r, shift right, "ph" below] \ar[rr, bend left = 50, "pk=pg;pf=ph;pf"] & I \ar[r, "pf" below] & J
\end{tikzcd}
\\~\\
Since $f$ is Cartesian and $pk$ factors through $pf$ via $pg$, we should have a unique arrow $\varphi : Z \to X$ above
$pg$ with $\varphi;f = k$. Obviously, this arrow is $g$. But this says nothing about $h$; I don't think we know that
$h$ is above $pg$. Perhaps Jacobs meant to say that $w$ as well as $h$ must be unique in his definition. 

BUT WAIT! $w$ must be unique in Jacobs' definition, for otherwise $h$ would not be unique. Okay, problem solved.
The key to reading this definition is that the unique arrow $h$ \emph{happens to be above $w$} it is not unique \emph{among
the arrows above $w$}. 

\subsection*{1.1.2}

$\Leftarrow$. As a shorthand, let's define $g \doteq f \circ (-)$, and write $g^{-1}$ for its inverse.
To show that $f$ is cartesian, we first assume the following situation.

\begin{tikzcd}
Z \ar[rrd, "t"] &               &   \\
                & X \ar[r, "f" below] & Y \\
                &               &   
                &               &   \\
pZ \ar[rrd, "pt"] \ar[rd, "v" below] & & \\
 & I \ar[r, "u" below] & J              
\end{tikzcd}

Here $v;u = pt$. Now, since $t$ is over $v;u$, we can map $t$ through $g^{-1}$ to get an arrow
$s$ over $v$ such that $s;f = t$. To show uniqueness, assume an arrow $s' : Z \to X$ such that
$s';f = t$. Then $s' = g^{-1}(g(s')) = g^{-1}(s';f) = g^{-1}(t) = s$.
\\~\\
$\Rightarrow$. Right-to-left is rather straightforward: The premise of a cartesian morphism $f : X \to Y$ is that
each arrow $t : Z \to Y$ over $v;u$ factors through $f$ via a unique arrow $s : Z \to X$, which happens to be over $v$. 
This defines a function which can be inverted simply by composing an arrow $s$ over $v$ with $f$ to get $s;f = t$ over $v;u$.

\subsection*{1.1.3}

\subsubsection*{i} 

Intuitively, my first instinct is to think of this as a formalization of the fact that any given arrow in the total category
(analogous to a derivation in a refinement system) may have a precondition that is not maximally weak given its postcondition.
We can therefore factor it into a vertical morphism (which serves the purpose of weakening to the point of maximality)
and a cartesian morphism (i.e. a morphism whose precondition is maximally weak given its postcondition). Now to prove this 
formally.
Consider an arbitrary arrow $f$ in $\mathbb E$ and its mapping through $p$.\\~\\
\begin{tikzcd}
X \ar[rr, "f"] & & Y \\
 & & \\
pX \ar[rr, "pf"] & & pY
\end{tikzcd}\\~\\
We apply the definition of fibrations, using $Y$ as the object of $\mathbb E$ and $pf$ as the arrow
of $\mathbb B$ going into $pY$ to get a Cartesian arrow $c : Z \to Y$ in $\mathbb E$ over $pf$:\\
\begin{tikzcd}
X \ar[rr, "f"] & & Y \\
 & Z \ar[ur, "c"] & \\
pX \ar[rr, "pf = pc"] & & pY
\end{tikzcd}\\~\\
Since $c$ is Cartesian and $pf = pX;pc$, we have an arrow $s : X \to Z$ such that $ps = pX$ ($s$ is vertical)
and $s;c = f$.

\subsubsection*{ii}

Let $f : X \to Y$ be a cartesian arrow over an isomorphism $pf$, and consider the identity map $Y$.\\
\begin{tikzcd}
X \ar[r, "f"] & Y \ar[loop right] \\
pX \ar[r, "pf"] & pY \ar[l, "(pf)^{-1}"]
\end{tikzcd}\\~\\
$pY$ factors to $(pf)^{-1};pf$. Hence (by the definition of \emph{Cartesian arrow}) there is some $s : Y \to X$
over $(pf)^{-1}$ such that $s;f = Y$. We also have $f;s;f = f;Y = X;f$. Hence, by 1.1.1 (ii) we have
$f;s = X$, and so $f$ and $s$ are an iso pair. 
\\~\\
It's important to note that we have proved something slightly stronger than the problem statement:
not only is $f$ an iso, but its inverse is over the inverse of the iso in $\mathbb B$, i.e.
$s = f^{-1}$ is over $(pf)^{-1}$.
\subsection*{1.1.4}

\subsubsection*{i}

This is a straightforward consequence of 1.1.2. 

\subsubsection*{ii}

To show that $f;g$ is Cartesian, we begin with the following scenario.

\begin{tikzcd}
S \ar[rrd, "x"] & & \\
X \ar[r, "f" below] & Y \ar[r, "g" below] & Z \\
 & & \\
pS \ar[rrd, "px"] \ar[d, "w" left] & & \\
pX \ar[r, "pf" below] & pY \ar[r , "pg" below] & pZ 
\end{tikzcd}\\~\\

Since $g$ is Cartesian and $px$ factors through $pg$ (via $w;pf$), 
We have a unique arrow $s : S \to Y$ such that $s;g = x$, which happens to be above $w;pf$.  

\begin{tikzcd}
S \ar[rrd, "x"] \ar[dr, dotted, "s" below] & & \\
X \ar[r, "f" below] & Y \ar[r, "g" below] & Z \\
 & & \\
pS \ar[rrd, "px"] \ar[d, "w" left] \ar[dr , "ps" below] & & \\
pX \ar[r, "pf" below] & pY \ar[r , "pg" below] & pZ 
\end{tikzcd}\\~\\

Since $f$ is Cartesian and $ps$ factors through $ps$ (via $w$), we have a unique arrow
$s' : S \to X$ such that $s';f = s$, which happens to be above $w$.

\begin{tikzcd}
S \ar[rrd, "x"] \ar[dr, dotted, "s" below] \ar[d, dotted, "s'" left] & & \\
X \ar[r, "f" below] & Y \ar[r, "g" below] & Z \\
 & & \\
pS \ar[rrd, "px"] \ar[d, "w" left] \ar[dr , "ps" below] & & \\
pX \ar[r, "pf" below] & pY \ar[r , "pg" below] & pZ 
\end{tikzcd}\\~\\

Substituting $s';f$ for $s$ in $s;g = x$ gives $s';f;g = x$. $s'$ is the unique arrow with 
this property, for suppose that $t';f;g = x$. Then $t';f = s$ by the uniqueness of $s$.
Then $t' = s'$ by the uniqueness of $s'$.  

\subsubsection*{iii}

To show that $f$ is Cartesian, we begin with the following scenario

\begin{tikzcd}
Q \ar[dr , "x"] &   &   \\
X \ar[r , "f" below] & Y \ar[r , "g" below] & Z \\
  &   &   \\
pQ \ar[dr, "px"] \ar[d , "w" left] & & \\
pX \ar[r , "pf" below] & pY \ar[r , "pg" below] & pZ
\end{tikzcd}

By composition, we get:

\begin{tikzcd}
Q \ar[dr , "x" left] \ar[drr , "x;g" right] &   &   \\
X \ar[r , "f" below] & Y \ar[r , "g" below] & Z \\
  &   &   \\
pQ \ar[dr, "px" left] \ar[d , "w" left] \ar[drr , "p(x;g)"] & & \\
pX \ar[r , "pf" below] & pY \ar[r , "pg" below] & pZ
\end{tikzcd}\\~\\

$p(x;g)$ factors through $pg$ via $w;pf$. Hence, there exists a unique arrow ($x$) 
such that $x;g = (x;g)$, which happens to be above $px$: i.e. $$(*)~\forall y .~ y;g=x;g \Rightarrow x=y$$

Furthermore, $p(x;g)$ factors through $pf;pg$ via $w$. Hence, there is a unique arrow (called $s$, pictured below)
such that $s;f;g = (x;g)$, where $s$ happens to be above $w$; for this reason, we rename $w$ to $ps$:

\begin{tikzcd}
Q \ar[dr , dotted, "x" left] \ar[d, dotted, "s"] \ar[drr , "x;g" right] &   &   \\
X \ar[r , "f" below] & Y \ar[r , "g" below] & Z \\
  &   &   \\
pQ \ar[dr, "px" left] \ar[d , "ps" left] \ar[drr , "p(x;g)"] & & \\
pX \ar[r , "pf" below] & pY \ar[r , "pg" below] & pZ
\end{tikzcd}\\~\\

Instantiating $y$ in $(*)$ with $s;f$, we have $s;f = x$. Suppose $t : Q \to X$ has the property
that $t;f = x$. Then $t;f;g = x;g$. Hence, by the uniqueness of $s$, $t=s$. 

\subsection*{iv}

I guess I don't understand this, because it seems like a simple fact that has nothing to do with $iii$.

\section*{1.1.5}

These are both straightforward consequences of the fact that pullback squares are cartesian morphisms
with respect to the cod fibration.

\section*{1.1.6}

I tried $\Rightarrow$ and had a bit of trouble. Maybe next time try $\Leftarrow$ first?

\section*{1.1.7}

Example 1: $\dispp{\mathbb B \times \mathbb C}{fst}{\mathbb C}$ \\~\\

Suppose we have some object $B_2 \times C$ of $\mathbb B \times \mathbb C$, 
and some arrow $B_1 \rTo^{f} B_2$ of $\mathbb B$. Then we have an arrow
$f \times C$ above $f$; we'll show this arrow is Cartesian.\\~\\
\begin{tikzcd}
B_1 \times C \ar[r , "f \times C"] & B_2 \times C \\
B_1 \ar[r, "f"] & B_2
\end{tikzcd}

Consider some $B_3 \times C_3 \rTo^{g \times h} B_2 \times C$ such that $fst (g \times h) = g$ factors through
$fst(f \times C) = f$, i.e. there is some $B_3 \rTo^{d} B_1$ with $d;f = g$. Then $g \times h$ factors through
$f \times C$ in $\mathbb B \times \mathbb C$ through a unique arrow: that arrow is $d \times h$, which is over
$d$.

\begin{tikzcd}
B_3 \times C_3 \ar[dr , "g \times h"] \ar[d , "d \times h" left] & ~ \\
B_1 \times C \ar[r , "f \times C" below] & B_2 \times C \\
B_3 \ar[dr, "g"] \ar[d, "d" left] & ~ \\
B_1 \ar[r, "f" below] & B_2
\end{tikzcd}\\~\\

Example 2: $\dispp{\mathbb B}{id}{\mathbb B}$\\~\\
Straightforward.\\~\\
Example 3: $\dispp{\mathbb B}{id}{1=\{ \ast \}}$\\~\\
Straightforward.
Example 4: Yup, they're trivial.
 
\section*{1.2}

\subsection*{1.2.1}

First we assume that $(u, (f_i)_{i \in I})$ is Cartesian
and show that every $f_i$ is an iso.\\~\\
$(\Rightarrow)$.\\~\\
\begin{tikzcd}
(K, (B_{k})_{k \in K}) \ar[dr, "(v \text{,}~(h_k) : B_k \to D_{v(k)})" right] \ar[d, dotted, "(w \text{,}~(g_k) : B_k \to C_{w(k)})" left] & ~ \\
(I, (C_i)_{i \in I}) \ar[r, "\pile{ ~ \\ ~ \\ (u \text{,}~(f_i) : C_i \to D_{u(i)}) }" below] & (J, (D_j)_{j \in J})
 & \\
K \ar[d, "w" left] \ar[dr, "v"] & \\
I \ar[r, "u" below] & J 
\end{tikzcd}\\~\\
An arrow in $Fam(\mathbb C)$ above $w$ is a family of arrows $(g_k)_{k \in K}$ of $\mathbb C_1$ such that 
$$B_k \rTo^{g_k} C_{w(k)} \rTo^{f_{w(k)}} D_{u(w(k))} = B_k \rTo^{h_k} D_{v(k)}$$
We can construct such a family by defining $g_k \triangleq h_k;f_{w(k)}^{-1}$.
In fact, for some other such family $(g'_k)$ we have for $k \in K$, $g'_k;f_{w(k)} = h_k = g_k;f_{w(k)}$. Since isos are
split monos, we have $g'_k = g_k$, and so $(w, g_k)$ is the unique family above $w$ making the top triangle commute.\\~\\
($\Leftarrow$)  On the other hand, assume $(u, (f_i) : C_i \to D_{u(i)})$ above is Cartesian. Dude. I absolutely cannot figure this out. Oh well.

\subsection*{1.2.2}

$(\Rightarrow)$

Let $f$ be monic.
Consider $f_i : X_i \to Y_i$, and $a,b : Z \to X_i$ with $a;f_i = b;f_i$. We define the 
family $(Z_k)_{k \in I}$ as $Z_i \doteq Z$ and for $j \neq i$, $Z_j \doteq X_j$.
Also define the family $(h_k)_{k \in I}$ as $h_i \doteq a$ and for $j \neq i$, $h_j \doteq 1_{X_j}$.
Likewise, define $(g_k)_{k \in I}$ as $g_i \doteq b$ and for $j \neq i$, $h_j \doteq 1_{X_j}$.

Then $((h_k)_{k \in I};f)_i = a;f_i$ and for $j \neq i \in I$, $((h_k)_{k \in I};f)_j = 1_{X_j};f_j = f_j$.
Likewise, $((g_k)_{k \in I};f)_i = b;f_i$ and for $j \neq i \in I$, $((g_k)_{k \in I};f)_j = 1_{X_j};f_j = f_j$.
Since $a;f_i = b;f_i$ by assumption, we have $(h_k)_{k \in I};f = (g_k)_{k \in I};f$. Since $f$ is monic,
$(h_k)_{k \in I} = (g_k)_{k \in I}$. But this implies that $a = h_i = g_i = b$.
~\\~\\
$(\Leftarrow)$
Suppose that each $f_i$ for $i \in I$ is monic, and consider $g,h : (Z_i \rightrightarrows X_i)_{i \in I}$.
Suppose $f;g = f;h$. Then for $i \in I$, $(f_i;g_i) = (f;g)_i = (f;h)_i = (f_i;h_i)$. Since $f_i$ is monic,
for each $i \in I$ we have $g_i = h_i$. Hence, $g = h$.

\subsection*{1.2.3}

\subsubsection*{(i)}

We start with an arrow $u : I \to J$ in $\mathbb B$ and an object $(\vrt{X}{J}\varphi, s)$ above $J$.

\begin{mathpar}
\begin{tikzcd}
              & X \ar[d, bend right = 20, "\varphi" left] \\
I \ar[r,"u" below]  & J \ar[u, bend right = 20, "s" right]  
\end{tikzcd}
\and
s;\varphi = 1_J
\end{mathpar}

To construct a Cartesian arrow over $u$, we first take the pullback of $u$ and $\varphi$.

\begin{mathpar}
\begin{tikzcd}
I \times_J X \ar[r, "\pi_1"] \ar[d, "\pi_2" left] \ar[dr, phantom, "\lrcorner", very near start] & X \ar[d, bend right = 20, "\varphi" left] \\
I \ar[r, "u" below]  & J \ar[u, bend right = 20, "s" right]  
\end{tikzcd}
\and
\pi_1;\varphi = \pi_2;u
\end{mathpar}

$\pi_2$ is the display of our Cartesian arrow's domain. We form its section using the pullback universal property.

\begin{mathpar}
\begin{tikzcd}
 I \ar[ddr, bend right = 10, "1_I~~" below] \ar[drr, bend left = 10, "u;s"] \ar[dr, dotted, "t"] & & \\
 & I \times_J X \ar[r, "\pi_1"] \ar[d, "\pi_2" left] \ar[dr, phantom, "\lrcorner", very near start] & X \ar[d, bend right = 20, "\varphi" left] \\
 & I \ar[r, "u" below]  & J \ar[u, bend right = 20, "s" right]  
\end{tikzcd}
\and
t;\pi_1 = u;s 
\and
t;\pi_2 = 1_I
\end{mathpar}

First, note that $(\vrt{I \times_J X}{I}\pi_2,t)$ is an object of $\mathbb B_{\bullet}^{\to}$, since $t;\pi_2 = 1_I$.
Also, $(\pi_1, u) : (\vrt{I \times_J X}{I}\pi_2,t) \to (\vrt{X}{J}\varphi,s)$ is an arrow,
since $\pi_2;u = \pi_1;\varphi$ and $t;\pi_1 = u;s$.

Now, we must prove that $(\pi_1,u)$ is Cartesian over $u$.
Assume some object $(\vrt{Z}{K}\psi,r)$ and an arrow $(f,v;u) : (\vrt{Z}{K}\psi,r) \to (\vrt{X}{J}\varphi,s)$.
Since $(f,v;u)$ is an arrow, it preserves points $v;u;s = r;f$.

Any arrow $(h;w)$ factoring $(f,v;u)$ through $(\pi_1,u)$ must have $h;\pi_1 = f$
and $h;\pi_2 = \psi;w$.   
Since $f : Z \to X$ and $\psi;v : Z \to I$, by the pullback property there exists a unique arrow
$z : Z \to I \times_J X$ with $z;\pi_1 = f$ and $z;\pi_2 = \psi;v$.
Thus $h = z$.

The arrow $(z,v) : (\vrt{Z}{K}\psi,r) \to (\vrt{I \times_J X}{I}\pi_2,t)$ is then the unique arrow
over $v$ factoring through $(\pi_1,u)$. We need only prove it is an arrow this point, since we've established 
its uniqueness. It commutes, since $z;\pi_2 = \phi;v$. It also preserves points, since
$v;t;\pi_1 = v;u;s = r;f = r;z;\pi_1$ and $\pi_1$ is monic.

\subsubsection*{(ii)}

First, we define an equivalence of categories 
$F : Fam(\mbf{Sets_{\bullet}}) \overset{\simeq}{\longleftrightarrow} \mbf{Sets_{\bullet}^{\to}} : U$.
$$F(I, (X_i,a_i)_{i \in I}) \doteq (\vrt{\coprod_{i \in I} X_i}{I}\pi, i \mapsto (i, a_i))$$


\begin{mathpar}
F( (u : I \to J, (f_i)_{i \in I} : X_i \to Y_{u(i)}) : (I, (X_i,a_i)_{i \in I}) \to (J, (Y_j,b_j)_{j \in J}) \doteq
\and
\begin{tikzcd}
\coprod_{i \in I} X_i \ar[rrr, "{(i , x_i) \mapsto (u(i), f_i(x_i))}"] \ar[d, "\pi_I" left, bend right = 15] & ~ & ~ & \coprod_{j \in J} Y_j \ar[d, bend left = 15, "\pi_J"] \\
I \ar[rrr, "u" below] \ar[u, "{i \mapsto (i,a_i)}" right, bend right = 15] & ~ & ~ & J \ar[u, bend left = 15, "{j \mapsto (j,b_j)}"] 
\end{tikzcd} 
\end{mathpar}

$$U(\vrt{X}{I}\varphi,s) \doteq (I, (\varphi^{-1}(i), s(i))_{i \in I})$$

$$U( (u : I \to J, f : X \to Y) : (\vrt{X}{I} \varphi, s) \to (\vrt{Y}{J}\psi,t) ) \doteq (u, (f \mid_{\varphi^{-1}(i)})_{i \in I})$$

An equivalence of categories consists of two natural isos, first $\eta : 1_{Fam(\mbf{Sets_{\bullet}})} \overset{\sim}{\longrightarrow} UF$

We need an arrow from $(I, (X_i, a_i)_{i \in I})$ to $(I, (\{i\} \times X_i, (i,a_i))_{i \in I})$, and choose the obvious one:
$$\eta_{(I, (X_i, a_i)_{i \in I})} \doteq (1_I, (x \mapsto (i,x))_{i \in I})$$
$$\eta^{-1}_{(I, (X_i, a_i)_{i \in I})} \doteq (1_I, ((i,x) \mapsto x)_{i \in I})$$

They are clearly an iso pair, and are both point-preserving. We show that $\eta$ is natural.
Let $(u,(f_i)_{i \in I}) : (I, (X_i,a_i)_{i \in I}) \to (J, (Y_j, b_j)_{j \in J}) \in Fam(\mbf{Sets_\bullet})_1$. 

Then we have 
$$F(u,(f_i)_{i \in I}) = (u, g)  : (\vrt{\coprod_{i \in I} X_i}{I}\pi_I, i \mapsto (i, a_i)) \to 
  (\vrt{\coprod_{j \in J} Y_j}{J}\pi_J, j \mapsto (j, b_j))$$ % (J, (Y_j,b_j)_{j \in J})$$
where $g(i,x_i) \doteq (u(i),f_i(x_i))$,
and
$UF(u,(f_i)_{i \in I}) = (u, (g \mid_{\{i\} \times X_i})_{i \in I})$. %= (u, ((i,x_i) \mapsto (u(i),f_i(x_i)))_{i \in I})$$
For naturality of $\eta$:
$$(u,(f_i)_{i \in I});\eta_{(J, (Y_j, b_j)_{j \in J})} = (u, (x \mapsto (u(i), f_i(x)))_{i \in I}) = 
  \eta_{(I, (X_i,a_i)_{i \in I})};(u,g)$$ 

Now we need another natural iso $\epsilon : 1_{\mbf{Sets_{\bullet}^{\to}}} \overset{\sim}{\longrightarrow} FU$.
Let $A \doteq (\vrt{X}{I}\varphi,s)$ be an object of $\mbf{Sets_{\bullet}^{\to}}$.
Then $FU(\vrt{X}{I}\varphi,s) = (\vrt{\coprod_{i \in I} \varphi^{-1}(i)}{I}\pi, \langle 1_I, s \rangle)$.
Let $A \doteq $ We define
$$\epsilon_{A} \doteq (1_I, \langle \varphi, 1_X\rangle) $$
$$\epsilon^{-1}_A \doteq (1_I, \pi') $$
Clearly this is an iso pair that preserves points. We must show that $\epsilon$ is natural.
Let $B \doteq (\vrt{Y}{J}\psi,t)$ and let $(u,f) : (\vrt{X}{I}\varphi,s) \to (\vrt{Y}{J}\psi,t)$ be an arrow in 
$\mbf{Sets^{\to}_{\bullet}}$.
Then $U(u,f) = (u,(f\mid_{\varphi^{-1}(i)})_{i \in I})$ and $FU(u,f) = (u, u \times f)$.\\~\\
$(u,f);\epsilon_B = (u,f);(1_J, \langle \psi, 1_Y \rangle) = (u, \langle f;\psi, f \rangle)
= (u, \langle \varphi;u, f \rangle) = \epsilon_A;(u, u \times f)$
\\~\\
*****TODO********
We're not quite done yet. We must show that a triangle like the one on pg 10012 commutes.

\subsection*{1.2.4}

Suppose $aRb$. Then by symmetry, $bRa$. By transitivity, $aRa$ and $bRb$.

\subsection*{1.2.5}

\subsubsection*{(i)}

Call the set of partitions $Part(I)$ and the set of equivalence relations $ER(I)$.
We define a bijection $\phi : Part(I) \to ER(I)$.
$$\phi(Q) \doteq \{ (i,j) \mid \exists a \in Q. i \in a, j \in a \}$$
If $Q$ is a partition then $\phi(Q)$ is an equivalence relation.
\begin{itemize}
\item It's symmetric: if $(i,j) \in \phi(Q)$ then $\exists a \in q. i \in Q, j \in Q$. Hence $(j,i) \in Q$.
\item It's transitive: if $(i,j), (j,k) \in \phi(Q)$ then $\exists a~b \in Q. i \in a, j \in a, j \in b, k \in b$.
 Since $j \in a$ and $j \in b$ we have $a = b$, and so $k \in b = a$; hence, $(i,k) \in \phi(Q)$.
\item It's reflexive: letting $i \in I$, since $\bigcup Q = I$ there exists some $a \in Q$ with $i \in A$.
 Then $(i,i) \in \phi(Q)$.
\end{itemize}

Note that only reflexivity relies on the totality of the partition.
We also define an inverse:
$$[n]_R \doteq \{ m \mid nRm \}$$
$$\phi^{-1}(R) = \{ [n] \mid nRn \}$$

If $R$ is a PER then $\phi^{-1}(R)$ is a partition.
\begin{itemize}
\item Let $a \in \phi^{-1}(R)$. Then $a = [n]$ for some $n$ with $nRn$. Therefore $n \in a$, and so $a$ is non-empty.
\item Let $a,b \in \phi^{-1}(R)$. Suppose $a \cap b \neq \emptyset$. Then there exists an $i \in I$ with $i \in a$ and
 $i \in b$. If $j \in I$ then $j \in a \Leftrightarrow jRi \Leftrightarrow j \in b$. Hence $a = b$.
\item Let $i \in I$. Then $iRi$. Hence $[i] \in \phi^{-1}(R)$, and so $i \in \bigcup \phi^{-1}(R)$.
\end{itemize}

The last of the above points relies on reflexivity. 

To cement the deal, we must prove that $\phi$ and $\phi^{-1}$ are inverses.
Let $R$ be a $PER$. We prove that $\phi(\phi^{-1}(R)) = R$. To this end,\\~\\
$(i,j) \in \phi(\phi^{-1}(R)) \Leftrightarrow$\\
$\exists a \in \phi^{-1}(R).~i \in a \wedge j \in a \Leftrightarrow$\\
$\exists n \in I.~iRn \wedge jRn \Leftrightarrow$\\
$iRj \Leftrightarrow$\\
$(i,j) \in R$\\~\\

Finally, let $Q \in Part(I)$. We prove $\phi^{-1}(\phi(Q)) = Q$.\\~\\
$a \in \phi^{-1}(\phi(Q)) \Leftrightarrow$
$\exists n \in I.~a = [n] \wedge (n,n) \in \phi(Q) \Leftrightarrow$\\
$\exists n \in I. a = [n] \wedge \exists b \in Q.~n \in b \Leftrightarrow$\\
$a \in Q$


%let $(i,j) \in \phi(\phi^{-1}(R))$. Then $(i,j) = \phi([n])$ for some $n$ with $nRn$.
%Then $$

\subsubsection*{(ii)}

We write $EqSub(I)$ for the set of equivalence relations on subsets of $I$; an element of $EqSub(I)$ has the form
$(X, R)$ where $X \subseteq I$ and $R$ is an equivalence relation on $X$. 
We show a bijective correspondence $\psi : PPart(I) \to EqSub(I)$ between the partial partitions on $I$ and 
equivalence relations on subsets on $I$: the remainder follows transitively from $(i)$.

$$\psi(Q) \doteq (\bigcup Q, R)$$
$$iRj \overset{def}{\Leftrightarrow} (\exists a \in Q.~i \in a \wedge j \in a)$$

Above, clearly $\bigcup Q \subseteq I$ and $R$ is an equivalence relation.
We prove it's a bijection by providing an inverse:

$$\psi^{-1}(X,R) \doteq \{ [i] \mid iRi \}$$
 
First, note that every set in $\psi^{-1}(X,R)$ is non-empty, since $i \in [i]$.
Second, assume some $a,b \in \psi^{-1}(X,R)$. Then $a = [i]$ and $b = [j]$ for some $i,j \in I$ with $iRi$ and $jRj$.
Assume some $k \in I$ with $k \in a = [i]$ and $k \in b = [j]$. Then $kRi$ and $kRj$. Transitively, $iRj$. 
Thus, if $l \in I$ then\\~\\
$l \in [a] \Leftrightarrow$\\
$lRi \Leftrightarrow$\\
$lRj \Leftrightarrow$\\
$l \in b$\\~\\
Hence $a = b$.

\subsection*{1.2.6}

Umm.... homsets in $\omega$-$\mbf{Sets}$ are not all countable.
What if we give a $\omega$-set an uncountable carrier set whose existence predicate maps each 
element to $\{ 0 \}$. There are uncountably many constant endofunctions on this omega set!!!
 

\section*{1.3}

\subsection*{1.3.1}

\subsubsection*{(i)}

Show that in the total category $s(\mathbb B)$ of a simple fibration $\vrt{s(\mathbb B)}{\mathbb B}$ a morphism
$(u,f) : (I,X) \to (J,Y)$ is Cartesian if and only if there is an isomorphism $h : I \times X \overset{\cong}{\to} I \times Y$
in $\mathbb B$ such that $\pi \circ h = \pi$ and $\pi' \circ h = f$.
\\~\\
We utilize the following lemma
\begin{lemma}
\label{lemma:one}
The arrow $(u, \pi') : (I,Y) \to (J,Y)$ is Cartesian over $u$.
\end{lemma}

\begin{proof}
Let $v : K \to I$ be an arrow in $\mathbb B$ and let $(v;u,f) : (K,Z) \to (J,Y)$ be an arrow in $s(\mathbb B)$.
Clearly, $(v,f) : (K,Z) \to (I,Y)$ is over $v$ and $(v,f);(u,\pi') = (v;u,f)$.\\~\\
\begin{center}
\begin{tikzcd}
(K,Z) \ar[dr, "(v \text{,} f)~~~~~" below] \ar[drr, "(v;u \text{,} f)"] & & \\
 & (I,Y) \ar[r,"(u \text{,} \pi')" below] & (J,Y) \\
 & & \\
K \ar[dr, "v~~~" below] \ar[drr, "v;u"] & & \\
 & I \ar[r,"u" below] & J 
\end{tikzcd}
\end{center}

Furthermore, $(v,f)$ is the unique arrow above $v$ with this property. For let $(v,h) : (K,Z) \to (I,Y)$ be another such
arrow; i.e., $(v,h);(u,\pi') = (v;u,f)$. Then by the definition of composition in $s(\mathbb B)$, $h = f$. 

\end{proof}
~\\
$(\Rightarrow)$ Let $(u,f) : (I,X) \to (J,Y)$ be Cartesian. 
Since $(u,\pi')$ is Cartesian by the above lemma, there exists a unique $a : (I,X) \to (I,Y)$ over $1_I$ 
such that $a;(u,\pi') = (u,f)$. The arrow $(1_I,f)$ satisfies this property, and so $a = (1_I, f)$. 
%Note that $h;\pi' = f$ and $h;\pi = \pi$.

On the other hand, since $(u,f)$ is Cartesian by assumption, 
there exists a unique arrow $b : (I,Y) \to (I,X)$ over $1_I$ such that
$b;(u,f) = (u,\pi')$. Thus, $1_{(I,X)};(u,f) = (u,f) = a;(u,\pi') = (a;b);(u,f)$. 
Since $b : (I,Y) \to (I,X)$ is over $1_I$, $b = (1_I, g)$ for some $g : I \times Y \to X$.

Since $(u,f)$ is Cartesian,
there can only exist one arrow $x : (I,X) \to (I,X)$ over $1_I$ such that $x;(u,f) = (u,f)$; hence, 
$(1_I, \pi') = 1_{(I,X)} = a;b = (1_I, \langle \pi_{I,X};1_I , f \rangle; g) = (1_I, \langle \pi_{I,X} , f \rangle; g)$,
and so $\pi' = \langle \pi_{I,X} , f \rangle; g$.

We define $h : I \times X \to I \times Y \in \mathbb B_1$ as $h \doteq \langle \pi_{I,X}, f \rangle$. Then we have $h;\pi_{I,Y} = \pi_{I,X}$ and
$h;\pi' = f$. Furthermore, $h$ has an inverse $h^{-1} = \langle \pi_{I,X}, g \rangle$.
$$ h;h^{-1} = \langle \pi_{I,X}, f \rangle; \langle \pi_{I,Y}, g \rangle = \langle \langle \pi_{I,X}, f \rangle; \pi_{I,Y} 
  , \langle \pi_{I,X}, f \rangle; g \rangle = \langle \pi_{I,X} , \pi'_{I,X} \rangle = 1_{I \times X}$$
 
$(\Leftarrow)$ Suppose there exists an isomorphism $h : I \times X \overset{\cong}{\to} I \times Y$
in $\mathbb B$ such that $h;\pi = \pi$ and $h;\pi' = f$. The following diagram commutes
\begin{center}
\begin{tikzcd}
(I, Y) \ar[d, "(1_I \text{,} h^{-1};\pi')" left, bend right = 20 ] \ar[rr, "(u \text{,} \pi')" above] & & (J,Y) \\
(I, X) \ar[u,"(1_I \text{,} h;\pi')" right, bend right = 20] \ar[rru, "(u \text{,} f)" below, bend right = 10] &
\end{tikzcd}
\end{center}

Since $(1_I, h;\pi')$ is an iso in $s(\mathbb B)$, it is Cartesian by exercise $1.1.4~(i)$. By the lemma \ref{lemma:one},
$(u,\pi')$ is Cartesian. The composition of two Cartesian arrows is Cartesian by exercise
$1.1.4~(ii)$, and so $(1_I, h;\pi');(u,\pi') = (u,f)$ is Cartesian. 

We now show that the diagram commutes.
$$(1_I, h;\pi');(u,\pi') = (u, \langle \pi;1_I,h;\pi' \rangle;\pi' = (u, \langle \pi;1_I, f \rangle ; \pi' = (u,f) $$

and that $(1_I, h;\pi')$ is an iso
\begin{center}
\begin{tabular}{ll}
$(1_I, h;\pi');(1_I,h^{-1};\pi')$ & $=$ \\ 
$(1_I \langle \pi;1_I, h;\pi' \rangle;(h^{-1};\pi'))$ & $=$ \\
$(1_I, \langle h;\pi, h;\pi' \rangle;h^{-1};\pi)$ & $=$ \\
$(1_I, h;\langle \pi, \pi' \rangle;h^{-1};\pi)$ & $=$ \\
$(1_I, h;h^{-1};\pi)$ & $=$ \\
$(1_I, \pi)$ & $=$ \\
$1_{(I,X)}$ &   
\end{tabular}
\end{center}
 
Before proving the other direction, note that $h^{-1};\pi = h^{-1};(h;\pi) = \pi$.

\begin{center}
\begin{tabular}{ll}
$(1_I, h^{-1};\pi');(1_I,h;\pi')$ & $=$ \\ 
$(1_I \langle \pi;1_I, h^{-1};\pi' \rangle;(h;\pi'))$ & $=$ \\
$(1_I, \langle h^{-1};\pi, h^{-1};\pi' \rangle;h;\pi)$ & $=$ \\
$(1_I, h^{-1};\langle \pi, \pi' \rangle;h;\pi)$ & $=$ \\
$(1_I, h^{-1};h;\pi)$ & $=$ \\
$(1_I, \pi)$ & $=$ \\
$1_{(I,Y)}$ &   
\end{tabular}
\end{center}

\subsubsection*{(ii)}

Show that the assignment $(I, X) \mapsto I*(X) = \ddisp{I \times X}{\pi}{I}$ extends to a full and faithful functor
$s(\mathbb B) \to \mathbb B^{\to}$. Prove that it maps Cartesian morphisms to pullback squares.\\~\\
For an arrow $(u : I \to J, f : I \times X \to Y)$ we let $(u,f) \mapsto (u,\langle \pi;u, f \rangle)$.

To see that this functor is full, consider some commuting pair $(a : I \to J, b : I \times X \to J \times Y)$
between $I*(X)$ and $J*(Y)$. Then $b;\pi = \pi;a$, and we have $(a, b;\pi') \mapsto (a, \langle \pi;a, b;\pi' \rangle)
 = (a, \langle b;\pi, b;\pi' \rangle) = (a, b; \langle \pi, \pi' \rangle) = (a,b)$.
 
To see that this functor is faithful, consider distinct arrows $(u : I \to J, f : I \times X \to Y)$ and 
$(v : I \to J, g : I \times X \to Y)$. Repsectively, they map to $(u, \langle \pi;u, f \rangle)$ and 
$(v, \langle \pi;v, g \rangle)$. Since $(u, f)$ and $(v, g)$ are distinct, either $u \neq v$ or $f \neq g$
\begin{description}
\item[Case $u \neq v$:]~\\
Since $u \neq v$, $(u, \langle \pi;u, f \rangle) \neq (v, \langle \pi;v, g \rangle)$.
\item[Case $f \neq g$]~\\
For contradiction, suppose $\langle \pi;u , f \rangle = \langle \pi;v, g \rangle$. By the 
UMP of products, we have $f = \langle \pi;u , f \rangle;\pi' = \langle \pi;v, g \rangle;\pi' = g$. $\Rightarrow \Leftarrow$.~\\~\\
Hence, $\langle \pi;u , f \rangle \neq \langle \pi;v, g \rangle$.


\end{description}
 

\subsection*{1.3.2}
\newcommand{\dd}{/\!/}
 
\subsubsection*{i}

Consider the fibre $\mathbb B \dd I$. 

First we show it has binary products. We make the obvious choice that, given two arbitrary objects of $\mathbb B / \! / I$
$(I,A)$ and $(I,B)$, their product in $B \dd I$ is $(I, A \times B)$. It projections are $(1_I, \pi';\pi)$ and 
$(1_I, \pi';\pi')$.

We prove that this actually is a product. To this end,
let $(I,X) \in B \dd I$ and let $(1_I, f) : (I,X) \to (I,A)$ and $(1_I, g) : (I,X) \to (I,B)$ be arrows in $B \dd I$.
Since $f : I \times X \to A$ and $g : I \times X \to B$ in $\mathbb B$ we have $\langle f, g \rangle : I \times X \to A \times B$ in $\mathbb B$. Thus, $(1_I, \langle f,g \rangle) : (I,X) \to (I, A \times B)$ in $\mathbb B \dd I$.
Furthermore, we have $(1_I, \langle f,g \rangle);(1_I,\pi';\pi) = (1_I, \langle \pi;1_I, \langle f,g \rangle \rangle;\pi';\pi)
 = (1_I, f)$ and
$(1_I, \langle f,g \rangle);(1_I,\pi';\pi') = (1_I, \langle \pi;1_I, \langle f,g \rangle \rangle;\pi';\pi')
 = (1_I, g)$.

We show $(1_I, \langle f,g \rangle)$ is the unique arrow in $\mathbb B \dd I$ with this property.
Let $(1_I,z) : (I,X) \to (I, A \times B) \in (\mathbb B \dd I)_1$ with $(1_I, z);(1_I,\pi';\pi) = (1_I, f)$
and $(1_I, z);(1_I,\pi';\pi) = (1_I, f)$.
Then $\langle \pi;1_I, z \rangle;\pi';\pi = f$, i.e. $z;\pi = f$ and $z;\pi' = g$. By the $UMP$ of $\langle f,g \rangle$ 
in $\mathbb B$ we have $z = \langle f, g \rangle$. Thus, $(1_I,z) = (1_I, \langle f,g \rangle)$.

Now we show it has a nullary product, i.e. a terminal object. For the terminal object, we consider the obvious choice of 
$(I,1)$ where $1$ is the terminal object of $\mathbb B$. Let $(I,X)$ be an object of $\mathbb B \dd I$. Then
we have an arrow $(1_I, \pi';\bigcirc_{X}) : (I,X) \to (I,1)$ where $\bigcirc_X : X \to 1$ is $X$'s terminal
arrow in $\mathbb B$. This arrow is unique, for any other arrow of matching signature in $\mathbb B \dd I$
must have the form $(1_I,z)$ where $z : I \times X \to 1$, and by the UMP of terminal objects in $\mathbb B$, we have
$z = \pi';\bigcirc_{X}$.

By the nature of products in $\mathbb B \dd I$, as we've described above, 
it's clear that $I^* : \mathbb B \to \mathbb B \dd I$ preserves products.

Finally, note that since $\langle (1_I, f) , (1_I, g) \rangle = (1_I, \langle f,g \rangle)$,
we have \\
$(1_I, f) \times (1_I, g) = \langle (1_I,\pi';\pi);(1_I, f), (1_I,\pi';\pi');(1_I,g) \rangle$\\ 
$= \langle (1_I,\langle \pi;1_I, \pi';\pi \rangle;f), (1_I, \langle \pi;1_I, \pi';\pi' \rangle;g) \rangle$\\
$= \langle (1_I,\langle \pi, \pi';\pi \rangle;f), (1_I, \langle \pi, \pi';\pi' \rangle;g) \rangle$\\
$= (1_I, \langle \langle \pi, \pi';\pi \rangle;f, \langle \pi, \pi';\pi' \rangle;g \rangle)$\\
Damn, that's the most I could simplify it.

\subsubsection*{ii}

$(a \Rightarrow b)$ -- Note that going from $b$ to $a$ might be alot easier; that may be the intended direction.

Suppose that $\mathbb B$ is Cartesian closed. By $(i)$ we know that $\mathbb B \dd I$ has finite products.
We must show $\mathbb B \dd I$ has exponentials. For the exponential $(I,A)^{(I,B)}$ we choose $(I,A^B)$.

Suppose, then, some arrow $(I, A \times B) \overset{(1_I,f)}{\longrightarrow} (I,C)$.
We make the obvious guess of $\overline{(1_I,f)}=(1_I, \overline{f})$ and
$ev_{(I,A),(I,C)} = (1_I, ev_{A,C})$. **** These guesses are nonsense, because $\overline{f}$ is does
not have signature $I \times (A \times B) \to C^A$.
We need to map $f$ through several isomorphisms to get the right signature:\\~\\
$I \times (A \times B) \to C$\\
$(I \times A) \times B \to C$\\
$(A \times I) \times B \to C$\\
$A \times (I \times B) \to C$\\
$I \times B \to C^A$\\~\\

%\\~\\
%\begin{tabular}{ll}
%$(1_{(I,A)} \times \overline{(1_I,f)});ev_{(I,A),(I,C)}$ & \\
%$((1_I,\pi'_{I,A}) \times (1_I,\overline{f}));(1_I,ev_{A,C})$ & $=$ \\
%$(1_I, \langle \langle \pi, \pi';\pi \rangle;\pi'_{I,A}, \langle \pi, \pi';\pi' \rangle;\overline{f} \rangle);(1_I, ev_{A,C})$ %& 
% $=$ \\
%$(1_I, \langle \pi'_{I,A \times B};\pi_{A,B}, \langle \pi_{I,A \times B}, \pi'_{I,A \times B};\pi'_{A,B}\rangle;\overline{f} \r%angle);
%(1_I, ev_{A,C})$ & $=$ \\
%$(1_I, \langle \pi_{I,A \times B}, \langle \pi'_{I,A \times B};\pi_{A,B}, \langle \pi_{I,A \times B}, \pi'_{I,A \times B};\pi'_%{A,B}\rangle;\overline{f} \rangle);
%(1_I, ev_{A,C}) \rangle$ & $=$ \\
%\end{tabular}

$(b) \Rightarrow (a)$

\newcommand{\sar}{\overset{\ast}{\Rightarrow}}

Should use the fact that $\mathbb B \dd 1 \cong \mathbb B$. If two categories is equivalent and one is Cartesian closed,
does that mean the other is Cartesian closed? I suspect the answer is yes. TODO: prove this.

$(a) \Rightarrow (c)$

We rename the $I \Rightarrow (-)$ functor to $I \overset{\ast}{\Rightarrow} (-)$, using $I \Rightarrow (-)$ instead
for the covariant exponential functor in $\mathbb B$.

For $I^* \dashv I \sar (-)$, we need the following adjunction diagram to hold.
\begin{center}
\begin{tikzcd}
I \sar (I,C) \ar[r,"I \sar (1_I \text{,} g)"] & I \sar (I,D) \\
C \ar[u, " "] \ar[ur, "f" below] &
 & & \\
(I,C) \ar[r, "(1_I \text{,} g)"] \ar[dr, "I^*(f)~~~~" below] & (I,D) \\
 & (I, I \sar (I,D)) \ar[u, " "]
\end{tikzcd}
\end{center}

First, we need a bijection between arrows $g : I \times C \to D$ and $f : C \to I \sar (I,D)$.
Such a bijection exists if we define $I \sar (I,C) \doteq I \Rightarrow C$, falling
out of the adjunction $(-) \times I \dashv I \Rightarrow (-)$.
We define $I \sar (1_I, g : I \times C \to D) \doteq \Lambda(s_{I \Rightarrow C, I};\langle \pi_{I,I \Rightarrow C}, ev_{I,C}\rangle;g)$.
\begin{center}
\begin{tikzcd}
I \Rightarrow C \ar[r,"I \sar (1_I\text{,}g)"] & I \Rightarrow D \\
C \ar[u, " "] \ar[ur, "\Lambda{g}" below] &
 & & \\
(I,C) \ar[r, "(1_I \text{,} g)"] \ar[dr, "(1_I \text{,} \pi';\Lambda g)~~~~~~~" below] & (I,D) \\
 & (I, I \Rightarrow D) \ar[u, " "]
\end{tikzcd}
\end{center}

We just need natural transformations making these triangles commute. Here is what we propose:

\begin{center}
\begin{tikzcd}
I \Rightarrow C \ar[r,"I \sar (1_I\text{,}g)"] & I \Rightarrow D \\
C \ar[u, "\Lambda \pi'_{I,C}"] \ar[ur, "\Lambda g" below] &
 & & \\
(I,C) \ar[r, "(1_I \text{,} g)"] \ar[dr, "(1_I \text{,} \pi';\Lambda g)~~~~~~~" below] & (I,D) \\
 & (I, I \Rightarrow D) \ar[u, "(1_I\text{,} ev_{I,D})" right]
\end{tikzcd}
\end{center}

TODO: show that these commute and that they are natural.

$(c) \Rightarrow (b)$

\subsection*{1.3.3}

We need a bijection between arrows in $\mathbb B$ of the form $$X \overset{a}{\longrightarrow} \Pi_{I} \ddisp{Y}{f}{I}$$
and arrows in $\mathbb B / I$ of the form 
\begin{center}
\begin{tikzcd}
I \times X \ar[d,"\pi" left] \ar[r,"h"] & \ar[dl,"f"] Y \\
I & 
\end{tikzcd}
\end{center}

For simplicity, we'll assume exponents curry the left component of products rather than the right, so given $A \times B \overset{\varphi}{\to} C$ we have $B \overset{\Lambda(\varphi)}{\to} (A \Rightarrow C)$
Let's start going from top to bottom. Suppose we have an arrow $X \overset{a}{\longrightarrow} \Pi_{I} (Y \overset{f}{\to} I)$.
By naturality of the isomorphism $\mathit{Hom}(I \times X, -) \cong \mathit{Hom}(X, I \Rightarrow -)$ corresponding to the adjunction $I \times - \dashv (I \Rightarrow -)$, we have that the conclusion of 

\begin{prooftree}

\AxiomC{$X \overset{a;e}{\longrightarrow} (I \Rightarrow Y)$}
\UnaryInfC{$X \overset{a;e}{\longrightarrow} (I \Rightarrow Y) \overset{I \Rightarrow f}{\longrightarrow} (I \Rightarrow I)$}
\doubleLine
\UnaryInfC{$I \times X \overset{\Lambda^{-1}(a;e;I \Rightarrow f)}{\longrightarrow} I$}

\end{prooftree}

is equal to the conclusion of 

\begin{prooftree}

\AxiomC{$X \overset{a;e}{\longrightarrow} (I \Rightarrow Y)$}
\doubleLine
\UnaryInfC{$I \times X \overset{\Lambda^{-1}(a;e)}{\longrightarrow} Y$}
\UnaryInfC{$I \times X \overset{\Lambda^{-1}(a;e)}{\longrightarrow} Y \overset{f}{\longrightarrow} I$}

\end{prooftree}

If we can show that the conclusion on the top is a left projection, then we can use $\Lambda^{-1}(a;e)$ for $h$ 
(the transpose of $a$). To do so, we first recall the ``two triangle'' definition of adjunctions, instantiated 
for the adjunction $(I \times -) \dashv (I \Rightarrow -)$

\begin{center}
\begin{tikzcd}
X \ar[r, "f", dotted] & (I \Rightarrow Z) \\
I \times X \ar[dr, "I \times f~~~" below] \ar[r, "\Lambda^{-1}f" above] & Z \\
 & I \times (I \times Z) \ar[u, "\mathit{ev}_{I,Z}" right]
\end{tikzcd}
\end{center}

With this and the equalizer property, we simplify the top conclusion:\\~\\
\begin{tabular}{ll}
$\Lambda^{-1}(a;e;I \Rightarrow f)$ & \\
$= \Lambda^{-1}(a;e;\Lambda(\pi_{I,I \Rightarrow Y}))$ & (equalizer property) \\
$= I \times (a;e;\Lambda(\pi_{I,I \Rightarrow Y}));\mathit{ev}_{I,I}$ & (adjunction triangle) \\
$= (I \times (a;e));(I \times \Lambda(\pi_{I,I \Rightarrow Y}));\mathit{ev}_{I,I}$ & (functoriality of $I \times -$) \\
$= (I \times (a;e));\Lambda^{-1}\Lambda(\pi_{I,I \Rightarrow Y})$ & (adjunction triangle) \\
$= (I \times (a;e));\pi_{I,I \Rightarrow Y}$ & ($\Lambda$ is an iso)  \\
$= \langle \pi_{I,X};\mathit{id}_I, \pi_{I,X}';a;e \rangle;\pi_{I,I \Rightarrow Y}$ & (basic property of products) \\
$= \pi_{I,X}$ & (basic property of products)
 
\end{tabular}\\~\\
To go from top to bottom, we start with an $h : I \times X \to Y$ such that $h;f = \pi_{I,X}$.

We have $\Lambda(h) : X \to (I \Rightarrow Y)$, and so if $\Lambda(h);(I \Rightarrow f) = \Lambda(h);\Lambda(\pi)$ 
then, by the equalizer property, there is a unique arrow $a : X \to \Pi_I(f : Y \to I)$ such that $a;e=h$.
We will choose $a$ as the transpose of $h$, but first we must show that it exists, i.e. that 
$\Lambda(h);(I \Rightarrow f) = \Lambda(h);\Lambda(\pi)$. To this end, we have:

\begin{center}
\begin{tabular}{ll}
$\Lambda h;\Lambda \pi$ & \\
$= \Lambda \Lambda^{-1} (\Lambda h; \Lambda \pi)$ & ($\Lambda$ is an iso) \\
$= \Lambda((I \times (\Lambda h;\Lambda \pi));\mathit{ev})$ & (adjunction triangle) \\
$= \Lambda((I \times \Lambda h);(I \times \Lambda\pi);\mathit{ev})$ & (functoriality of $I \times -$) \\
$= \Lambda((I \times \Lambda h);\Lambda^{-1}\Lambda \pi)$ & (adjunction triangle) \\
$= \Lambda((I \times \Lambda h);\pi)$ & ($\Lambda$ is an iso) \\
$= \Lambda(\langle \pi;\mathit{id}, \pi';\Lambda h \rangle;\pi)$ & (a basic property of products) \\
$= \Lambda \pi$ & (a basic property of products) \\
$= \Lambda(h;f)$ & (substitution) \\
$= \Lambda h;(I \Rightarrow f)$ & (naturality of $\mathit{Hom}(I \times X, -) \cong \mathit{Hom}(X, I \Rightarrow -)$) 
\end{tabular}
\end{center}
TODO: show adjunction is isomorphism \\
TODO: show adjunction is natural

\subsection*{1.3.4}

First, we review the functoriality of $I \times (-) : \mathbb B \to \mathbb B$.
We have $I \times (A) \doteq I \times A$, $I \times (f : A \to B) \doteq 1_I \times f : I \times A \to I \times B$.
This preserves identities, since 
$I \times 1_A = 1_I \times 1_A = \langle \pi;1_I, \pi';1_A \rangle = \langle \pi, \pi' \rangle = 1_{(I \times A)}$.
Also, it preserves composition, since 
$I \times (f;g) = 1_I \times (f;g) = (1_I;1_I) \times (f;g) = (1_I \times f) ; (1_I \times g)$.\\~\\
We define $\delta_A : I \times A \to I \times (I \times A)$ as 
$\delta_A \doteq \langle \pi_{I,A}, 1_{I \times A}\rangle$. We define $\epsilon_A : I \times A \to A$ as 
$\epsilon_A \doteq \pi'_{I,A}$.
We must show the comonad laws.\\~\\
1.) $\delta_{A};\delta_{I \times A} = \delta_{A};(I \times \delta_A)$\\~\\
\begin{tabular}{lll}
$\delta_A;\delta_{I \times A}$ & $=$ & \\
$\langle \pi_{I,A},1_{I \times A} \rangle;\langle \pi_{I, I \times A}, 1_{I \times (I \times A)} \rangle$ & $=$ & \\ 
$\langle \langle \pi_{I,A},1_{I \times A} \rangle;\pi_{I, I \times A}, \langle \pi_{I,A};1_{I \times A} \rangle;1_{I \times (I \times A)} \rangle$ & $=$ & \\ 
$\langle \pi_{I,A}, \langle \pi_{I,A},1_{I \times A} \rangle \rangle$ & $=$ & \\

$\langle \pi_{I,A}, \delta_A \rangle$ & $=$ & \\

$\langle \pi_{I,A}, 1_{I \times A};\delta_A \rangle$ & $=$ & \\

$\langle \langle \pi_{I,A},1_{I \times A} \rangle;\pi_{I,I \times A}, \langle \pi_{I,A},1_{I \times A} \rangle;\pi'_{I,I \times A};\delta_A \rangle$ & $=$ & \\

$\langle \pi_{I,A},1_{I \times A} \rangle;\langle \pi_{I,I \times A}, \pi'_{I,I \times A};\delta_A \rangle$ & $=$ & \\
$\langle \pi_{I,A},1_{I \times A} \rangle;\langle \pi_{I,I \times A};1_I, \pi'_{I,I \times A};\delta_A \rangle$ & $=$ & \\
$\delta_A;(I \times \delta_A)$ & &

\end{tabular}
\\~\\~\\
2.) $\delta_A;(I \times \epsilon_A) = \delta_{A};\epsilon_{I \times A}$\\~\\
\begin{tabular}{lll}
$\delta_A;(I \times \epsilon_A)$ & $=$ & \\
$\langle \pi_{I,A}, 1_{I \times A}\rangle; \langle \pi_{I,I \times A};1_I, \pi'_{I,I \times A};\epsilon_A \rangle$ & $=$ & \\
$\langle \pi_{I,A}, 1_{I \times A}\rangle; \langle \pi_{I,I \times A}, \pi'_{I,I \times A};\epsilon_A \rangle$ & $=$ & \\
$\langle \langle \pi_{I,A}, 1_{I \times A}\rangle;\pi_{I,I \times A}, \langle \pi_{I,A}, 1_{I \times A}\rangle;\pi'_{I,I \times A};\epsilon_A \rangle$ & $=$ & \\
$\langle \pi_{I,A}, 1_{I \times A};\epsilon_A \rangle$ & $=$ & \\
$\langle \pi_{I,A}, \pi'_{I,A} \rangle$ & $=$ & \\
$1_{I \times A}$ & $=$ & \\
$\langle \pi_{I,A}, 1_{I \times A} \rangle;\pi'_{I,I \times A} $ & $=$ & \\
$\delta_A;\epsilon_{I \times A}$ & &
\end{tabular}

\subsubsection*{ii}

The (co)Kleisli category generated by this monad is as follows. Objects are objects of $\mathbb B$.
An arrow $f : A \to B$ is an arrow $f^* : (I \times A) \to B$ of $\mathbb B$.

We defined the composition of two arrows $f : A \to B$ and $g : B \to C$ as
$f;g \doteq \delta_{A};(I \times f^*);g^*$.\\~\\
Then,\\~\\
\begin{tabular}{lll}
$f;g$ & $=$ & \\
$\delta_{A};(I \times f^*);g^*$ & $=$ & \\
$\langle \pi_{I,A}, 1_{I \times A} \rangle;(I \times f^*);g^*$ & $=$ & \\
$\langle \pi_{I,A}, 1_{I \times A} \rangle;\langle \pi_{I, I \times A};1_I, \pi'_{I, I \times A};f^* \rangle;g^*$ & $=$ & \\
$\langle \pi_{I,A}, 1_{I \times A} \rangle;\langle \pi_{I, I \times A}, \pi'_{I, I \times A};f^* \rangle;g^*$ & $=$ & \\
$\langle \langle \pi_{I,A}, 1_{I \times A} \rangle;\pi_{I, I \times A}, \langle \pi_{I,A}, 1_{I \times A} \rangle;\pi'_{I, I \times A};f^* \rangle;g^*$ & $=$ & \\
$\langle \pi_{I,A}, 1_{I \times A};f^* \rangle;g^*$ & $=$ & \\

$\langle \pi_{I,A}, f^* \rangle;g^*$ & $=$ & \\
\end{tabular}\\~\\
Note that this is essentially the simple slice category $\mathbb B / \! / I$.
\\~\\~\\
I've never learned about Eilenberg-Moore categories, so I'm first going to copy down their definition from
Awodey, to make sure I have things straight. The category $\mathbb C^{I \times (-)}$ has as \emph{objects}
the ``T-Coalgebras'', which are pairs $(A,\alpha)$ of the form $\alpha : A \to I \times A$ in $\mathbb C$,
such that
\begin{mathpar}
1_A = \alpha;\epsilon_{A} \and \text{and} \and \alpha;\delta_A = \alpha;(I \times \alpha)
\end{mathpar} 
A \emph{morphism} of T-Algebras,
$$ h : (A,\alpha) \to (B,\beta) $$
is simply an arrow $h : A \to B$ in $\mathbb C$, such that, 
$$h;\beta = \alpha;(I \times h)$$

Comparing the form of the objects of the EM category to the form of the objects of $\mathbb B / I$,
it seems clear that they are not the same category. What I suspect is that there is an equivalence of
categories. TODO: define and prove this equivalence.

\subsection*{1.3.5}

Let $\mathbb B$ have finite limits. (Umm... I proved this without using finite limits. What is going on?)
~\\~\\
$(\Rightarrow)$~\\
Let $(u,f) : \ddisp{X}{p}{I} \to \ddisp{Y}{q}{J}$ be a mono in $\mathbb B^{\to}$. Then for all 
$(v,g),(w,h) : \ddisp{Z}{r}{K} \to \ddisp{X}{p}{I}$ with $r;v = g;p$ and $r;w = h;p$, 
we have $(v,g);(u,f) = (w,h);(u,f)$ implies 
$(v,g) = (w,h)$, i.e. $(v;u, g;f) = (w;u,h;f)$ implies $v = w$ and $g = h$.

In particular, letting $K = I$ and $v,w = 1_I$, we have $g;f = h;f$ implies $g = h$.
Hence $f$ is a mono in $\mathbb B$. 

On the other hand, letting $Z = X$ and $g,h = 1_X$, we have $v;u = w;u$ implies $v = w$.
Hence $u$ is a mono in $\mathbb B$.
~\\~\\
$(\Leftarrow)$~\\
Let $(u,f) : \ddisp{X}{p}{I} \to \ddisp{Y}{q}{J}$ be an arrow in $\mathbb B^{\to}$ %(i.e., $p;u = f;q$)
such that $u$ and $f$ are monos, i.e.,
\begin{itemize}
\item For all $v,w : K \to I$, $v;u = w;u$ implies $v = w$.
\item For all $g,h : Z \to X$, $g;f = h;f$ implies $g = h$.
\end{itemize}
Let $(v,g),(w,h) : \ddisp{Z}{r}{K} \to \ddisp{X}{p}{I}$ %(and so $r;v = g;p$ and $r;w = h;p$)
with $(v,g);(u,f) = (w,h);(u,f)$, i.e. $(v;u,g;f) = (w;u,h;f)$. Then $v = w$ and $g = h$. 




\section*{1.5}

\subsection*{1.5.1}

Yep, $FinFam(\mbf{Sets})$ consists of families of arbitrary sets indexed by finite sets,
while $Fam(\mbf{FinSets})$ consists of families of finite sets indexed by arbitrary sets.

\subsection*{1.5.2}

\begin{tikzcd}
PERFam(\mbf{\omega\text{-}Sets}) \ar[d] \ar[r] & \ar[d] UFam(\omega\text{-}Sets) \\
\mbf{PER} \ar[r, hookrightarrow] & \omega\text{-}\mbf{Sets} 
\end{tikzcd}

There's nothing particularly interesting about this. It's split because change-of-base on a split
fibration produces a split fibration. 

\subsection*{1.5.3}

\subsubsection*{(i)}

It's clear that restricting a fibration ($p$ in this case), does not destroy the properties of fibrations.

\subsubsection*{(ii)}

TODO: having trouble with this one.

\subsection*{1.5.4}
\subsection*{1.5.5}

\subsection*{1.5.6}

\subsubsection*{(i)}

It's important to note that distributivity doesn't posit the existence of an \emph{arbitrary} iso pair 
$(X \times X) + (Y \times X) \cong (X + Y) \times X$; instead it requires that a specific arrow (the ``canonical'' one)
$\alpha : (X \times X) + (Y \times X) \to (X + Y) \times X$, where 
$\alpha \doteq [(\kappa \times id), (\kappa' \times id)]$ has an inverse $\alpha^{-1} : (X  + Y) \times X \to (X \times X) + (Y \times X)$.\\~\\
Then we have the following diagram:\\~\\
\begin{tikzcd}
 & (X \times X) + (Y \times X) \ar[r, "\text{[}id\text{,} \langle \pi\text{,} \pi \rangle\text{]}"] \ar[d,"\text{[}(\kappa \times id)\text{,}(\kappa' \times id)\text{]}"] & X \times X \\
X \times X \ar[ur, "\kappa", bend left = 20] \ar[r,"\kappa \times id"] & (X + Y) \times X \ar[u,"\alpha^{-1}"] &
\end{tikzcd}

It's clear from the UMP of coproducts that the path along the top edge is the identity arrow at $X \times X$.
So if we can just prove that the left triangle commutes ($(\kappa \times id);\alpha^{-1} = \kappa$), 
we will have that $\alpha^{-1};[id,\langle \pi,\pi\rangle]$ is a retraction of $\kappa \times id$,
demonstrating that it is a split mono.\\~\\
All we need is to make another appeal to the UMP of coproducts:\\~\\
\begin{tikzcd}[sep=90]
X \times X \ar[dr, "\kappa \times id~~~~~" left] \ar[r,"\kappa"] & (X \times X) + (X \times Y) \ar[d, "\text{[}(\kappa \times id)\text{,} (\kappa' \times id)\text{]}" right, bend left = 20, dashed] & \ar[l,"\kappa'" above] \ar[dl, "\kappa' \times id"] Y \times X \\
 & (X + Y) \times X \ar[u,"\alpha^{-1}", bend left = 20] & 
\end{tikzcd}\\~\\
We thus have $\kappa = \kappa;[(\kappa \times id), (\kappa' \times id)];\alpha^{-1} = (\kappa \times id);\alpha^{-1}$,
making the left triangle in the previous diagram commute as desired.

\subsection*{(ii)}

We have 

$$ g = f $$

if and only if (apparent if we squint long enough at the UMP of products)

$$\langle g , f \rangle = \langle f, f \rangle$$

if and only if (because $\kappa \times id$ is a mono)

$$\langle g, f \rangle;(\kappa \times id) = \langle f, f \rangle;(\kappa \times id)$$

if and only if (from basic properties of products)

$$\langle g;\kappa, f;id \rangle = \langle f;\kappa, f;id \rangle$$

if and only if (again, squinting at the UMP)

$$ g;\kappa = f;\kappa $$

But we are assuming $g;\kappa = f;\kappa$. Therefore, $g = f$.

\subsubsection*{(iii)}

First, we establish that $0 + 0 \cong 0$.

\begin{tikzcd}[sep=70]
0 \ar[r, "\kappa_{0,0}"] \ar[dr, "id_0~~" below] & 0 + 0 \ar[d, "\text{[}id_0\text{,}id_0\text{]}", bend left = 20] & \ar[dl, "~~id_0" below] \ar[l, "\kappa'_{0,0}" above] 0 \\
 & 0 \ar[u, "\kappa_{0,0}", bend left = 20] &
\end{tikzcd}

Since $0$ is initial, we have $\kappa_{0,0} = \kappa'_{0,0}$. From basic properties of coproducts, we get 
$$[id_0,id_0];\kappa_{0,0} = [\kappa_{0,0},\kappa_{0,0}] = [\kappa_{0,0}, \kappa'_{0,0}] = id_{0 + 0}$$.
On the other hand,
$$\kappa_{0,0};[id_0,id_0] = id_{0}$$
follows from the UMP of coproducts.

Getting back to the scenario presented by the exercise, where the coproduct under consideration is 
$(0 \times Z) + (0 \times Z)$, we now have a helpful clue:
$$0 \times Z \cong (0 + 0) \times Z \cong (0 \times Z) + (0 \times Z)$$ 


%First, complete the diagram by adding arrows $\kappa : X \to (X + Y)$ and $\pi : (X + Y) \times X \to (X + Y)$.
%From basic properties of products, we see that both of the triangles and also the square of this diagram commute.
%Starting from the ``mono hypthesis''. $\kappa \circ f = \kapp \circ g$, by substitution we get
%$\kappa \circ \pi \circ \langle f, f \rangle = \kappa \circ \pi \circ \langle f , g \rangle$, and then
%$\pi \circ (\kappa \times id) \circ \langle f, f \rangle= \pi \circ (\kappa \times id) \circ \langle f , g \rangle$.



%

\section*{1.6}

\subsection*{1.6.1}

Groups:\\
identity: $() \to X$\\
inverse: $X \to X$\\
actions: $(X,X) \to X$\\
\\~\\
Vector Spaces:\\
$+ : (V,V) \to V$\\
$* : (K,V) \to V$\\
$0_V : () \to V$\\
$+ : (K,K) \to K$\\
$* : (K,K) \to K$\\
$1_K : () \to K$\\
$0_K : () \to K$

\subsection*{1.6.2}

\begin{tikzcd}
FinSign \ar[r] \ar[d] \ar[dr, phantom, "\lrcorner", very near start]        & FinFam(FinSets) \ar[d] \\
FinSets \ar[r, "T \mapsto T^* \times T" below] & FinSets 
\end{tikzcd}

\subsection*{1.6.3}

\subsubsection*{i}

An object of this pullback is a set $T$ of types paired with a display map $d : X \to (T^* \times T)$
indexed by the function signatures generated by $T$. A ``fibre'' of $d$ is a set of all function symbols
sharing a specific signature.

\subsubsection*{ii}

Intuitively, it seems like the key is that the $Sets^{\to}$ and $Fam(Sets)$ are equivalent categories.
We need some lemma saying that if the total categories of two fibrations sharing a base category are equivalent
(or, perhaps stronger: we have an equivalence in the 2-category $\mbf{Fib}$), then their pullbacks along a common arrow
are also equivalent.

TODO: prove this lemma. Actually, this lemma has already been proven in lemma 1.7.9, pdf pg 10056

\subsubsection*{iii}

The ``fibres'' of a display map are disjoint. Therefore no function symbol can have two distinct signatures.

\subsection*{1.6.4}

An object is a set (modelling the single type) paired with a signature $\Sigma$ where $|\Sigma|$ is a singleton.
An arrow $M \to N$ is a function from $M$'s carrier set to $N$'s, paired with a mapping from $M$'s functions to $N$'s,
commuting algebraically, as morphisms of models do.

I hope that's enough detail.

\subsection*{1.6.5}

The diagram is similar to the formation of the $Sign$ category. The only difference is the bottom edge.
If we want only predicates, we put $T^*$ at the bottom edge. If we want predicates and functions,
we put $T^* + (T^* \times T)$.

\subsection*{1.6.6}


\section*{Lemma 1.7.9}

Lemma 1.7.9 states the following:\\~\\ 
Let $K : \mathbb A \to \mathbb B$ be a functor. Change-of-base along $K$ yields a 2-functor 
$K^* : \mbf{Fib}(\mathbb B) \to \mbf{Fib}(\mathbb A)$.\\~\\
The proof is elided, because it is easy. Maybe so, but its an important lemma, and though its proof may be easy, it is not
trivial. There is quite a bit to expand. \\~\\
First, we consider how $K^*$ transforms $1$-cells, i.e. fibred functors 
$$F : \ddisp{\mathbb E}{p}{\mathbb B} \to \ddisp{\mathbb D}{q}{\mathbb B}$$\\
\begin{center}
\begin{tikzcd}
  & \mathbb A \times_{\mathbb B} \mathbb D \ar[rrrr] \ar[dd, "K^*q" near start] & & & & \mathbb D \ar[ddl, "q"] &  \\
\mathbb A \times_{\mathbb B} \mathbb E  \ar[rrr] \ar[dr, "K^*p~~~~~" below] \ar[ur, "K^*F", bend left = 10] & & & \mathbb{E} \ar[urr, "F", bend left = 10] \ar[rd, "p~" left, rightarrow] & &  \\
 & \mathbb A \ar[rrr, "K"] & & & \mathbb B & 
\end{tikzcd}
\end{center}
 
$\mathbb A \times_{\mathbb B} \mathbb E$ has as objects pairs $(A,E)$ where $A \in \mathbb A_0$, $E \in \mathbb E_0$,
and $KA = pE$. We define $(K^*F)(A,E) \doteq (A,FE)$. $(A,FE)$ is indeed an element of $\mathbb A \times_{\mathbb B} \mathbb D$,
\begin{center}
\begin{tabular}{lll}
$KA$ & $=$ & (by assumption) \\
$pE$ & $=$ & (commutativity of triangle) \\
$q(FE)$ & &
\end{tabular}
\end{center}

$\mathbb A \times_{\mathbb B} \mathbb E$ has as arrows pairs $(a : A_1 \to A_2, e : E_1 \to E_2)$, where $Ka = pe$.
We define $(K^* F)(a,e) \doteq (a,Fe)$. $(a, Fe)$ is indeed an arrow of $\mathbb A \times_{\mathbb B} \mathbb D$,
since
\begin{center}
\begin{tabular}{lll}
$Ka$ & $=$ & (by assumption) \\
$pe$ & $=$ & (commutativity of triangle) \\
$q(Fe)$ & &
\end{tabular}
\end{center}

Now, we must prove that $K^*F$ is a fibred functor in $\mbf{Fib}(\mathbb B)$. This entails two things:
\begin{enumerate}
\item $K^*F$ preserves Cartesian arrows
\item $(K^*F);(K^*q) = K^*p$
\end{enumerate}
~\\~\\
(1) Let $(u : A_1 \to A_2, e : E_1 \to E_2)$ be a Cartesian arrow in $\mathbb A \times_{\mathbb B} \mathbb E$.
\\~\\
For any total category $\mathbb C$ over $\mathbb B$, it's not hard to see that $(v,f)$ is Cartesian
in $\mathbb A \times_{\mathbb B} \mathbb C$ iff $f$ is Cartesian in $\mathbb C$. (Draw a diagram.)
Thus $e$ is Cartesian in $\mathbb E$. Since $F$ is a fibred functor, $Fe$ is Cartesian in $\mathbb D$,
and so $(u,Fe)$ is Cartesian in $A \times_{\mathbb B} \mathbb D$.
\\~\\
(2) This is obvious from the way that we have defined $K^*$: it does not change the $\mathbb A$ (left) component.

Now, letting 
$$F,G : \ddisp{\mathbb E}{p}{\mathbb B} \rightrightarrows \ddisp{\mathbb D}{q}{\mathbb B}$$
we consider how $K^*$ transforms $2$-cells, i.e.
$\eta : F \to G$

Recall that such a $2$-cell $\eta$ is defined as a natural transformation from $F$ to $G$ such that each component
of $\eta$ is vertical, i.e., $\eta_A : FA \to GA$ is over the identity arrow of object $pA$.

We define $(K^*\eta)_{(X,A)} \doteq (1_X, \eta_A)$. Clearly $K^*\eta$ is a vertical natural transformation between fibred 
functors.
 


%We define $a' \doteq (I, i \mapsto a) : (I, i \mapsto Z) \to (I, X_i)$ and $b' \doteq (I,i \mapsto b) : 
%(I, i \mapsto Z) \to (I%, X_i)$. Then $a';f = (I, i \mapsto (a;f_i))$ = a';f = b';f = (I, i \mapsto (b;f_i))$.
%In particular, for some $i \in I$

\section*{Exercises 1.7}

\subsection*{1.7.1}

I will just show that $\mbf{Fib}$ has finite products.

Given two fibrations $\vrt{\mathbb E}{\mathbb B}p$ and $\vrt{\mathbb D}{\mathbb A}q$, we can form their product.
$\vrt{\mathbb E \times \mathbb D}{\mathbb B \times \mathbb A}$ as follows. The total category is $\mathbb E \times \mathbb D$.
The fibration functor $p \times q$ maps an object $(E,D)$ to $(pE,qD)$ and an arrow $(f,g)$ to $(pf,qg)$.

We define $\pi_{p,q} : p \times q \to p \doteq (\pi_{\mathbb B, \mathbb A},\pi_{\mathbb E, \mathbb D})$ 
We define
$\pi'_{p,q} : p \times q \to q \doteq (\pi'_{\mathbb B, \mathbb A}, \pi'_{\mathbb E, \mathbb D})$.
$\pi_{p,q}$ is an arrow in $\mathbb{Fib}$, since:

\begin{itemize}
\item $\pi_{\mathbb E, \mathbb D}$ preserves Cartesian arrows. 
Let $(f,g) : (X_E,X_D) \to (Y_E,Y_D)$ be a cartesian arrow in $\mathbb E \times \mathbb D$
over $(u,v) \doteq (pf,qg)$.
Then for all $(h,k) : (Z_E,Z_D) \to (Y_E,Y_D)$... TODO: this is tedious and I will finish it later.

\item $\pi_{\mathbb E,\mathbb D};p = (p \times q);\pi_{\mathbb B, \mathbb A}$
% show that they agree on all arrows from the generating category 2?
\end{itemize}

Let $\vrt{\mathbb F}{\mathbb C} r$ be a fibration and let $H : $

\section*{Lemma 1.8.9}

On page 10065, we have the statement of lemma 1.8.9.
It involves fibrations $\vrt{\mathbb E}{\mathbb B}p$ and $\vrt{\mathbb D}{\mathbb B}q$
and a fibred functor $H : \mathbb E \to \mathbb D$. For each object $I \in \mathbb B$ the functor 
$H_I : \mathbb E_I \to \mathbb D_I$ is the restriction of $H$ to the fibre over $I$.

Later, an arbitrary map $u : I \to J$ and two reindexing functors $u^* : \mathbb E_J \to \mathbb E_I$ and $u^{\#} : \mathbb D_J \to \mathbb D_I$ are assumed.

At the bottom of the lemma statement, a natural isomorphism $u^{\#} H_J \cong H_I u^*$ is referenced.
But just what is this natural isomorphism?

Since $H$ preserves Cartesian arrows, for $X \in \mathbb E_J$,  $H \overline{u^*}(X) : H_I u^*(X) \to H_J X$ is Cartesian. Also, $\overline{u^{\#}} H_J(X) : u^{\#} H_J(X) \to H_J(X)$ is Cartesian. The domains of any pair of Cartesian arrows with the same 
codomain are isomorphic, and so $H_I u^*(X) \cong u^{\#} H_J(X)$: this is the component of our natural isomorphism at $X$.

That is, our component is the vertical arrow $\alpha_X$ in the following diagram:\\~\\
\begin{tikzcd}
H_I(u^*(X)) \ar[dd, "\alpha^{-1}_X" left, bend right = 20] \ar[rrd, "H(\overline{u^*}(X))"] & & \\
 & & H_{J}(X) \\ 
u^{\#}(H_J(X)) \ar[uu, "\alpha_X" right, bend right = 20] \ar[rru,"~~~~~~~~~~~~\overline{u^{\#}}(H_J(X))" below]  & &
\end{tikzcd}~\\~\\

%\ar[ur, ""]
Now that we have the components, we must show naturality. In other words, we must show that the following square commutes.

\begin{tikzcd}
u^{\#} H_J(X) \ar[dd,"\alpha_X" left] \ar[rr, "u^{\#}(H_J(f))"] & & u^{\#} H_J(Y) \ar[dd,"\alpha_Y"] \\
 & & \\
H_I(u^*(X)) \ar[rr,"H_I(u^*(f))" below] & & H_I(u^*(Y))
\end{tikzcd}

We might use the equalities $$u^*(f);\overline{u^*}(Y) = \overline{u^*}(X);f$$ and 
$$u^{\#}(H_J(f));\overline{u^{\#}}(H_J(Y)) = \overline{u^{\#}}(H_J(X));H_J(f)$$
which follow from the definition of reindexing functors.


\subsection*{Oh thats it}

In the above attempt, I had two diagrams, but I should have pasted them together. 
(In fact, I also needed to ``clone'' the first diagram, making the second copy at $Y$ rather than $X$). 
This would have yielded a commutative 
diagram shaped like a wedge of cheesecake. The naturality of the square (the lower diagram) would then follow
from Cartesianness.

The canonical natural transformation $K_I u^{\#} \Rightarrow u^* K_J$ is simply the transpose of:

$$u^{\#} \overset{u^{\#}\eta_J}{\longrightarrow} u^{\#} H_J K_J \overset{\cong}{\longrightarrow} H_I u^* K_J $$

Across the adjunction $K_I \vdash H_I$ (with $\eta_J$ being the unit of this adjunction). 
The right part of this composite is the horizontal composition of the natural isomorphism $u^{\#}H_J \cong H_I u^*$ with
the identity natural transformation $K_J$. The existence and
naturality of the former natural isomorphism can be derived from Cartesianness, as described in the section header.

Now, we just ``take the transpose'' of 

$$u^{\#} \overset{u^{\#}\eta_J}{\longrightarrow} u^{\#} H_J K_J \overset{\cong}{\longrightarrow} H_I u^* K_J $$

to obtain the canonical transformation.

$$K_I u^{\#} \to u^* K_J $$

How does this taking the transpose work? Let's generalize to explain.
Suppose we have:

\begin{itemize}
\item An adjunction $F : \mathbb C \leftrightarrows \mathbb D : U$, where the bijection underlying this adjunction
 is $\phi_{C,D} : Hom(FC,D) \cong Hom(C,UD) : \psi_{C,D}$, natural separately in both arguments.
\item A functor $G : \mathbb E \to \mathbb C$, and
\item A functor $H : \mathbb E \to \mathbb D$.
\end{itemize}

We claim that for any natural transformation $\alpha : FG \to H$, we obtain another natural transformation
$\phi(\alpha) : G \to UH$ by mapping each component $\alpha_C : FGC \to HC$ through $\phi$ to get 
$\phi(\alpha)_C : GC \to UHC$. Here's why. We start with the assumption that $\alpha$ is a natural transformation:

\begin{center}
\begin{tikzcd}[sep=80.0]
FGC \ar[rr, "\alpha_C"] \ar[dd, "FGf" left] \ar[ddrr, "FGf;\alpha_D = \alpha_C;Hf "] & & HC \ar[dd,"Hf"] \\
 & & \\ 
FGD \ar[rr, "\alpha_D" below] & & HD 
\end{tikzcd}
\end{center}

We would like to prove that the following square commutes

\begin{center}
\begin{tikzcd}[sep=80.0]
GC \ar[rr, "\phi(\alpha_C)"] \ar[dd, "Gf" left] \ar[ddrr, "Gf;\phi(\alpha_D) = \phi(\alpha_C);UHf "] & & UHC \ar[dd,"UHf"] \\
 & & \\ 
GD \ar[rr, "\phi(\alpha_D)" below] & & UHD 
\end{tikzcd}
\end{center}

We use the naturality of $\phi$:

\begin{tikzcd}
Hom(FGC, HC) \ar[r,"\phi"] \ar[d,"Hom(FGC\text{,}Hf)" left] & Hom(GC,UHC) \ar[d,"Hom(GC\text{,}UHf)"] \\
Hom(FGC, HD) \ar[r,"\phi"]                                  & Hom(GC,UHD) \\
Hom(FGD, HD) \ar[u, "Hom(FGf\text{,}HD)"] \ar[r, "\phi"]           & Hom(GD,UHD) \ar[u, "Hom(Gf\text{,} UHD)" right]  
\end{tikzcd}
~\\~\\
The top square gives us $\phi(\alpha_C;Hf) = \phi(\alpha_C);UHf$. The bottom square gives
$\phi(FGf;\alpha_D) = Gf;\phi(\alpha_D)$. By naturality of $\alpha$ we have $FGf;\alpha_D = \alpha_C;Hf$, and so
$\phi(FGf;\alpha_D) = \phi(\alpha_C;Hf)$. Hence $\phi(\alpha_C);Uhf = \phi(\alpha_C;Hf) = \phi(FGf;\alpha_D) = Gf;\phi(\alpha_D)$.

\section*{1.8}

\subsection*{1.8.1}

Explain in detail what a ``fibred LCCC'' is.\\~\\

I'll assume that this means we must have LCCC structure in each fibre of a fibration.
That is, an $LCCC$ is a fibration $\vrt{\mathbb E}{\mathbb B}p$ such that each fibre
$\mathbb E_I$ is locally Cartesian Closed. Furthermore, each reindexing functor $u^* : \mathbb E_J \to \mathbb E_I$
must preserve $LCCC$ structure. In other words, each $d \in \mathbb E_J$, $f : a \to d$, and $g : b \to d$,
the product $(f \times_{LCCC} g) : c \to d$ must reindex to $u^*(f \times_{LCCC} g) \cong u^*(f) \times_{LCCC} u^*(g)$. 
Likewise for exponentials in fibred slices. 

\subsection*{1.8.2}

Suppose that the collection of types $T$ contains one of $\mathbb B$'s terminal objects: let's call it $T_1$.
Let $\vrt{s(T)}{\mathbb B}p$ be the fibration under consideration.
Let $(I,X)$ be an object in $s(T)_I$. Suppose we have an arrow $(1_I,f) : (I,X) \to (I, T_1)$ in $s(T)_I$.
The terminality of $T_1$ tells us that there is only one possible choice for $f : I \times X \to T_1$.
Thus $(I, T_1)$ is the terminal in the fibre $S(T)_I$. 

Now we must show that reindexing preserves terminal objects. Let $u : I \to J$ be an arrow in $\mathbb B$,
and $u^* : S(T)_J \to S(T)_I$ some reindexing functor associated with $u$. We must show that each
$(I,X) \in s(T)_I$ has a unique arrow $(1_I,f) : (I,Y) \to u^*(J,T_1)$, i.e., that there exists a unique
arrow $f : I \times X \to T_1'$ where $u^*(J,T_1) \doteq (I,T_1')$.

Since $T_1$ is terminal in $\mathbb B$, there exists a unique arrow of signature $h : I \times T_1' \to T_1$ in $\mathbb B$.
Then we also have an arrow $(u,h) : (I, X) \to (J,T_1)$ in $s(T)$. By the Cartesianness of $\overline{u^*}$, there
is a unique arrow $(1_u,g) : (I,X) \to (I,T_1')$ such that $(1_u,g);\overline{u^*}(I,T_1) = (u,h)$. But the latter
equation is a vacouous constraint, since it equates two arrows to a terminal object (for the second component).
Thus, there is a unique arrow $(1_u,g) : (I,X) \to (I,T_1')$ \emph{period}, and reindexing preserves terminal objects.
\\~\\
Conversely, assume that $\vrt{s(T)}{\mathbb B}$ has a fibred terminal object. 
Let $1$ be the terminal object of $\mathbb B$. Let the terminal object of $s(T)_1$ be $(1,X)$, so that 
for any element $(1,Y)$ of $s(T)_1$, we have a unique arrow $(id_1, f) : (1,Y) \to (1,X)$. Thus
for any $Y \in T$ there is a unique arrow $1 \times Y \to X$, or alternatively, because $1 \times Y \cong Y$,
there is a unique arrow $Y \overset{!^{X}_Y}{\longrightarrow} X$. In particular, we can choose $Y \doteq X$, which tells us that there is a
unique arrow $1_X : X \to X$. We proceed to show that $X \cong 1$, demonstrating that some element of $T$ (namely $X$)
is a terminal object.

Since $(\mathbb B, T)$ is non-trivial, there is a $Z \in T$ and an $h : 1 \to Z$. Then, we have an arrow $1 \overset{h;!_Z^{X}}{\longrightarrow} X$.
Furthermore, since $1$ is a terminal object, we have an arrow $X \overset{!_X}{\longrightarrow} 1$.
Now, $(h;!_Z^X);!_X : 1 \to 1$ is equal to $id_1 : 1 \to 1$ since there is exactly one arrow from any object to $1$.
Furthermore, $!_x;(h;!_Z^X) : X \to X$ is equal to $id_X : X \to X$, since we've established in the previous paragraph
that there exists a unique arrow from $X$ to $X$. Thus $!_X$ and $h;!_Z^X$ form an iso pair, and so $1 \cong X$.

\subsection*{1.8.3}

\subsubsection*{(i)}

As a refresher, a category with finite products $(1,\times)$ and coproducts $(0,+)$ has distributive coproducts
if the canonical arrow $[(\kappa \times id),(\kappa' \times id)] : (X \times Z) + (Y \times Z) \to (X + Y) \times Z$ 
is an isomorphism.\\~\\

$(\Rightarrow)$

Let $\mathbb C$ be a category with Cartesian products. Suppose that it also has distributive coproducts.

First, $s(\mathbb C)$ has fibred Cartesian products. Letting $I \in \mathbb C$, for any $(I,X),(I,Y) \in s(\mathbb C)_I$
we have the fibred product $(I, X \times Y) \in s(\mathbb C)_I$ with $\pi_{X,Y} \doteq (id_I, \pi'_{I,(X \times Y)};\pi_{X,Y})$ and $\pi'_{X,Y} \doteq (id_I, \pi'_{I,(X \times Y)};\pi'_{X,Y})$, where the inner $\pi$s and $\pi'$s correspond to projections in $\mathbb C$ rather than $s(\mathbb C)$.
Furthermore, for $f : (I,Z) \to (I,X) \doteq (id_I, f')$ and $g : (I,Z) \to (I,Y) \doteq (id_I,g')$ we have
$\langle f, g \rangle : (I,Z) \to (I, X \times Y) \doteq (id_I, \langle f',g'\rangle)$.
 
Likewise, we have the fibred coproduct $(I, X + Y) \in s(\mathbb C)_I$ with 
$\kappa_{X,Y} \doteq (id_I, \pi'_{I,X};\kappa_{X,Y})$ and $\kappa'_{X,Y} \doteq (id_I, \pi'_{I,Y};\kappa'_{X,Y})$.
For each $f : (I, X) \to (I,Z) \doteq (id_I, f')$ and $g : (I,Y) \to (I,Z) \doteq (id_I, g')$,
let $\alpha^{-1} : I \times (X + Y) \to (I \times X) + (I \times Y)$ be the inverse of the canonical map
$[(id \times \kappa_{X,Y}),(id \times \kappa'_{X,Y})]$ in $\mathbb C$,
guaranteed by distributivity.
Letting $[f,g] : (I, X + Y) \to (I , Z)$ be defined as $[f,g] \doteq (id_I, \alpha^{-1};[f', g'])$ gives us our
coproduct UMP, for \\~\\
$\kappa_{X,Y};[f,g]$\\ 
$= (id_I, \langle \pi_{I,X}, \pi'_{I,X};\kappa_{X,Y} \rangle;\alpha^{-1};[f',g'])$\\
$= (id_I, \langle \pi_{I,X}, \pi'_{I,X} \rangle;(id_I \times \kappa_{X,Y});\alpha^{-1};[f',g'])$\\
$= (id_I, id_{I \times X};(id_I \times \kappa_{X,Y});\alpha^{-1};[f',g'])$\\
$= (id_I, id_{I \times X};\kappa_{(I \times X),(I \times Y)};[f',g'])$\\
$= (id_I, f')$\\
$= f$

Equating $(id_I, id_{I \times X};(id_I \times \kappa_{X,Y});\alpha^{-1};[f',g'])$ to 
$(id_I, id_{I \times X};\kappa_{(I \times X),(I \times Y)};[f',g'])$ is a bit subtle, but apparent from the following diagram

\begin{tikzcd}[sep = 90]
I \times X \ar[dr, "id \times \kappa_{X,Y}~~~~~~~" below] \ar[r, "\kappa_{(I \times X) \text{,} (I \times Y)}"] & (I \times X) + (I \times Y) \ar[d, "\text{[} (id \times \kappa_{X \text{,} Y}) \text{,} (id \times \kappa'_{X \text{,} Y})\text{]}", bend left = 20] \\
 & I \times (X + Y) \ar[u, "\alpha^{-1}", bend left = 20]
\end{tikzcd}

TODO: prove uniqueness\\
TODO: show that coproducts in $s(\mathbb C)$ are distributive.~\\

Interesting that this would not have worked without the guarantee that coproducts in $\mathbb C$ are $\mbf{distributive}$.

%\ar[r, "\kappa_{(I \times X) \text{,} (I \times Y)}"]
%

$(\Leftarrow)$

Suppose that $\mathbb C$ is a category with Cartesian products and that $s(\mathbb B)$ has fibred products
and distributive coproducts. Since $s(\mathbb C)_1 \cong \mathbb C$, we automatically know that $\mathbb C$ 
has distributive coproducts.

\subsubsection*{(ii)}

TODO: Can't figure it out.

Having pullbacks guarantees that a category has binary products. But does it guarantee a terminal object?
Doesn't seem like it.

\subsection*{1.8.4}

We first need to show how products in $\mbf{Fib}(\mathbb B)$ work. Given fibrations $\vrt{\mathbb E}{\mathbb B}p$ and
$\vrt{\mathbb D}{\mathbb B}q$, their product is $\vrt{\mathbb E \times_{\mathbb B} \mathbb D}{\mathbb B}(p \times q)$, 
where $\mathbb E \times_{\mathbb B} \mathbb D$ is the pullback along $p$ and $q$. TODO: prove this is a product.\\~\\

$(\Leftarrow)$~\\

Now, the diagonal $\Delta : p \to p \times p$ is defined as $\Delta(X) \doteq (X,X)$, an element 
of $\mathbb E \times_{\mathbb B} \mathbb E$ since $pX = pX$, and likewise $\Delta(f) \doteq (f,f)$.
Consider a hypothetical fibred right adjoint $U$ to $\Delta$:

\begin{mathpar}

\mprset{fraction={===}}
\inferrule
  {\Delta X \longrightarrow (Y,Z)}
  {X \longrightarrow U(Y,Z)}
\end{mathpar}

Arrows of the top signature have the form $(f : X \to Y,g : X \to Z)$ where there exists a $u : pX \to (pY = pZ) \in \mathbb B_1$ such
that $pf = pg = u$. Arrows of the bottom signature have the form $f : X \to U(Y,Z)$, where $U(Y,Z)$ is some object of $\mathbb E$ over $pY$.

We claim that $U(Y,Z)$ is the fibred product of $Y$ and $Z$ in $\mathbb E_{pY}$. We must show both that it is a product and 
that fibred products are preserved by reindexing. First, to see that it is a product, choose some $X$ with $pX = pY$. 
Examining the counit side of the adjunction makes it clear that $U(Y,Z)$ is a product of $Y$ and $Z$ in $\mathbb E_{pY}$:

\begin{center}
\begin{tikzcd}
 X \ar[r, "f", dashed] & U(Y,Z) \\ 
 & \\
 \Delta X \ar[r, "(g_1\text{,}g_2)" above] \ar[dr, "\Delta f~~" below] & (Y,Z) \\
  & \Delta U(Y,Z) \ar[u, "\epsilon_{Y\text{,}Z} = (\pi_{Y\text{,}Z}\text{,} \pi'_{Y\text{,}Z})" right]
\end{tikzcd}
\end{center}

For each pair of vertical arrows $g_1 : X \to Y$ and $g_2 : X \to Z$ in $\mathbb E \times_{\mathbb B} \mathbb E$ 
over $id_{pX}$, we have a unique arrow $f : X \to U(Y,Z)$ 
(vertical, since transposition of fibred adjunctions preserves base arrows) in 
$\mathbb E$ such making the lower triangle commute.
But that's just the UMP of products: 
$$(f,f);(\pi,\pi') = (g_1,g_2) \Leftrightarrow (f;\pi = g_1 \wedge f;\pi' = g_2)$$

By the functorality of reindexing, the above diagram is preserved by reindexing functors, and therefore
so are the fibred products. **Actually, no. Functorality of reindexing does not guarantee that the uniqueness part is 
preserved. 

$(\Rightarrow)$~\\ 

Suppose that a fibration $\vrt{\mathbb E}{\mathbb B}p$ has fibred Cartesian products. 
This is direction is not as interesting; we can just use fibred cartesian products to define the adjoint.
 
\section*{Lemma 1.9.2}

In the proof of this lemma, on pg 10075, it says ``Beck-Chevalley holds, by an easy calculation''.
Since I'm not fluent in the Beck-Chevalley condition yet, I'm going to try to carry out this ``easy calculation''.

First, note that while $\Pi_{(I,J)}$ clearly isn't a fibred functor, it does need to preserve Cartesian arrows
for Beck-Chevalley to make sense. More precisely, we need the ``non-fibred'' version of $\Pi_{(I,J)}$ (let's
call it $\Pi_{(-,J)}$) to preserve Cartesian arrows. $\Pi_{(-,J)}$ is not a functor and does not transform between categories;
instead, for all $I,K \in \mbf{Sets}$,  $\Pi_{(-,J)}$ transforms objects of the category $Fam(\mathbb C)_{I \times J}$ 
objects of $Fam(\mathbb C)_{I}$, and each arrow $f : X \to Y$ where $X \in Fam(\mathbb C)_{I \times J}$ and $Y \in Fam(\mathbb C)_{K \times J}$ to an arrow of type $\Pi_{(-,J)} X \to \Pi_{(-,J)} Y$. 

Let $u : I \to K$ be a function between sets. Then our canonical natural transformation 
$$u^* (Y \mapsto \Pi_{j \in J} Y_{(k,j)}) \Longrightarrow (Y \mapsto \Pi_{j \in J} Y_{i,j}) (u \times id)^*$$
arises as the transpose of

$$\pi^*_{I,J} u^* (Y_{K \times J} \mapsto \Pi_{j \in J} Y_{(k,j)}) \overset{\cong}{\longrightarrow} \pi^*_{I,J} (Y_{I \times J} \mapsto \Pi_{j \in J} Y_{(i,j)}) (u \times id)^* \overset{\epsilon (u \times id)^*}{\longrightarrow} (u \times id)^*$$

%$$ u^* \overset{u^* \eta}{\longrightarrow} u^* (Y \mapsto \Pi_{j \in J} Y_{i,j}) \pi^*_{I,J} 
%   \overset{\cong}{\longrightarrow} (Y \mapsto \Pi_{j \in J} Y_{k,j}) (u \times id)^* \pi^*_{I,J}$$

\section{Lemma 1.9.5}

We give the full proof of Lemma 1.9.5, sketched on pg 10077. Without loss of generality, we only cover products.

First, define the functor $\Pi_u$'s mapping on arrows as $\Pi_u( (f_i : X_i \to Y_i)_{i \in I} ) \doteq (\times   \{ f_i \mid u(i) = j \})_{j \in J}$

Next, we need to show that $\Pi_u$ is the right adjoint to $u^*$. We need a bijection of the following form:

\begin{center}
\mprset{fraction={===}}
\inferrule
  {(X_j)_{j \in J} \longrightarrow \Pi_u (Y_i)_{i \in I} }
  {u^*( (X_j)_{j \in J} ) \longrightarrow (Y_i)_{i \in I}}
\end{center}~\\
which is essentially a bijection of the following form:~\\
\begin{center}
\mprset{fraction={===}}
\inferrule
  {(X_j)_{j \in J} \longrightarrow (\Pi~\{ Y_i \mid u(i) = j\})_{j \in J}}
  {(X_{u(i)})_{i \in I} \longrightarrow (Y_i)_{i \in I}}
\end{center}

This bijection clearly exists. We must show that it underlies an adjunction $u^* \vdash \Pi_u$.

\begin{center}
\begin{tikzcd}[sep=70]
(\Pi \{ X_j \mid u(i) = j \})_{j \in J}  
  \ar[r,"(\times \{ f_i \mid u(i) = j \})_{j \in J}"] & 
  (\Pi \{ Y_i \mid u(i) = j \} )_{j \in J} \\
(X_j)_{j \in J} \ar[ur, "~~~~~~(g_j)_{j \in J}" below] \ar[u, "(\Pi_{(i \mid u(i) = j)} id_{X_j})_{j \in J}" left] \\
(X_{u(i)})_{i \in I} \ar[r, "(f_i)_{i \in I}"] \ar[dr, "(g_{u(i)})_{i \in I}~~~~~" below] & (Y_i)_{i \in I} \\
 & (\Pi \{ Y_{i'} \mid u(i') = u(i) \} )_{i \in I} \ar[u, "(\pi_i)_{i \in I}" right]  
\end{tikzcd}
\end{center}

TODO: we must also show Beck-Chevalley

\section*{Exercises 2.1}

\subsection*{2.1.1}

(i) We proceed by induction on $m$.\\~\\
\begin{description}

\item[Base case: $m = 0$]~\\

Trivial.

\item[Inductive step: $m > 0$]~\\

\begin{prooftree}

\AxiomC{$\Gamma,v_n : \sigma, \Delta, v_{n + (m-1)} : \iota, v_{n + m} : \rho \vdash M : \tau$}
\RightLabel{E}
\UnaryInfC{$\Gamma,v_n : \sigma, \Delta, v_{n + (m-1)} : \rho, v_{n + m} : \iota \vdash M[v_{n+(m-1)} / v_{n + m}, v_{n + m} / v_{n + (m-1)}] : \tau$}
\UnaryInfC{$\Gamma,v_n : \rho, \Delta, v_{n + (m-1)} : \sigma, v_{n+m} : \iota \vdash M[v_{n} / v_{n + m}, v_{n+m}/v_{n+(m-1)}, v_{n+(m-1)}/v_{n}] : \tau$}
\RightLabel{E}
\UnaryInfC{$\Gamma,v_n : \rho, \Delta, v_{n+(m-1)} : \iota, v_{n + m} : \sigma \vdash M[v_{n} / v_{n + m}, v_{n+m} / v_{n}]$}
\end{prooftree}

The middle inference is an application of the inductive hypothesis.(note: the result of section 2.1.3 (i) is used here) \\~\\
(ii) We proceed by induction on $m$.\\~\\
TODO:

\end{description}

\section*{Exercises 2.2}

\subsection*{2.2.1}

(i) Let $\mbf{Sets}_{\bullet}$ be the category of pointed sets as described in Exercises 1.2.3 (ii), 
at the top of page 10018. 
It can be seen as the category of sets and partial functions (where undefined arguments map to the point). 
Show that $\mbf{Sets}_{\bullet}$ has finite products.\\~\\
\underline{Terminal object}. A singleton set $(\{ \ast \},\ast)$ whose sole element $\ast$ is its point
is a terminal object in $\mbf{Sets}_{\bullet}$, because for any pointed set $(X,x)$ there is only
a single point-preserving mapping from $X$ to $\{ \ast \}$.\\~\\
\underline{Binary products}. Let $(X,x)$ and $(Y,y)$ be pointed sets. We claim that their product
is $(X \times Y, (x,y))$. Its projections have underlying functions $\pi : X \times Y \to X$ 
and $\pi' : X \times Y \to Y$, which clearly preserve points. We now demonstrate the UMP of products.
Let $(Z,z)$ be a pointed set and $f : (Z,z) \to (X,x)$ and $g : (Z,z) \to (Y,y)$ be arrows of $\mbf{Sets}_{\bullet}$.
(We write $|f|$ and $|g|$ for the functions underlying $f$ and $g$.) There exists an arrow 
$\langle f, g \rangle : (Z,z) \to (X \times Y, (x,y))$ such that $\langle f, g \rangle;\pi = f$ and 
$\langle f,g \rangle;\pi' = g$ whose underlying functions is defined as $a \mapsto (|f|(a), |g|(a))$.
It preserves points, since $z \mapsto (|f|(z), |g|(z)) = (x,y)$. Furthermore, $|\langle f, g \rangle;\pi|(a) = 
f(a)$ and $|\langle f,g \rangle;\pi'|(a) = g(a)$.\\~\\
(ii) The partial algebra maps each type of $\Sigma$ to a pointed set. The point is used for partiality, 
and thus should not be used for ``standard'' values; for example, a natural numbers type $\mathbb N$
would map to the pointed set $(\{ 0,1,2,\ldots \} \cup \{ \bullet \}, \bullet)$. Importantly, 
the point $\bullet$ is disjoint from the actual natural numbers. A term (i.e. an arrow in $C \ell(\Sigma)$)
is mapped to a point-preserving function. The domain of this function is in general
the product of several pointed sets. It's not clear how such a function is defined where some of the components
of the argument are points and others aren't.

\subsection*{2.2.2}

NOTE: They have a typo in the definition of $\Gamma^k$: it should say $v_{k-1} : \sigma_{k-1}$ rather than $v_k : \sigma_k$.\\~\\
Consider the interpretation of $\mathcal M$ associated with a morphism of signatures $\phi : \Sigma \to \mbf{Sign}(\mathbb B)$
in the proof og Theorem 2.2.5 on page 10109. \\~\\
(i) Let $\Gamma = (v_1 : \sigma_1, \ldots, v_n : \sigma_n)$ be a context with a term $\Gamma \vdash M : \tau$ such that
the variable $v_k$ in $\Gamma$ does not occur (free) in $M$. Prove that 

$$ \mathcal M(\Gamma \vdash M : \tau) = \mathcal M(\Gamma^k \vdash M^k : \tau) \circ \langle \pi_1, \ldots, \pi_{k-1},\pi_{k+1}, \ldots, \pi_n \rangle $$
where
$$ \Gamma^k \doteq v_1 : \sigma_1, \ldots, v_{k-1} : \sigma_{k-1}, v_k : \sigma_{k+1}, \ldots, v_{n-1} : \sigma_n $$
and
$$ M^k \doteq M[v_k/v_{k+1},\ldots,v_{n-1}/v_n]$$~\\

The proof proceeds by induction on the structure of $\Gamma \vdash M : \tau$

\begin{description}
\item[Case $v_1 : \sigma \vdash v_1 : \sigma$:]~\\
This case is vacuous, since every variable occurs in the term.
\item[Case $\Gamma \vdash F(M_1,\ldots,M_m) : \tau_{m+1}$:]~\\
Here $F : \tau_1, \ldots, \tau_m \to \tau_{m+1}$. 

By the IH we have $\Gamma^k \vdash M_i^k : \tau_i$ for 
$i \in 1..m$, and furthermore we have $\mathcal M(\Gamma \vdash M_i : \tau_i) = \mathcal M(\Gamma^k \vdash M_i^k : \tau_i)
\circ \langle \pi_1, \ldots, \pi_{k-1},\pi_{k+1},\ldots,\pi_n \rangle$.\\~\\
Hence we have\\~\\
$\mathcal M(\Gamma \vdash F(M_1,\ldots,M_m) : \tau_{m+1})$ \\
\begin{tabular}{lll}
$= \phi(F)$ & $\circ$ & $\langle \mathcal M(\Gamma \vdash M_1 : \tau_1), \ldots, \mathcal M(\Gamma \vdash M_m : \tau_m) \rangle$ \\
$= \phi(F)$ & $\circ$ & $\langle \mathcal 
  M(\Gamma^k \vdash M_1^k : \tau_1) \circ \langle \pi_1,\ldots,\pi_{k-1},\pi_{k+1},\ldots,\pi_n\rangle$ \\
 & & $\ldots$ \\
 & & $\mathcal M(\Gamma^k \vdash M_m^k : \tau_m) \circ \langle \pi_1,\ldots,\pi_{k-1},\pi_{k+1},\ldots,\pi_n \rangle \rangle$ \\
$= \phi(F)$ & $\circ$ & $\langle \mathcal M(\Gamma^k \vdash M_1^k : \tau_1), \ldots, \mathcal M(\Gamma^k \vdash M_m^k : \tau_m) \rangle$ \\
 & $\circ$ & $\langle \pi_1, \ldots, \pi_{k-1}, \pi_{k+1}, \ldots, \pi_n \rangle$
\end{tabular}\\
$= \mathcal M(\Gamma^k \vdash F(M_1^k,\ldots, M_m^k) : \tau_{m+1}) \circ \langle \pi_1, \ldots, \pi_{k-1}, \pi_{k+1}, \ldots, \pi_n \rangle$ \\
$= \mathcal M(\Gamma^k \vdash F(M_1,\ldots, M_m)^k : \tau_{m+1}) \circ \langle \pi_1, \ldots, \pi_{k-1}, \pi_{k+1}, \ldots, \pi_n \rangle$

\item[Case $\mbf{weakening}$:]~\\

By the IH, we have $\Gamma^k \vdash M^k : \tau$ and 
$\mathcal M(\Gamma \vdash M : \tau) = \mathcal M(\Gamma^k \vdash M^k : \tau) \circ \langle \pi_1, \ldots, \pi_{k-1},\pi_{k+1},\ldots,\pi_n \rangle$ for $1 \leq k \leq n$

\begin{description}

\item[$k = n+1$:]~\\
Here we have $(\Gamma,v_{n+1}:\sigma_{n+1})^k = \Gamma$ and $M^k = M$.
Then $\mathcal M(\Gamma,v_{n+1} : \sigma_{n+1} \vdash M : \tau) = \mathcal M(\Gamma \vdash M : \tau) \circ \pi
= \mathcal M((\Gamma,v_{n+1}:\sigma_{n+1})^k \vdash M^k : \tau) \circ \pi 
= \mathcal M((\Gamma,v_{n+1}:\sigma_{n+1})^k \vdash M^k : \tau) \circ \langle \pi_1, \ldots, \pi_{k-1} (= \pi_{n}) \rangle$.

\item[$k < n+1$:]~\\
Here we have $\mathcal M(\Gamma,v_{n+1} : \sigma_{n+1} \vdash M : \tau) = \mathcal{M}(\Gamma \vdash M : \tau) \circ \pi
= \mathcal{M}(\Gamma^{k} \vdash M^{k} : \tau) \circ \langle \pi_1, \ldots, \pi_{k-1},\pi_{k+1},\ldots,\pi_n \rangle \circ \pi
= \mathcal{M}(\Gamma^{k} \vdash M^{k} : \tau) \circ \pi \circ \langle \pi_1, \ldots, \pi_{k-1},\pi_{k+1},\ldots,\pi_n,\pi_{n+1} \rangle = \mathcal M( (\Gamma, v_{n+1})^k \vdash M^k : \tau) \circ \langle \pi_1, \ldots, \pi_{k-1},\pi_{k+1},\ldots,\pi_n,\pi_{n+1} \rangle$ 

\end{description}

\item[Case $\mbf{contraction}$:]~\\

By the IH we have for all $0 \leq k \leq n + 1$, $(\Gamma,v_{n} : \sigma, v_{n+1} : \sigma)^k \vdash M^k : \tau$
and 
$\mathcal M(\Gamma,v_n:\sigma,v_{n+1}:\sigma \vdash M : \tau) 
= \mathcal M((\Gamma,v_n:\sigma,v_{n+1}:\sigma)^k \vdash M^k : \sigma) \circ \langle \pi_1, \ldots, \pi_{k-1},\pi_{k+1},\ldots,\pi_n \rangle$.



\begin{description}

\item[Case $k = n$:]~\\
 

This case suggests that our IH must be generalized: rather than removing a single variable from context, we should
allow the removal of a set of variables (all of which do not occur in the term).\\~\\

Here $M[v_n/v_{n+1}]^k = M$, and neither $v_n$ nor $v_{n+1}$ occurs in $M$.
Additionally, $(\Gamma,v_n:\sigma,v_{n+1}:\sigma)^k = \Gamma,v_n:\sigma$\\~\\
$\mathcal M(\Gamma,v_n : \sigma \vdash M[v_n/v_{n+1}] : \tau) = \mathcal M(\Gamma,v_n : \sigma,v_{n+1} : \sigma \vdash M : \tau)
 \circ \langle id, \pi' \rangle = \mathcal M((\Gamma,v_n:\sigma,v_{n+1}:\sigma)^{n+1} \vdash M^k : \tau) \circ \langle \pi_1, \ldots, \pi_{n-1},\pi_{n+1} \rangle \circ \langle id, \pi' \rangle = \mathcal M(\Gamma,v_n : \sigma \vdash M : \tau) \circ 
\langle \pi_1, \ldots, \pi_{n-1},\pi_{n+1} \rangle \circ \langle id, \pi' \rangle = \mathcal M(\Gamma,v_n : \sigma \vdash M : \tau)
\circ a$.



\item[Case $k < n$:]~\\


\end{description}

\end{description}




\subsection*{2.2.4}

Let $\phi : \Sigma \to \Sigma'$ be a morphism of signatures and let $(\Sigma', \mathbb A', \mathcal M')$
be a $\mbf{C-Model}$ of signature $\Sigma'$.\\~\\

Then we define 
$\phi^*(\Sigma', \mathbb A', \mathcal M') \doteq (\Sigma, \mathbb A', \mathcal M)$, 
where $\mathcal M(\sigma_1, \ldots, \sigma_n) \doteq \mathcal M'(\phi \sigma_1) \times \ldots \times \mathcal M'(\phi \sigma_n) 
$ and $\mathcal M(\Gamma \vdash M_1 : \tau_1, \ldots, \Gamma \vdash M_n : \tau_n) \doteq 
\mathcal M(\phi M_1, \ldots, \phi M_n)$.\\~\\
Further, we define $\overline{\phi^*}(\Sigma', \mathbb A', \mathcal M') : (\Sigma,\mathbb A', \mathcal M) \to (\Sigma, \mathbb A', \mathcal M')$ as... 



\subsection*{2.2.5}


\section*{Exercses 3.4}

\subsection*{3.4.1}

We write $I,J,K,X,Y$ for objects of $\mathbb B$ and $u,v,f,g$ for arrows.\\~\\
(i) Extend the assignment $(I,X) \mapsto \delta(I,X) = \langle id, \pi' \rangle : I \times X \to (I \times X) \times X$
to a functor $\delta : s(\mathbb B) \to \mathbb B^{\to}$.\\~\\
We map an arrow $(u,f) : (I,X) \to (J,Y)$ to the following commutative square:
\begin{center}
\begin{tikzcd}[sep=70.0]
I \times X \ar[r, "\langle \pi;u\text{,} f \rangle"] \ar[d,"\langle id\text{,} \pi' \rangle" left] & 
  \ar[d,"\langle id \text{,} \pi' \rangle"] J \times Y \\
(I \times X) \times X \ar[r, "\langle \pi;\langle \pi;u \text{,} f \rangle \text{,} \langle \pi;\pi \text{,} \pi' \rangle; f \rangle" below] & (J \times Y) \times Y 
\end{tikzcd}
\end{center}
It clearly commutes. To show that this mapping is functorial, we must show that it preserves identities and composition.
An identity in $s(\mathbb B)$ has the form $(id, \pi')$. This maps to the following square:
\begin{center}
\begin{tikzcd}[sep=80.0]
I \times X \ar[r, "\langle \pi \text{,} \pi' \rangle"] \ar[d,"\langle id \text{,} \pi' \rangle" left] & 
  \ar[d,"\langle id \text{,} \pi' \rangle"] I \times X \\
(I \times X) \times X \ar[r, "\langle \pi;\langle \pi \text{,} \pi' \rangle \text{,} \langle \pi;\pi \text{,} \pi' \rangle; \pi' \rangle" below] & (I \times X) \times X 
\end{tikzcd}
\end{center}
The bottom edge simplifies to $id_{(I \times X) \times X}$, and so identities are preserved.
Now consider a the composition $(u;v, \langle \pi;u , f \rangle; g)$ of $(u,f) : (I,X) \to (J, Y)$ 
and $(v,g) : (J, Y) \to (K,Z)$. The functor $\delta$ maps this composite to the commutative square whose 
top and bottom edges are:\\~\\
Top: $~~~~~\langle \pi;u;v~~,~~\langle \pi;u , f \rangle; g \rangle$\\
Bottom:\\$\langle \pi; \langle \pi;u;v, \langle \pi;u , f \rangle; g \rangle, \langle \pi;\pi, \pi' \rangle; \langle \pi;u, f \rangle; g \rangle$\\
$= \langle \pi; \langle \pi;u;v, \langle \pi;u , f \rangle; g \rangle, \langle \pi;\pi;u, \langle \pi;\pi, \pi' \rangle; f \rangle; g \rangle$\\~\\
On the other hand, mapping first and then composing in $\mathbb B^{\to}$ gives:\\~\\
Top: $\langle \langle \pi;u, f \rangle; \pi; v, \langle \pi;u, f \rangle;g \rangle$\\
Bottom: $\langle \pi;\langle \pi;u , f \rangle, \langle \pi;\pi, \pi' \rangle; f \rangle$;
$\langle \pi;\langle \pi;v , g \rangle, \langle \pi;\pi, \pi' \rangle; g \rangle$\\~\\
which simplifies to: \\~\\
Top: $\langle \pi;u;v , \langle \pi;u, f \rangle;g \rangle$\\
Bottom:\\$\langle \pi; \langle \pi;u, f \rangle; \langle \pi;v, g \rangle, \langle \pi;\langle\pi;u, f \rangle;\pi, 
\langle \pi;\pi, \pi' \rangle;f \rangle;g \rangle$\\
$= \langle \pi; \langle \pi;u;v, \langle \pi;u, f \rangle;g \rangle, \langle \pi;\pi;u, 
\langle \pi;\pi, \pi' \rangle;f \rangle;g \rangle$
\\~\\
Hence composition is preserved.\\~\\
(ii) Show that $\delta$ sends Cartesian morphisms (for the simple fibration on $\mathbb B$) to pullback squares in 
$\mathbb B$ (i.e. Cartesian morphisms for the codomain functor on $\mathbb B$).\\~\\
A quick sketch on paper shows that Cartesian arrows in $s(\mathbb B)$ over $U : I \to J \in \mathbb B$ 
have the form $(u,\pi')$. Instantiating the first square in part (i) gives:
\begin{center}
\begin{tikzcd}[sep=70.0]
I \times X \ar[r, "\langle \pi;u\text{,} \pi' \rangle"] \ar[d,"\langle id\text{,} \pi' \rangle" left] & 
  \ar[d,"\langle id \text{,} \pi' \rangle"] J \times Y \\
(I \times X) \times X \ar[r, "\langle \pi;\langle \pi;u \text{,} \pi' \rangle \text{,} \langle \pi;\pi \text{,} \pi' \rangle; \pi' \rangle" below] & (J \times Y) \times Y 
\end{tikzcd}
\end{center}
which simplifies to
\begin{center}
\begin{tikzcd}[sep=70.0]
I \times X \ar[r, "\langle \pi;u\text{,} \pi' \rangle"] \ar[d,"\langle id\text{,} \pi' \rangle" left] & 
  \ar[d,"\langle id \text{,} \pi' \rangle"] J \times X \\
(I \times X) \times X \ar[r, "\langle \pi;\langle \pi;u \text{,} \pi' \rangle \text{,} \pi' \rangle" below] & (J \times X) \times X 
\end{tikzcd}
\end{center}
We then have the following pullback diagram

\begin{center}
\begin{tikzcd}[sep=70.0]
Z  \ar[rrd, "f"] \ar[ddr, "g~~" below] \ar[dr, "~\langle g;\pi \text{,} f;\pi' \rangle" right, dashed] & & \\
& I \times X \ar[r, "\langle \pi;u\text{,} \pi' \rangle"] \ar[d,"\langle id\text{,} \pi' \rangle" left] & 
  \ar[d,"\langle id \text{,} \pi' \rangle"] J \times X \\
 & (I \times X) \times X \ar[r, "\langle \pi;\langle \pi;u \text{,} \pi' \rangle \text{,} \pi' \rangle" below] & (J \times X) \times X 
\end{tikzcd}
\end{center}
Note this makes use of the assumption that 
$$f;\langle id, \pi' \rangle = \langle f, f;\pi' \rangle = \langle g;\pi;\langle \pi;u, \pi' \rangle, g;\pi' \rangle
= g;\langle \pi; \langle \pi;u \text{,} \pi' \rangle \text{,} \pi' \rangle$$
(I.e. the pullback assumption that the outer quadrilateral commutes.) 

\subsection*{3.4.2}~\\
Reminder:\\~\\
As described on pg 4147, a finite product preserving functor is said to be \emph{Cartesian closed} when
the transpose of
$$ F(A \Rightarrow B) \times F(A) \overset{\cong}{\longrightarrow} F((A \Rightarrow B) \times A) \overset{F(ev)}{\longrightarrow} FB$$
is an isomorphism $\lambda(F(ev) \circ \cong) : F(A \Rightarrow B) \to (FA \Rightarrow FB)$.\\~\\
Assume $\vrt{\mathbb E}{\mathbb B}p$ is a fibred $CCC$ with equality, i.e., in addition to having fibred 
equality $Eq \dashv \delta^*$, substitution functors are Cartesian closed.\\~\\
(i) Show that the Frobenius property for equality (canonical $Eq(\delta^*(X) \times Y) \longrightarrow X \times Eq(Y)$ has an inverse) holds automatically.\\~\\
The aforementioned canonical map arises as the transpose of 
$$\delta^*(X) \times Y \overset{\delta^*(X) \times \eta_Y}{\longrightarrow} \delta^*(X) \times \delta^*Eq(Y) 
  \overset{rapl}{\longrightarrow} \delta^*(X \times Eq(Y)) $$

%% right adjoints preserve limits: is \delta^*'s preservation of products stronger than vanilla rapl?
Ugh... can't figure it out. Didn't try to use the Beck-Chavalley condition for equality: maybe that would help?

\subsection*{3.4.4}

The canonical transformation of def 3.4.1 is easy: it just uses the $u^*v^*(X) \cong (u;v)^*(X)$ property for 
substitution functors. The canonical transformation of 3.4.7 is tricky.

We have $\gamma_1 : KI \times KJ \to K(I \times J)$ and $\gamma_2 : K(I \times J) \times KJ \to K((I \times J) \times J)$, and the following Beck-Chevalley diagram.
\begin{center}
\begin{tikzcd}
\mathbb E_{I \times J} \ar[r, "Eq_{I,J}" above, bend left = 20] \ar[d, "L" left]
& \mathbb E_{(I \times J) \times J} \ar[d, "L" right] \ar[l, "\delta^*_{I,J}" below, bend left = 20] \\
\mathbb D_{K(I \times J)} \ar[d, "\gamma_1^*" left] & \mathbb D_{K((I \times J) \times J)} \ar[d, "((\gamma_1 \times id);\gamma_2)^*"] \\
\mathbb D_{KI \times KJ} \ar[r, "Eq_{KI,KJ}" above, bend left = 20] & \ar[l, "\delta^*_{KI,KJ}" below, bend left = 20] \mathbb D_{(KI \times KJ) \times KJ}
\end{tikzcd}
\end{center}
We must demonstrate a natural isomorphism $\gamma_1^*L\delta_{I,J}^* \Rightarrow \delta_{KI,KJ}^*((\gamma_1 \times id);\gamma_2)^*L$. This isomorphism comes from the unique 
commuting arrows between the following two Cartesian arrows.
\begin{center}
\begin{tikzcd}
\gamma_1^*L\delta_{I,J}^*(A) \ar[r, "\overline{\gamma_1^*}\cdots"] \ar[dd,"? (natural)", dotted] & L\delta_{I,J}^*(A) \ar[rd, " ~~~~ L\overline{\delta^*}(A)" above] & \\
                                            &                          & LA \\
\delta_{KI,KJ}^*((\gamma_1 \times id);\gamma_2)^*L(A) \ar[r, "\overline{\delta^*}_{KI,KJ}\cdots" below] & ((\gamma_1 \times id);\gamma_2)^*L(A) \ar[ur, "~~~~~~~~~~~~\overline{((\gamma_1 \times id);\gamma_2)^*}\cdots" below] &
\end{tikzcd}
\end{center}
Now, these unique commuting arrows only exist if the base arrows along the top edge ($\gamma_1;K\delta$) factor through the base arrows along the bottom edge ($\delta;((\gamma_1 \times id);\gamma_2)$). We're in luck, because not only does the former 
factor through the latter, but the two arrows are exactly equal.\\~\\
To show this, we're going to need a lemma
\begin{lemma}
Let $K : \mathbb A \to \mathbb B$ be a product-preserving functor. Let $\gamma_{A,B} : KA \times KB \to K(A \times B)$ be 
the inverse of the canonical transformation $\langle K\pi_{A,B}, K\pi'_{A,B} \rangle : K(A \times B) \to KA \times KB$.
Then for any $f : Z \to A$ and $g : Z \to B$ we have $\langle Kf, Kg \rangle;\gamma_{A,B} = K\langle f, g \rangle$.
\end{lemma}

\begin{proof}
We abbreviate $\pi_{A,B}$ with $\pi$, $\pi'_{A,B}$ with $\pi'$, and $\gamma_{A,B}$ with $\gamma$ for convenience.\\~\\
First, we have $K\langle f,g \rangle;\langle K\pi, K\pi' \rangle = \langle Kf, Kg \rangle$ because:
\begin{center}
\begin{tabular}{lll}
$(K\langle f,g \rangle;\langle K\pi, K\pi' \rangle)$ & $=$ & (by ``sliding in'') \\
$\langle K \langle f,g \rangle;K\pi, K\langle f,g \rangle;K\pi' \rangle$ & $=$ & (by functoriality of $K$) \\
$\langle K(\langle f,g \rangle;\pi), K(\langle f,g \rangle;\pi') \rangle$ & $=$ & \\
$\langle K(\langle f,g \rangle;\pi), K(\langle f,g \rangle;\pi') \rangle$ & $=$ & \\
$\langle Kf, Kg \rangle$ & &
\end{tabular}
\end{center}
Next, we compose both sides of this equation by $\gamma$ on the right to get:
$$K\langle f,g \rangle = \langle Kf, Kg \rangle;\gamma$$
\begin{tabular}{ll}
\end{tabular}
\end{proof}~\\~\\
$\delta;((\gamma_1 \times id);\gamma_2)$\\
$= \langle id, \pi'_{KI,KJ} \rangle;((\gamma_1 \times id);\gamma_2)$\\
$= \langle \gamma_1, \pi' \rangle;\gamma_2$\\
$= \langle \gamma_1, \gamma_1;\langle K \pi, K \pi' \rangle;\pi' \rangle;\gamma_2$\\
$= \gamma_1;\langle id, K \pi' \rangle;\gamma_2$\\
$= \gamma_1;\langle K(id), K\pi' \rangle;\gamma_2$\\
$= \gamma_1;K\langle id, \pi' \rangle$\\
$= \gamma_1;K\delta$\\~\\
Note that the lemma was applied to obtain the second-to-last line from the third-to-last line.

 
\subsection*{3.4.6}

Let $\vrt{\mathbb E}{\mathbb B}p \overset{(K,L)}{\longrightarrow} \vrt{\mathbb E'}{\mathbb B'}p'$ be a morphism between
fibrations $p$ and $p'$ with fibred terminal object and equality.\\~\\
(i) Assume $(K,L)$ preserves the terminal object and equality. Verify that for parallel arrows $u,v$ in $\mathbb B$
the canonical vertical map 
$$Eq'(Ku,Kv) \longrightarrow L(Eq(u,v))$$
is an isomorphism.

 


\end{document}
